\documentclass[11pt]{article}
\input{\string~/.macros}
\usepackage[a4paper, total={6in, 8in}, margin=0.7in]{geometry}
\usepackage{graphicx}
\usepackage{url}
\usepackage{bm}
\usepackage{hyperref}
\hypersetup{colorlinks=true, linktoc=all, linkcolor=blue}
\newcommand{\linkbook}[3][../../a_first_coures_in_numerical_analysis_of_differential_equations.pdf]{
    \noindent\href[page=#2]{#1}{\urlstyle{rm}{#3}}
}

\newcommand{\heading}[1]{(#1)}
\newcommand{\bheading}[1]{\textbf{(#1)}}


\renewcommand\bf{\ensuremath{\bm{f}}}
\renewcommand\bx{\ensuremath{\bm{x}}}
\renewcommand\by{\ensuremath{\bm{y}}}
\renewcommand\be{\ensuremath{\bm{e}}}
\renewcommand\bv{\ensuremath{\bm{v}}}
\renewcommand\bz{\ensuremath{\bm{z}}}
\renewcommand\bu{\ensuremath{\bm{u}}}
\renewcommand\bb{\ensuremath{\bm{b}}}
\newcommand\bsY{\ensuremath{\boldsymbol{\mathcal{Y}}}}

\newcommand{\tu}{\tilde{u}}

\newcommand{\vD}{\boldsymbol{\varDelta}}

\newcommand{\shift}{\boldsymbol{\sE}}
\newcommand{\forw}{\boldsymbol{\varDelta}_+}
\newcommand{\back}{\boldsymbol{\varDelta}_-}
\newcommand{\cent}{\boldsymbol{\varDelta}_0}
\newcommand{\avg}{\boldsymbol{\varUpsilon}_0}
\newcommand{\diff}{\boldsymbol{\sD}}
\newcommand{\id}{\boldsymbol{\sI}}

\newcommand{\Dx}{\Delta x}
\newcommand{\Dy}{\Delta y}
\newcommand{\closure}[1]{cl\,#1}

\newcommand{\abs}[1]{\ensuremath{\left|#1\right|}}

\begin{document}

\linkbook{7}{table of contents}
\linkbook{21}{flow chart}
\tableofcontents

\newpage
\begin{center}
    Basics
\end{center}

\begin{enumerate}
    \item \bheading{Taylor Expansion} Given $f \in C^{\infty}(\R)$, the Taylor expansion of $f$ at $a$ is given by 
    \begin{align*}
        T(a)
        &= \sum_{n=0}^{\infty} \frac{f^{(n)}(a)}{n!} (x-a)^n \\
        &= f(a) + \frac{f'(a)}{1!}(x-a) + \frac{f''(a)}{2!} (x-a)^2 + \cdots
    \end{align*}
    \item \bheading{Taylor's Theorem} Let $k\geq 1$ and function $f:\R\to\R$ be $k$ times differentiable at a point $a\in \R$, then exists $h_k:\R\to\R$ such that 
    \[
        f(x)
        = \sum_{n=0}^{k} \frac{f^{(n)}(a)}{n!} (x-a)^n + h_k(x) (x-a)^k
    \]
    and $\lim_{x\to a} h_k(x) = 0$, i.e. the reminder term $R_k(x) = f(x) - P_k(x)$ is asymptotically trivial. If $f$ is $k+1$ times differentiable on the open interval and $f^k$ continous on the closed interval $[a,x]$, then the Lagrange remind is given by 
    \[
        R_k(x) = \frac{f^{k+1}(\zeta)}{(k+1)!} (x-a)^{k+1}
    \]
    for soem $\zeta \in [a,x]$ by the mean value theorem
    \item \bheading{$\sO$ notation} $f(x) = \sO(g(x))$   describes asymptotic behavior of function $f$
    \begin{enumerate}
        \item \heading{as $x\to \infty$} if there exists $M\geq 0$ and $x_0\in\R$ such that $|f(x)| \leq M g(x)$ for all $x>x_0$. 
        \item \heading{as $x\to a$} if there exists $M\geq 0$ and $\delta \in\R$ such that $|f(x)| \leq M g(x)$ when $0 < |x-a| < \delta$. Alternatively we can say 
        \[
            \lim_{x\to a} \sup \abs{\frac{f(x)}{g(x)}} < \infty
        \]
    \end{enumerate}
    \item \bheading{power series expansions}
    \begin{align*}
        \frac{1}{1-x} 
            &= \sum_{n=0}^{\infty} x^n = 1+x+x^2+x^3+\cdots \\
        e^x 
            &= \sum_{n=0}^{\infty} \frac{x^n}{n!} = 1+x+\frac{x^2}{2}+\frac{x^3}{3!}+\cdots \\
        (1+x)^{\alpha}
            &= \sum_{k=0}^{\infty} \binom{\alpha}{k} x^k \\
        \ln(1+x)
            &= \sum_{n=1}^{\infty} (-1)^{n+1} \frac{x^n}{n} = x - \frac{x^2}{2} + \frac{x^3}{3} \tag{convergent if $|x|<1$} \\
        \ln(1+x)
            &= -\sum_{n=1}^{\infty} \frac{x^n}{n} = - x - \frac{x^2}{2} - \frac{x^3}{3} \tag{convergent if $|x|<1$}
    \end{align*}
\end{enumerate}


\section{\linkbook{25}{Euler's Method and Beyond}}

\subsection{\linkbook{25}{Ordinary differential equations and Lipschitz condition}}


\begin{enumerate}
    \item \bheading{Goal} Approximate solution to 
    \[
        \by' = \bf ( t, \by)
        \quad \text{with initial condition} \quad 
        \by ( t_0 ) = \by_0    
    \]
    where $t > t_0$ and $\bf: [t_0, \infty) \times \R^d \to \R^d$ is a sufficiently well behaved function
    \item \bheading{Lipschitz condition} Given $\bf$ and norm $\norm{\cdot}$, the Lipschitz condition is defined by 
    \[
        \norm{
            \bf(t, \bx) - \bf(t, \by)
        } \leq \lambda \norm{\bx - \by}
        \quad \quad
        \text{for all}
        \quad \bx, \by \in \R^d \;,\; t> t_0
    \]
    where $\lambda \in \R$ is called Lipschitz constant. 
    \item \bheading{Picard Lindelof theorem} Consider initial value problem 
    \[
        y'(t) = f(t, y(t))
        \quad
        y(t_0) = y_0
    \]
    If $f$ is uniformly Lipschitz continous in $y$ and continous in $t$, then for some $\epsilon > 0$, there exists unique solution $y(t)$ to the initial value problem on the interval $[t_0 - \epsilon, t_0 + \epsilon]$ 
    \item \bheading{Analytic function} A function $\bf$ is an analytic function if it is a function that is locally given by a convergent power series, i.e. an infinitely differentiable function such that at any point $(t,\by_0) \in [0, \infty)\times \R^d$ in its domain, the Taylor series converges to $\bf(\bx)$ for $\bx$ in a neighborhood of $(t,\by_0)$.
    \begin{enumerate}
        \item \heading{example} polynomial, exponential, trigonometric, logarithm, power function
        \item \heading{note} if $\bf$ is analytic, solution $\by$ to the initial value problem is also analytic
    \end{enumerate}
\end{enumerate}


\subsection{\linkbook{26}{Euler's method}}

\begin{definition*}
    \bheading{Euler's Method} Given initial value problem $\by' = \bf(t, \by)$ for $t\geq t_0$ and initial value $\by(t_0) = \by_0$. If we assume $\by'(t) = \bf(t, \by(t)) \approx \bf(t_0, \by(t_0))$ for $t\in [t_0,t_0+h)$ (i.e. deriviative in $[t_n, t_{n+1}]$ is approximated by value of derivative at $t_n$) for some sufficiently small time step $h>0$, we can approximate the value of $\by(t)$ by
    \begin{align*}
        \by(t) 
        &= \by(t_0) + \int_{t_0}^t \bf(\tau, \by(\tau)) d\tau  \\
        &\approx \by_0 + (t-t_0) \bf(t_0, \by_0)
    \end{align*}
    Given a sequence of times $(t_n)_{n\in\N} = \p{t_0, t_0+h,\cdots}$ we have numerical approximation $(\by_n)_{n\in\N}$ by 
    \[
        \by_{n+1} = \by_n + h\bf(t_n, \by_n)
    \]
    \begin{enumerate}
        \item \heading{intuition} euler's method is a time-stepping numerical method that covers interval by an equidistant grid and produe numerical solution at the grid points. we can show that euler's method is convergent, i.e. as $h\to 0$, grid is refined, the numerical solution tends to exact solution
    \end{enumerate}
\end{definition*}

\begin{definition*}
    \bheading{convergent method} Given a time-stepping numerical method on a compact inteval $[t_0, t_0+t^*]$, we can compute numerical solutions dependent upon $h$
    \[
        \by_n = \by_{n,h}
        \quad \text{for}
        \quad n = 0,1,\cdots,\lfloor t^*/h\rfloor    
    \]
    A method is said to be convergent if for every ODE with Lipschitz function $\bf$, the numerical solution tends to the true solution as the grid becomes increasingly fine. More rigorously, if every ODE with Lipschitz function $\by$ and for every $t^*>0$, then following holds 
    \[
        \lim_{h\to 0^+} \max_{n=0,1,\cdots,\lfloor t*/h\rfloor    } \norm{\by_{n,h} - \by(t_n)} = 0
    \]
\end{definition*}

\begin{theorem*}
    \bheading{Euler's method is convergent} 
    \begin{proof}
        Assume $\bf$ and therefore also $\by$ is analytic, i.e. convergent Taylor expansion. Let $\be_{n,h} = \by_{n,h} - \by(t_n)$ be the numerical error. Show $\lim_{h\to 0^+} \max_n \norm{\be_{n,h}} = 0$. By Taylor's theorem 
        \[
            \by(t_{n+1})
            = \by(t_n) + h\by'(t_n) + \sO(h^2)
            = \by(t_n) + h\bf(t_n,\by(t_n)) + \sO(h^2)
        \]
        given $\by$ continously differentiable, $\sO(h^2)$ can be bounded uniformly for all $h>0$ by a term $ch^2$ for some $c>0$. Subtract previous from iterative formula of euler's method 
        \[
            \be_{n+1,h} = 
            \be_{n,h} + h\p{
                \bf(t_n, \by(t_n) + \be_{n,h}) - \bf(t_m, \by(t_n))
            } + \sO(h^2)
        \]
        By triangle inequality and Lipschitz condition 
        \begin{align*}
            \norm{\be_{n+1, h}}
            &\leq \norm{\be_{n,h}} + h\norm{\bf(t_n, \by(t_n) + \be_{n,h}) - \bf(t_n, \by(t_n))} + ch^2 \\
            &\leq (1+h\lambda) \norm{\be_{n,h}} + ch^2
        \end{align*}
        for $n=0,1,\cdots,\lfloor t^*/h \rfloor - 1$. By induction on $n$, we can show $\norm{\be_{n,h}} \leq \frac{c}{\lambda} h \p{(1+h\lambda)^n - 1}$. Since $1+h\lambda < e^{h\lambda}$ we have $(1+h\lambda)^n < e^{nh\lambda} < e^{\lfloor t*/h\rfloor h\lambda} \leq e^{t^* \lambda}$. Therefore 
        \[
            \norm{\be_{n,h}}   \leq \frac{c}{\lambda} (e^{t^*\lambda} - 1)h
        \]
        for $n=0,1,\cdots,\lfloor t*/h\rfloor$. This is an upper bound on the error that is independent of $h$, hence $\lim_{h\to 0} \norm{\be_{n,h}} = 0$. from which we can infer that error decays globally as $\sO(h)$
    \end{proof}
\end{theorem*}

\begin{definition*}
    \bheading{order $p$ method} Given arbitrary time-stepping method 
    \[
        \by_{n+1} = \bsY_n (\bf, h. \by_0, \by_1, \cdots, \by_n) \quad \quad n = 0,1,\cdots   
    \]
    for initial value problem, it is of order $p$ if 
    \[
        \by(t_{n+1}) - \bsY_n (\bf, h, \by(t_0), \by(t_1), \cdots, \by(t_n)) = \sO(h^{p+1})
    \]
    for every analytic $\bf$ and $n=0,1,\cdots$. Intuitively, a method is of order $p$ if it recovers exactly every polynomial oslution of degrees $p$ or less.
    \begin{enumerate}
        \item \heading{intuition} order of a method gives information about local behavior, i.e. advancing from $t_n$ to $t_{n+1}$ where $h>0$ is sufficiently small, we are incurring an error of $\sO(h^{p+1})$. Generally want the the global (convergence) behavior of the method instead.
        \item \bheading{fact} euler's method is order 1
        \begin{proof}
            Euler's method can be written as $\by_{n+1} - \p{\by_n + h\bf(t_n, \by_n)} = 0$. Replace $\by_k$ by $\by(t_k)$ and expand terms of Taylor series about $t_n$ we have 
            \[
                \by(t_{n+1}) - \p{\by(t_n) + h\bf(t_n, \by(t_n))}
                 = \p{
                     \by(t_n) + h\by'(t_n) + \sO(h^2)
                 } - \p{
                     \by(t_n) + h\by'(t_n)
                 } = \sO(h^2)
            \]
        \end{proof}
    \end{enumerate}
\end{definition*}



\subsection{\linkbook{30}{The trapezoidal rule}}


\begin{definition*}
    \bheading{Trapezoidal Rule} Instead of approximating derivative by a constant in $[t_n, t_{n+1}]$, namely by its value at $t_n$, the trapezoidal rule approximates the value of the derivate by average of values at the endpoints. We can approximate solution $\by(t)$ by 
    \begin{align*}
        \by(t) 
        &= \by(t_n) + \int_{t_n}^t \bf(\tau, \bf(\tau)) d\tau \\
        &\approx \by(t_n) + \frac{1}{2} (t-t_n) \p{\bf(t_n, \by(t_n)) + \bf(t, \by(t))}
    \end{align*}
    Given a sequence of times $(t_n)_{n\in\N} = \p{t_0, t_0+h,\cdots}$ we have numerical approximation $(\by_n)_{n\in\N}$ by 
    \[
        \by_{n+1} = \by_n + \frac{1}{2} h \p{
            \bf(t_n, \by_n) + \bf(t_{n+1}, \by_{n+1})
        }
    \]
    \begin{enumerate}
        \item \bheading{theorem} order of trapezoidal rule is 2
        \begin{proof}
            Compute by performing Taylor expansion on $\by(t_{n+1})$ and $\by'(t_{n+1})$ about $t_n$
            \[
                \by(t_{n+1}) - \pc{
                    \by(t_n) + \frac{1}{2}h \pc{
                        \bf(t_n, \by(t_n)) + \bf(t_{n+1}, \by(t_{n+1}))
                    }
                } = \sO(h^3)
            \]
        \end{proof}
        \item \bheading{theorem} trapezoidal rule is convergent
        \begin{proof}
            Detail of proof \linkbook{31}{here}. We can show error is bounded by 
            \[
                \norm{\be_{n,h}}\leq \frac{ch^2}{\lambda} exp \p{
                    \frac{t^* \lambda}{1 - \frac{1}{2}h\lambda}
                }
            \]
            from which we can infer that error decays globally as $\sO(h^2)$
        \end{proof}
        \item \heading{note} euler's method is explicit, since we can compute $\by_{n+1}$ with a few arithmetic operations by computing $\bf$, a function of a known $\by_n$. Trapezoidal rule is implicit, i.e. finding $\by_{n+1}$ is not trivial and $\bf$ is a function of both $\by_n$ and $\by_{n+1}$. We might need to solve a nonlinear equation of $\by_{n+1}$
        \[
            \by_{n+1} - \frac{1}{2}h \bf(t_{n+1}, \by_{n+1}) = \bv
        \]
        where $\bv = \by_n + \textstyle \frac{1}{2} h \bf(t_n, \by_n)$ can be evaluated easily from assumptions.
    \end{enumerate}
\end{definition*}



\subsection{\linkbook{35}{The theta method}}

\begin{definition*}
    \bheading{theta method} is a generalization of Euler's method ($\theta=1$) and the trapezoidal rule ($\theta = 1/2$), whereby the derivates are assumed to be piecewise constant and provided by a linear combination of derivatives at the endpoints of each interval. The numerical approximates are,
    \[
        \by_{n+1} = \by_n + h\p{
            \theta \bf(t_n, \by_n) + (1-\theta) \bf(t_{n+1}, \by_{n+1})
        }
        \quad \quad n = 0,1,\cdots
    \]
    for some fixed $\theta \in [0,1]$
    \begin{enumerate}
        \item \heading{fact} theta method is explicit for $\theta=1$ and implicit otherwise
        \item \bheading{theorem} theta method is of order 2 for $\theta=1/2$ and order 1 otherwise.
        \item \bheading{theorem} theta method is convergent for every $\theta \in [0,1]$
    \end{enumerate}
\end{definition*}



\section{\linkbook{41}{Multistep Method}}



\section{\linkbook{161}{8 Finite Differences Schemes}}

\subsection{\linkbook{161}{8.1 Finite differences}}



\newcommand{\shift}{\boldsymbol{\sE}}
\newcommand{\forw}{\boldsymbol{\varDelta}_+}
\newcommand{\back}{\boldsymbol{\varDelta}_-}
\newcommand{\cent}{\boldsymbol{\varDelta}_0}
\newcommand{\avg}{\boldsymbol{\varUpsilon}_0}
\newcommand{\diff}{\boldsymbol{\sD}}
\newcommand{\id}{\boldsymbol{\sI}}



\begin{enumerate}
    \item \bheading{Finite difference operators} Given real sequences $\bz = \pc{z_k}_{k\in\Z} = z(kh)$ for $k\in\Z$ as discrete sampling of a function $z$ for some $h>0$. Let $x_k = kh$. We can define finite difference operators mapping the space $\R^{\Z}$ of all such sequences to itself.
    \begin{align*}
        (\shift \bz)_k 
            &= z_{k+1} \tag{shift} \\
        (\forw \bz)_k  
            &= z_{k+1} - z_k \tag{forward difference} \\
        (\back \bz)_k 
            &= z_k - z_{k-1} \tag{backward difference} \\
        (\cent \bz)_k 
            &= z_{k+\frac{1}{2}} - z_{k - \frac{1}{2}} \tag{central difference} \\
        (\avg \bz)_k 
            &= \textstyle\frac{1}{2} (z_{k-\frac{1}{2}} + z_{k+\frac{1}{2}}) \tag{averaging}
    \end{align*}
    Finite difference operators are composed under function composition.
    \begin{enumerate}
        \item \bheading{fact} $\sT \in \pc{\shift, \forw, \back, \cent, \avg, \diff}$ are linear operators
        \[
            \sT(a\bw + b\bz) = a\sT\bw + b\sT\bz
            \quad \text{ for } \quad
            \bw,\bz\in\R^{\Z}\;,\;\; a,b\in\R    
        \]
        \item \bheading{convention} $\sT z_k$ stands for $(\sT \bz)_k$
    \end{enumerate} 
    \item \bheading{Differential operator} The goal is to approximate derivatives $\diff$ by expresing it with a linear combination of values along the grid.
    \begin{align*}
        (\diff\bz)_k &= z'(kh) \tag{differential}
    \end{align*}
    \item \bheading{Functions of operators} Finite difference operators are functions of $h$. Given an analytic function as Taylor series, $g(x) = \textstyle\sum_{j=0}^{\infty} a_j x^j$, we can expand $g$ about $\pc{\shift-\id, \avg - \id, \forw, \back, \cent, h\diff}$,
    \[
        g(\forw)\bz = \p{
            \sum_{j=0}^{\infty} a_j \forw^j
        } \bz = 
        \sum_{j=0}^{\infty} a_j (\forw^j \bz)
    \]
    \item \bheading{Asymptotics of operators}
    \[
        \pc{\shift-\id, \avg - \id, \forw, \back, \cent, h\diff} \overset{h\to 0+}{\longrightarrow} O
    \]
    \begin{enumerate}
        \item \bheading{example} 
        \[
            \forw z_k = z_{k+1} - z_k =  z(x_k + h) - z(x_k) = hz'(\eta_k) = \sO(h)
        \]
        by some $\eta_k \in [x_k, x_{k+1}]$ by mean value theorem
    \end{enumerate}
    \item \bheading{Operator $\shift^{1/2}$}
    \[
        (\shift^{1/2} \bz)_k = z((k+\frac{1}{2})h)
    \]
    by defining it with a power series expansion of $g(x) = \sqrt{1+x}$
    \[
        \shift^{1/2} = (\id + (\shift - \id))^{1/2} = \id + \sum_{j=0}^{\infty} \frac{ (-1)^{j-1} }{ 2^{2j - 1} } \frac{(2j - 2)!}{(j-1)!j!}  (\shift - \id)^j
    \]
    \item \bheading{Operator ommutativity} Idea is all operator can be expressed w.r.t. $\shift$. 
    \begin{align*}
        \forw 
            &= \shift - \id \\
        \back
            &= \id - \shift^{-1} \\ 
        \cent
            &= \shift^{1/2} - \shift^{-1/2} \\
        \avg
            &= \frac{1}{2} (\shift^{1/2} + \shift^{-1/2}) \\
        \id
            &= \shift^0
    \end{align*}
\end{enumerate}






\end{document}
