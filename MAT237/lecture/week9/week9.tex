\documentclass[11pt]{article}
\input{/Users/markwang/.preamble}
\begin{document}

\section*{Local Invertibility}

\subsection*{Implicit Function Theorem}

\begin{defn*}
  A \textbf{level set} of a real-valued function $f$ of $n$ variables is a set of the form
  \[
    L_c(f) = \{ (x_1, \cdots, x_n) | f(x_1, \cdots, x_n) = c\}
  \]
  That is, a set where the functino takes on a given constant value $c$.
\end{defn*}


\begin{defn*}
  The \textbf{level set} of a real-valued function $f$ of $n$ real variables is a set of the form
  \[
    L_{c}(f)=\left\{(x_{1},\cdots ,x_{n})\,\mid \,f(x_{1},\cdots ,x_{n})=c\right\}
  \]
that is, a set where the function takes on a given constant value c.
\end{defn*}

\begin{theorem*}
   If the function $f$ is differentiable, the gradient of $f$ at a point is either zero, or perpendicular to the level set of $f$ at that point.
\end{theorem*}


\begin{defn*}
  the \textbf{zero set} of a real valued function $f: X\to \R$ is the subset $f^{-1}(0)$ of $X$ (the inverse image of $\{ 0\}$)
\end{defn*}


\begin{defn*}
  A \textbf{hyperplane} is a subspace of one dimension less than its ambient space.
\end{defn*}

\begin{defn*}
  Let $f$ be a function whose domain is the set $X$ with image as $Y$. Then $f$ is \textbf{invertible} if there exists a function $g$ with domain $Y$ and image $X$ with property
  \[
    f(x) = y \iff g(y) = x
  \]
  \begin{rem}
    $f = x^2$ is not injective, so the function is not invertible.
  \end{rem}
\end{defn*}

\begin{defn*}
  \textbf{Horizontal Line Test} Let $f: \R \to \R$. If any horizontal line $y=c$ intersects the graph in more than one point, the function is not injective. The function $f$ is surjective (i.e., onto) if and only if its graph intersects \textit{any} horizontal line at least once.
\end{defn*}



\begin{defn*}
  The \textbf{inverse} of a square matrix $A$ is a matrix $A^{-1}$ such that $A A^{-1} = I$
\end{defn*}

$ $\\
 \textbf{Motivation} If one is given the zero locus of a $C^1$ function $F:\R^{n+k} \to \R^k$, that is,
 \[
  F(x_1, \cdots, x_n, y_1, \cdots, y_k) = 0
 \]
 and is asked to determine $y$ as a function of $x$. More precisely if there exists $f_i:\R^n \to\R$ such that $y_i = f(x_1, \cdots, x_n)$. Note here $y_i$ are placed after $x_i$ for convenience sake. The implicit functoin thoerem is a tool that allows relation to be converted to functions of several variables by representing the relation $f(x_1, \cdots, x_n) = (y_1, \cdots, y_k)$ as the graph of a function $F(x_1, \cdots, x_n, y_1, \cdots, y_k) = 0$


\begin{theorem*}
  \textbf{Implicit Function Theorem for scalar valued function (k=1)} If $F(x,y)$ is $C^1$ on some neighbourhood of $U\subseteq \R^{n+1}$ of the point $(a,b)\in\R^{n+1}$, $F(a,b) = 0$, and $\frac{\partial F}{\partial y}(a,b) \neq 0$, then there exists an $r\in \R_{\geq 0}$ together with a unique $C^1$ function $f: B_a(r) \to \R$ such that $F(x, f(x))=0$ for all $x\in B_a(r)$
  \begin{proof}
    We prove the theorem using the graph of $F$, specifically,
    \[
      G:\R^n \to \R^2 (x,y) \mapsto (x, F(x,y))
    \]
    So at point $(a,b)$ we have $G(a,b) = (a, 0)$. Note since $\frac{\partial F}{\partial y} (a,b) \neq 0$ we have
    \[
      DG(a,b) =
      \begin{bmatrix}
        1 & 0 \\
        \frac{\partial F}{\partial x} (a,b) &   \frac{\partial F}{\partial y} (a,b) \\
      \end{bmatrix}
    \]
    invertible. Therefore, $G$ is a bijection and hence $G^{-1}(x, 0)$ has unique point $(x, f(x))$

  \end{proof}
  \begin{rem}
    The function is not only existent but also differentiable in the neighbourhood
    \[
      \frac{\partial f}{\partial x_i} = - \frac{\frac{\partial F}{\partial x_i} (x_0, y_0)}{\frac{\partial F}{\partial y} (x_0, y_0)}
    \]
  \end{rem}
\end{theorem*}


\begin{corollary*}
   If $F\in C^1(\R^{n+1}, \R)$ satisfies $\nabla F\neq 0$, then for every $x_0\in S = \{ x: F(x) = 0\}$ there is a neighbourhood $N$ containing $x_0$ such that $S\cap N$ is the graph of a $C^1$ function.
   \begin{rem}
     Follows from the theorem because $\nabla F\neq 0$ means that at every point, one of the coponent $\partial_j F\neq 0$ We can apply the theorem to solve for $x_j$ in terms of the remaining variable.
   \end{rem}
\end{corollary*}

\begin{theorem*}
  \textbf{General Impicit Function Theorem} Let $F: \R^{n+k} \to\R^k$ be $C^1$ function, and write $(x_1, \cdots, x_n, y_1, \cdots, y_k)$ for the coordinates in $\R^{n+k}$. If $(a,b)$ satisfies $F(a,b) = 0$ and the Jacobian matrix  $J_{F,y} (a,b)= [\frac{\partial F_i}{\partial y_j}(a,b)]$ is invertible, there exists $r > 0$ and a unique $C^1$ function $f: U = B_r(a)\to\R^k$ such that for all $x\in B_r(a)$, $F(x, f(x)) = 0$

  \begin{rem}
    The partial derivative of arbitrary $f_k(x)$ can be obtained by diffeentiation the equation $F(x, f(x))=0$ with respect to $x_j$ to determine the result, but this does not clear things up as much as a simple example
  \end{rem}
\end{theorem*}


\begin{defn*}
  A bijection from the set $X$ to the set $Y$ has an inverse function from $Y$ to $X$
\end{defn*}

\begin{theorem*}
  \textbf{Inverse Function Theorem} Let $U,V\subseteq \R^n$ and fix some point $a\in U$. If $f: U\to V$ is of class $C^1$ and $Df(a)$ is invertible, then there exists some neighbourhoods $U' \subseteq U$ of $a$ and $V' \subseteq V$ of $f(a)$ such that $f|_{U'}: U' \to V'$ is bijective with $C^1$ inverse $(f|_{U})^{-1}: V' \to U'$. Moreover, if $b = f(a)$ then the derivative of the inverse map is given by
  \[
    [Df^{-1}](b) = [Df(a)]^{-1}
  \]
  \begin{rem}
    The inverse function theorem gives conditino for a function to be invertible in a neighborhood of a point. Informally, if $F$ is $C^1$ with invertible Jacobian matrix at a point $a$, then $F$ is invertible in a neighborhood of $a$. That is, an inverse function to $F$ exists in some neighborhood of $F(a)$. Moreover, the inverse function $F^{-1}$ is also $C^1$
  \end{rem}
  \begin{rem}
    Given $U,V\in\R$. Think of a point $a\in U$ such that $f'(a) = 0$. Here the curve in the neighborhood of $a$ is not injective, so inverse does not exist. Whereas if $f'(a) \neq 0$, $f$ is a bijection and inverse exists.
  \end{rem}
\end{theorem*}


\subsection{Parameterization}


\begin{defn*}
  The \textbf{graph} of a function $f$ is the collection of all ordered pairs $(x, f(x))$.
  \begin{rem}
    If the function input is scalar, the graph is 2-dimensinoal and for a continuous function is a \textbf{curve}. if the function is an ordered pair $(x_1, x_2)$, the graph is the collection of all ordered triples $(x_1, x_2, f(x_1, f_2))$ and for a continuous function is a \textbf{surface}. note that only injective function can be represented as a graph. And scalar-valued function are easily represented using graph
  \end{rem}
\end{defn*}


\begin{defn*}
  A \textbf{curve} is a map $\gamma : (a,b) \in \mathbb{I} \to \R^n$ such that $t \mapsto (f_1(t),\cdots, f_n(t))$. A curve is \textbf{simple} if $\gamma$ is injective; Geometrically, a simple curve is one that does not intersect with itself. A curve is \textbf{smooth} if
  \begin{enumerate}
    \item $\gamma\in C^1$
    \item $\gamma'(t) \neq 0$ for all $t\in (a,b)$
  \end{enumerate}
  \begin{rem}
    Intuitively, a smooth curve have no cornes. There are two ways it could have a corner
    \begin{enumerate}
      \item parameterization fail to be differentiable
      \item parameterization could slow to a stop, then start up again in a different direction.
    \end{enumerate}
  \end{rem}
\end{defn*}

\begin{theorem*}
  Let $\gamma: (a,b) \in \R \to \R^n : t \mapsto (f_i(t))$ be a parameterization such that $\gamma'(t_0) \neq 0$ where $t_0\in (a,b)$. Then in the neighborhood of $\gamma(t_0)$, $\gamma(a,b)$ can be written as the graph of a $C^1$ function $f$
  \[
    \gamma(t) = (\gamma(t), (f\circ \gamma)(t))
  \]
\end{theorem*}

\begin{defn*}
  A \textbf{Spherical Coordinate} is a coordinate system for 3-dimensional space where position of a point is specified by 3 numbers: the \textbf{radial distance $\tau$} from a fixed origin, its \textbf{polar angle $\phi$} measured from a fixed zenith direction, and the \textbf{azimuth angle $\theta$}  of its orthogonal projection on the reference plane that passes through the origin and is orthogonal to the zenith, measured from a fixed reference direction on that plane. Here we define $\varphi: \R^2 \to \R^3$ that maps spherical coordinates to cartesian coordinates. Specifically
  \[
    (\theta, \phi) \to (x,y,z) \text{ where } x = \tau \sin \phi cos \theta, y = \tau \sin \phi sin \theta, z = \tau \cos \phi
  \]
\end{defn*}





\end{document}
