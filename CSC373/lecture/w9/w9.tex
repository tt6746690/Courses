\documentclass[11pt]{article}
\input{/Users/markwang/.preamble}

\begin{document}

\subsection*{Simplex}

\begin{proposition*}    
    \textbf{Simplex}
    \begin{enumerate}
        \item If Simplex returns a result, it is a feasible solution 
        \item Simplex will terminate in at most $\binom{n+m}{m}$ steps (prevent cycling by choosing entering/leaving variable of least possible index)
        \item Simplex algorithm yields an optimal solution (proves that on termination, the maximization and minimization problem is tight on a single solution set)
    \end{enumerate}
\end{proposition*}

\begin{defn*}
    \textbf{} Given a \textbf{primal} version of LPP, 
    \begin{align*}
        \textbf{Maximize} \quad & c^T x \\
        \textbf{Subject to} \quad & Ax \leq b \\
                                & x \geq 0 \\
    \end{align*}
    The corresponding \textbf{dual} version is given by 
    \begin{align*}
        \textbf{Minimize} \quad & b^T y \\
        \textbf{Subject to} \quad & A^T y \geq c \\
                                & y \geq 0 \\
    \end{align*}
\end{defn*}



\begin{lemma*}
    Let $\overline{x}$ be any feasible solution to the primal and $\overline{y}$ be any feasible solution to the dual LPP. Then 
    \[
        \sum_{j=1}^n c_j \overline{x}_j \leq \sum_{i=1}^m b_i \overline{y}_i \quad \iff \quad c^T \overline{x} \leq b^T \overline{y}
    \]
    \begin{proof}
        \begin{align*}
            \sum_{j=1}^n c_j \overline{x}_j &\leq \sum_{j=1}^n \left( \sum_{i=1}^m a_{ij} \overline{y}_i \right) \overline{x}_j \\
            &= \sum_{i=1}^m \sum_{j=1}^n a_{ij} \overline{x}_j \overline{y}_i \\ 
            &\leq \sum_{i=1}^m b_i \overline{y}_i
        \end{align*}
    \end{proof}
\end{lemma*}

\begin{corollary*}
    Let $\overline{x}$ be a feasible solution to the primal and $\overline{y}$ be feasible solution to the dual such that 
    \[
        \sum_{j=1}^n c_j \overline{x}_j = \sum_{i=1}^m b_i \overline{y}_i
    \]
    Then $\overline{x}$ is an optimal solution to the primal LPP and $\overline{y}$ is an optimal solution to dual LPP
\end{corollary*}

\begin{theorem*}
    \textbf{LPP Duality} Assume simplex returns values $\overline{x }= ( \overline{x}_1, \cdots, \overline{x}_n)$ for the LPP $(A,b,c)$. Let $N$ be the nonbasic and $B$ be basic variables for the final slack form $(N, B, A', b', c', v')$. Define $\overline{y} = (\overline{y}_1,\cdots, \overline{y}_m)$ as follows,
     \[
        \overline{y}_i = 
        \begin{cases}
            c_{n+i} & \text{if } n + i \in N \\
            0 & \text{otherwise}            \\ 
        \end{cases}
     \]
     Then $\overline{x}$ is an optimal solution to the primal and $\overline{y}$ is an optimal solution to the dual, and 
     \[
        \sum_{j=1}^n c_j \overline{x}_j = \sum_{i=1}^m b_i \overline{y}_i \quad \iff \quad c^T x = b^T y
     \]
\end{theorem*}


\begin{algorithm}[H]
    \SetKwFunction{init}{Initialize-Simplex}
    \SetKwFunction{pivot}{Pivot}
    \SetKwFunction{simplex}{Simplex}
    \SetKwFunction{infeasible}{Infeasible}



    \Fn{$\init(A,b,c)$}{
        $k \leftarrow $ index of minimum $b_i$ \\
        \If{$b_k \geq 0$}{
            \Return{$(\{ 1,\cdots, n\}, \{ n+1, \cdots, n+m\}, A, b, c, 0)$}
        }

        $L_{aux} \leftarrow $ be $L$ by adding $x_0$ to LHS of each constant, \\
        \quad \quad and setting the objective function to $-x_0$ \\ 

        $(N,B,A,b,c,v) \leftarrow $ the resulting slack from $L_{aux}$ \\ 
        $l \leftarrow n + k$ \\
        $(N,B,A,b,c,v) = \pivot(N,B, A, b, c, l, 0)$ \quad \quad \text{//basic solution feasible} \\

        $x \leftarrow \simplex(N,B,A,b,c,v)$ \\

        \If{$\overline{x}_0 \neq 0$}{
            \Return{\infeasible}
        } \Else {
            \If{$\overline{x}_0$ is basic variable}{
                do another pivot to make it nonbasic \\ 
            }\Else{
                remove $\overline{x}_0$ from constraints \\ 
                restore original objective function of $L$, but replace each basic variable  \\ 
                \quad \quad in each objective function by RHS of its associated constraints \\ 
            }
        }
        \Return{Modified slack form} \\
    }
\end{algorithm}

\begin{defn*}
    During initialization, 
    \begin{enumerate}
        \item if LPP is infeasible, stops right away
        \item if LPP is feasible, but the basic solution is not feasible, transform LPP to another problem such that the basic solution is feasible, do simplex
        \begin{enumerate}
            \item Define $L_{aux}$ 
            \begin{align*}
                \textbf{Maximize} \quad & -x_0 \\
                \textbf{Subject to} \quad & x_1 - x_2 - x_0 \leq -5 \\
                                            & x_1 + x_2 - x_0  \leq 11\\
                                            & x_1, x_1, x_2 \geq 0 \\ 
            \end{align*}
            $L$ is feasible if and only if the optimal objective value of $L_{aux}$ is 0
        \end{enumerate}
        \item Otherwise, do simplex
    \end{enumerate}
\end{defn*}


\section*{NP-Completeness} 

\begin{defn*}
    \textbf{Polynomial-time Reducibility}
    Let $X$ and $Y$ be two problems. We say $Y$ is polynomial-time reducible to $X$ if arbitrary instances of problem $Y$ can be solved using a polynomial number of standard computational steps, plus a polynomial number of calls to a blackbox that solves problem $X$. We specify this as 
    \[
        Y \leq_p X
    \]
    $X$ is at least as hard as $Y$
    \begin{enumerate}
        \item Suppose $Y \leq_p X$, If $X$ can be solved in polynomial time, then so can be $Y$
        \item Suppose $Y \leq_p X$, If $Y$ cannot be solved in polynomial time, then $X$ cannot be solved in polynomial time either
        \item If $X$, $Y$ and $Z$ are 3 problems, and $Z\leq_p Y$ and $Y\leq_p X$, then $Z\leq_p X$
    \end{enumerate}
    If want to show $X\in NP$, we reduce $Y\leq_p X$ such that $Y\in NP$ 
\end{defn*}


\begin{example} \textbf{Independent set of vertex cover}
    \begin{enumerate}
        \item \textbf{Independence Set} \quad Given a set $S\subseteq V$ of nodes in graph $G = (V,E)$ is called independent if no two vertices in $S$ is connected in $G$
        \item \textbf{Independent Set problem} Find the independent set of maximum size
        \item \textbf{An equivalent decision problem} Given a graph $G$ and a number $k$, ask if there exists an independent set of size $k$.
        \item \textbf{Vertex cover} A set $S\subseteq V$ in $G = (V,E)$ is called a vertex cover if every $e\in E$ has at least one end in $S$ 
        \item \textbf{Vertex cover problem} Given a graph $G$, find a vertex cover of minimum size.
        \item \textbf{An equivalent decision problem} Given a graph $G$ and $k\in \R$, ask if there exists a vertex cover of size of at most $k$
    \end{enumerate}
    \begin{lemma*}
        Let $G = (V,E)$ be a graph, then $S\subseteq V$ is an independent set if and only if $V\setminus S$ is a vertex cover
        \begin{proof}
            \begin{enumerate}
                \item $=>$ \quad If $S$ is independent, want to show $V\setminus S$ is a vertex cover. Let $e = (u,v) \in E$. Assume $u,v\not\in V\setminus S$, then $u,v\in S$, then $S$ not independent
            \end{enumerate}
        \end{proof}
    \end{lemma*}

    \begin{corollary*}
        Independent set problem $\leq_p$ vertex cover problem. Similarly for vertex cover problem $\leq_p$ independent set problem (equally hard)
        \begin{proof}
            If we have a black box to solve vertex cover, then we can decide whether $G$ has an independent set of size at least $k$ by asking the blackbox if $G$ has a vertex cover of size at most $n-k$
        \end{proof}
    \end{corollary*}

\end{example}

\begin{example}
    \textbf{Set cover} Given a set $U$ of $n$ elements, a collection $S_1, S_2, \cdots, S_m \subseteq U$, and a number $k$, does there exists a collection of at most $k$ of these sets whose union is $U$
    Relation between vertex cover and set cover. For $i\in V$, define 
    \[
        S_i = \{ \text{edges going out of $i$} \}
    \]
    so $\cup S_i = E$ so $U = E$. 


    \begin{lemma*}
        Vertex cover $\leq_p$ Set cover
    \end{lemma*}
\end{example}

\begin{example}
    \textbf{Set packing} Given a set $U$ of $n$ elements, a collection $S_1, \cdots, S_,$ of subsets of $U$ and a number $k$< does there exists a collection of at least $k$ of these sets with the property that no two of them intersect?


    \begin{lemma*}
        Independent set $\leq_p$ set packing
    \end{lemma*}

\end{example}


\begin{defn*}
    \textbf{Problem} 
    \begin{enumerate}
        \item Solving a problem, i.e. finding a solution 
        \item Verification, i.e. checking if something is a solution
    \end{enumerate}
    \textbf{Formulation}
    \begin{enumerate}
        \item \textbf{Input} A binary string $s\in \{ 0, 1\}^*$, where 
        \[
            I = \{ 0, 1\}^* = \bigcup_{n\in N} \{ 0, 1\}^n
        \]
        \item \textbf{Decision problem} 
        \[
            X = \{ s\in I \,|\, X(s) = yes \}
        \]
        \item \textbf{Algo for decision problem} An algorithm $A$ for a decision problem is a function that takes a binary string input and outputs $yes$ or $no$. 
        \item \textbf{Solution} We say $A$ solves problem $X$ if for all strings $s\in I$, we have 
        \[
            A(s) = yes \quad \quad \text{if and only} \quad \quad  \text{if } s\in X
        \]
        \item \textbf{Polynomial-time solution} We say $A$ is a polynomial-time algorithm if there is a polynomial function $P$ such that for every input $s\in I$, the algorithm termiantes in at most $O(P(|s|))$ steps. 
        \[
            \mathbb{P} = \{X \,|\, \text{if there exists a polynomial-time algorithm $A$ that solves $x$} \}
        \]
    \end{enumerate}
\end{defn*}




\end{document}
