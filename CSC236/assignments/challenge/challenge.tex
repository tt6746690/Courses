
\documentclass[11pt]{article}

\input{/Users/markwang/.preamble}

\begin{document}


Consider language $L = \{ s\in \{ a,b\}: s \text{ has even } a \text{ and odd } b \}$. Devise an FSA for $L$ and then a RE for $L$. Provec that $L(R) = L$

\begin{solution}
  $ $\\
  We can construct DFSA $L$ for by,
  \begin{center}
    \begin{tikzpicture}[shorten >=1pt,node distance=3cm,on grid,auto]
     \node[state, initial] (q_0)   {$q_0$};
     \node[state, accepting] (q_1) [right=of q_0] {$q_1$};
     \node[state] (q_2) [below=of q_1] {$q_2$};
     \node[state] (q_3) [below=of q_0] {$q_3$};
      \path[->]
      (q_0) edge [bend left]  node {b} (q_1)
               edge [bend left]  node {a} (q_3)
      (q_1) edge [bend left]  node {a} (q_2)
               edge [bend left]  node {b} (q_0)
      (q_2) edge [bend left]  node {a} (q_1)
               edge [bend left]  node {b} (q_3)
      (q_3) edge [bend left]  node {a} (q_0)
               edge [bend left]  node {b} (q_2);
    \end{tikzpicture}
  \end{center}
  By state removal method we can construct a simpler FSA
  \begin{center}
    \begin{tikzpicture}[shorten >=2pt,node distance=3cm,on grid,auto]
     \node[state, initial] (q_0)   {$q_0$};
     \node[state, accepting] (q_1) [right=of q_0] {$q_1$};
     \node[state] (q_2) [below=of q_1] {$q_2$};
      \path[->]
      (q_0) edge [bend left]  node {b} (q_1)
               edge [left]  node {ab} (q_2)
               edge [loop above] node {aa} ()
      (q_1) edge [bend left]  node {a} (q_2)
               edge [bend left]  node {b} (q_0)
      (q_2) edge [bend left]  node {a} (q_1)
               edge [bend left]  node {ba} (q_0)
               edge [loop below] node {bb} ()
    \end{tikzpicture}
  \end{center}
  Then,
  \begin{center}
    \begin{tikzpicture}[shorten >=2pt,node distance=3cm,on grid,auto]
     \node[state, initial] (q_0)   {$q_0$};
     \node[state, accepting] (q_1) [right=of q_0] {$q_1$};
      \path[->]
      (q_0) edge [bend left]  node {b+ab(bb)*a} (q_1)
               edge [loop above] node {aa+ab(bb)*ba} ()
      (q_1) edge [bend left]  node {b+a(bb)*ba} (q_0)
               edge [loop above]  node {a(bb)*a} ()
    \end{tikzpicture}
  \end{center}
  Here we can derive the regular expression by going through the loop. A complete roundtrip from $q_0$ to $q_0$ via $q_1$ requires a regular expression of
  \[
    E = (aa+ab(bb)^*ba) + ((b+ab(bb)^*a)(a(bb)^*a)^*(b+a(bb)^*ba))
  \]
  This loop could be repeated many times before the string leads the automaton to its final $q_1$ accepting state via
  \[
    F = (b+ab(bb)^*a)(a(bb)^*a)^*
  \]
  So altogether the regular expression for $L$ derived from FSA is
  \[
    R = E^*F = ((aa+ab(bb)^*ba) + ((b+ab(bb)^*a)(a(bb)^*a)^*(b+a(bb)^*ba)))^*(b+ab(bb)^*a)(a(bb)^*a)^*
  \]

  Now we prove $L(R) = L$. Let $x\in \L(R)$. If $E$ is repeated zero times then $x\in L((b+ab(bb)^*a)(a(bb)^*a)^*) = (L(b) + L(ab(bb)^*a)) \circ L((a(bb)^*a)^*)$. So then either $x\in L(b)\circ L((a(bb)^*a)^*)$ or $x\in L(ab(bb)^*a)\circ L((a(bb)^*a)^*)$. Now consider $y\in L(a(bb)^*a)$, $y = a(bb)^ka$ for some $k\in \N$. We see that $y$ has two , therefore even number, of $a$'s; $y$ has even number of $b$'s by the exponentiation operation. ALso consider $z\in L(ab(bb)^ka)$, then $z$ has two therefore even number of $a$ and odd number of $b$. Since either $x = by^m$ or $x=zy^m$ for arbitrary $m\in\N$ We see that $x$ has even number of $a$ and odd number of $b$ in both cases. The same rationale can be applied to the case where $E$ is not an empty string, with the exact same result. Then $L(R)\subseteq L$\\

\end{solution}

% \begin{proof}
%   $ $\\
%   Here we denote $L(R)$ as the language described by $R$ over alpabet $\Sigma$. We show containment in both directions to prove equivalence, i.e.,
%   \[
%     L(R) = L \iff L(R)\subseteq L \land L\subseteq L(R)
%   \]
%   To prove $L(R)\subseteq L$, let $x\in L(R)$, then by definition of regular language,
%   \[
%     x\in L(a((a+b)(a+b))^*) \cup L(b(a+b)((a+b)(a+b))^*)
%   \]
%   \textbf{Case 1:} When $x\in L(a((a+b)(a+b))^*)$\\
%   Then $x \in L(a)\circ (L((a+b)(a+b))^*$. Let $y\in L(a) = \{ a\}$, $w\in L((a+b)(a+b) = \{ aa, ab, ba, bb\}$. Then $y=a$ and $w^*$ is just concatenation of strings in $\{ aa, ab, ba, bb\}$, where the length of any string in the set is 2. Here $\exists k\in \N, | w^* | = 2k$ as follows. Therefore $yw^*$ is of length $2k+1$, hence odd. Note that $x=yw^*$, then $x$ is a string that starts with $a$ and has odd length. We see that $x\in L$ as expected.\\
%   \textbf{Case 2:} When $x\in L(b(a+b)((a+b)(a+b))^*)$\\
%   Then $x\in L(a)\circ L(a+b)\circ L((a+b)(a+b))$. Let $y\in L(a) = \{ b\}$, $u\in L(a+b) = \{ a, b\}$, and $w\in L((a+b)(a+b)) = \{ aa, ab, ba, bb\}$. Then $y=b$, $u$ is either $a$ or $b$ so of length 1, and $w^*$ has length $2k$ for some $k$ as explained previously. Then $|yuw| = 1+1+2k = 2(1+k)$, hence even. Note $x=yuw$, therefore $x$ is a string that starts with $b$ and has even length. We see that $x\in L$ as expected. \\
%   Therefore, $x\in L(a((a+b)(a+b))^* + b(a+b)((a+b)(a+b))^*) \Rightarrow x\in L$, we proved that $L(R)\subseteq L$ \\
%
%   To prove $L\subseteq L(R)$, let $x\in L$. Then
%   \[
%     x\in \{s: s \text{ starts with a and has odd length}\}\cup \{s|s \text{ starts with b and has even length} \}
%   \]
%   \textbf{Case 1:} $x$ is a string that starts with $a$ and has odd length $2k+1$ for some $k\in\N$. Then the substring excluding prefix $a$ has even length $2k$. Let
%   \[
%     x=yw_1w_2 \dots w_k
%   \]
%   where $y=a$ and $w_i \in \{aa, ab, ba, bb\}$ for $1\leq i\leq k$. This construction is valid because $|y|=1$ and $| w_1w_2\dots w_k | = 2k$ and therefore $|x| = 1 + 2k$ is as wanted. Also we see that $y\in L(a)$ and $w_i\in L((a+b)(a+b))$. Because $x$ contains arbitrary $k$ number of $w_i$, then $w_1w_2 \dots w_k \in L((a+b)(a+b))^* = L(((a+b)(a+b))^*)$ Then $x \in L(a)\circ L(((a+b)(a+b))^*) = L(a((a+b)(a+b))^*)$. Therefore, $x\in L(a((a+b)(a+b))^*) \cup L(b(a+b)((a+b)(a+b))^*)$ as wanted.\\
%   \textbf{Case 2:} $x$ starts with $b$ and has even length, $2k+2$ for some $k\in\N$ (i.e. $x$ cannot be empty). Then the substring excluding prefix $b$ has odd length $2k+1$. let
%   \[
%     x=yuw_1w_2 \dots w_k
%   \]
%   where $y=a$, $u\in\{ a,b\}$, and $w_i\in \{ aa,ab,ba, bb\}$ for $1\leq i \leq k$. This construction is valid because $|y|=1$, $|u|=1$ and $|w_1w_2 \dots w_k| = 2k$ and therefore $|x| = 2+2k$ as wanted. Also note that $y\in L(a)$, $u\in L(a+b)$, $w_i \in L((a+b)(a+b))$. Because $x$ contains arbitrary number $k$ of $w_i$, then $w_1w_2 \dots w_k\in L((a+b)(a+b))^* = L(((a+b)(a+b))^*)$. Then by $x=yuw_1w_2 \dots w_k$, $x\in L(a)\circ L(a+b) \circ L(((a+b)(a+b))^*) = L(a(a+b)((a+b)(a+b))^*)$. Therefore $x\in L(a((a+b)(a+b))^*) \cup L(b(a+b)((a+b)(a+b))^*)$ as wanted. \\
%
%   Since $x\in L \Rightarrow x\in L(R)$, we proved that $L\subseteq L(R)$.\\
%
%   To conclude, because $L(R)\subseteq L \land L\subseteq L(R)$, then $L = L(R)$.
%
% \end{proof}
% \end{enumerate}
%

\end{document}
