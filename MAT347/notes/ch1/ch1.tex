\documentclass[11pt]{article}
\input{\string~/.macros.tex}
\input{\string~/.macros}
\usepackage[a4paper, total={6in, 8in}, margin=0.7in]{geometry}
\usepackage{graphicx}
\usepackage{url}
\usepackage{bm}
\usepackage{hyperref}
\hypersetup{colorlinks=true, linktoc=all, linkcolor=blue}
\newcommand{\linkbook}[3][../../a_first_coures_in_numerical_analysis_of_differential_equations.pdf]{
    \noindent\href[page=#2]{#1}{\urlstyle{rm}{#3}}
}

\newcommand{\heading}[1]{(#1)}
\newcommand{\bheading}[1]{\textbf{(#1)}}


\renewcommand\bf{\ensuremath{\bm{f}}}
\renewcommand\bx{\ensuremath{\bm{x}}}
\renewcommand\by{\ensuremath{\bm{y}}}
\renewcommand\be{\ensuremath{\bm{e}}}
\renewcommand\bv{\ensuremath{\bm{v}}}
\renewcommand\bz{\ensuremath{\bm{z}}}
\renewcommand\bu{\ensuremath{\bm{u}}}
\renewcommand\bb{\ensuremath{\bm{b}}}
\newcommand\bsY{\ensuremath{\boldsymbol{\mathcal{Y}}}}

\newcommand{\tu}{\tilde{u}}

\newcommand{\vD}{\boldsymbol{\varDelta}}

\newcommand{\shift}{\boldsymbol{\sE}}
\newcommand{\forw}{\boldsymbol{\varDelta}_+}
\newcommand{\back}{\boldsymbol{\varDelta}_-}
\newcommand{\cent}{\boldsymbol{\varDelta}_0}
\newcommand{\avg}{\boldsymbol{\varUpsilon}_0}
\newcommand{\diff}{\boldsymbol{\sD}}
\newcommand{\id}{\boldsymbol{\sI}}

\newcommand{\Dx}{\Delta x}
\newcommand{\Dy}{\Delta y}
\newcommand{\closure}[1]{cl\,#1}

\newcommand{\abs}[1]{\ensuremath{\left|#1\right|}}

\begin{document}
\begin{center}
    {\Huge Introduction to Groups}
\end{center}
\tableofcontents
\newpage


\section{\linkbook{29}{Basic Axioms and Examples}}


\begin{definition*} \bheading{Binary Operation}
    \begin{enumerate}
        \item \bheading{binary operation} $\star$ on a set $G$ is a function $\star:G\to G$. write $a\star b$ instead of $\star(a,b)$
        \item \bheading{associative $\star$} A binary operation on $G$ is associative if for all $a,b,c\in G$ $a\star (b\star c) = (a\star b) \star c$
        \item \bheading{commutative $\star$} A binary operation on $G$ is commutative if for all $a,b\in G$, $a\star b = b\star a$
        \item \bheading{closed under $\star$} $\star$ is a binary operation on $G$ and $H\subset H$, if $\star|_H$ is a binary operation on $H$, i.e. for all $a,b\in H$, $a\star b \in H$, then $H$ is closed under $\star$. Associativity/Commutativity of $\star$ is inherited on $H$
    \end{enumerate}
    \begin{itemize}
        \item \heading{examples}
        \begin{enumerate}
            \item $+$ on $\Z,\Q,\R,\C$ is a commutative binary operation
            \item $\times$ on $\Z,\Q,\R,\C$ is a commutative binary operation
            \item $-$ is not commutative on $\Z$ ($a-b \neq b-a$ usually)
            \item $-$ is not commutative on $\Z^+$ ($1,2\in\Z^+$, but $1-2 = -1 \not\in \Z^+$)
        \end{enumerate}
    \end{itemize}
\end{definition*}


\begin{definition*}
    \bheading{Group}
    \begin{enumerate}
        \item \bheading{group} A group is an ordered pair $(G,\star)$ where $G$ is a set and $\star$ is a binary operation on $G$ satisfying 
        \begin{enumerate}
            \item (associative) $\forall a,b,c\in G$, $(a\star b) \star c = a\star (b \star c)$
            \item (identity) $\exists e\in G \;\; \forall a\in G \;\;  a\star e = e \star a = a$ ($e$ is an identity of $G$)
            \item (inverse) $\forall a\in G \;\; \exists a^{-1}\in G$, $a\star a^{-1}  = a^{-1} \star a = e$ ($a^{-1}$ is an inverse of $a$)
        \end{enumerate}
        \item \bheading{abelian group} A group if abelian/commutative if $a\star b = b\star a$ for all $a,b\in G$ 
        \item \bheading{finite group} $G$ is a finite group if $G$ is a finite set
        \item \bheading{direct product} If $(A, \star)$ and $(B, \circ)$ are groups, a new group $A\times B$ called direct product are defined as 
        \[
            A\times B = \pc{\p{a,b} \mid a\in A \;\; b\in B}
        \]
        with binary operation defined component-wise 
        \[
            (a_1, b_1)(a_2, b_2) = (a_1 \star a_2, b_1 \circ b_2)    
        \]
    \end{enumerate}
    \begin{itemize}
        \item \heading{examples}
        \begin{itemize}
            \item $\Z,\Q,\R,\C$ are groups under $+$ ($e=0$, $a^{-1} = -a$, associativity by axioms of $+$)
            \item $\Q - \pc{0}, \R-\pc{0}, \C-\pc{0}, \Q^+, \R^+$ are gorups under $\times$ ($e = 1$, $a^{-1} = 1/a$, associativity by $\times$))
            \item $(\Z - \pc{0}, \times )$ is not a group ($2^{-1} = 1/2\not\in \Z - \pc{0}$)
            \item $(V,+)$ is an abelian group, where $V$ is a vector space (commutativity by axioms of a vector space)
            \item $(\integermodn, +)$ is an abelian group ($e = \overline{1}$, $a^{-1} = \overline{-a}$)
            \item $(\integermodnmul, \times)$ is abelian group ($e = \overline{1}$, $a^{-1}$ exists by definition of $\integermodnmul$)
        \end{itemize}
        \item \bheading{theorem} direct product of two groups is a group
        \item \bheading{proposition} identity/inverse are unique
        \begin{enumerate}
            \item identity of $G$ is unique
            \item inverse $a^{-1}$ of any $a$ in $G$ is unique
            \item $(a^{-1})^{-1} = a$ for all $a$ in $G$
        \end{enumerate}
    \end{itemize}
\end{definition*}





\end{document}
