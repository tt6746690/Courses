\documentclass[11pt]{article}
\input{\string~/.macros.tex}
\input{\string~/.macros}
\usepackage[a4paper, total={6in, 8in}, margin=0.7in]{geometry}
\usepackage{graphicx}
\usepackage{url}
\usepackage{bm}
\usepackage{hyperref}
\hypersetup{colorlinks=true, linktoc=all, linkcolor=blue}
\newcommand{\linkbook}[3][../../a_first_coures_in_numerical_analysis_of_differential_equations.pdf]{
    \noindent\href[page=#2]{#1}{\urlstyle{rm}{#3}}
}

\newcommand{\heading}[1]{(#1)}
\newcommand{\bheading}[1]{\textbf{(#1)}}


\renewcommand\bf{\ensuremath{\bm{f}}}
\renewcommand\bx{\ensuremath{\bm{x}}}
\renewcommand\by{\ensuremath{\bm{y}}}
\renewcommand\be{\ensuremath{\bm{e}}}
\renewcommand\bv{\ensuremath{\bm{v}}}
\renewcommand\bz{\ensuremath{\bm{z}}}
\renewcommand\bu{\ensuremath{\bm{u}}}
\renewcommand\bb{\ensuremath{\bm{b}}}
\newcommand\bsY{\ensuremath{\boldsymbol{\mathcal{Y}}}}

\newcommand{\tu}{\tilde{u}}

\newcommand{\vD}{\boldsymbol{\varDelta}}

\newcommand{\shift}{\boldsymbol{\sE}}
\newcommand{\forw}{\boldsymbol{\varDelta}_+}
\newcommand{\back}{\boldsymbol{\varDelta}_-}
\newcommand{\cent}{\boldsymbol{\varDelta}_0}
\newcommand{\avg}{\boldsymbol{\varUpsilon}_0}
\newcommand{\diff}{\boldsymbol{\sD}}
\newcommand{\id}{\boldsymbol{\sI}}

\newcommand{\Dx}{\Delta x}
\newcommand{\Dy}{\Delta y}
\newcommand{\closure}[1]{cl\,#1}

\newcommand{\abs}[1]{\ensuremath{\left|#1\right|}}

\begin{document}
\begin{center}
    {\Huge Chapter 2 Subgroups}
\end{center}
\tableofcontents
\newpage


\section{\linkbook{59}{Definition and Examples}}


\begin{definition*}
    \bheading{Subgroup}
    \begin{enumerate}
        \item \bheading{subgroup} Let $G$ be a group. The subset $H$ of $G$ is a subgroup of $G$, denoted as $H\leq G$ if
        \begin{enumerate}
            \item $H$ is nonempty
            \item $H$ is closed under products and inverses, i.e. $x,y\in G$ implies $x^{-1},xy\in H$
        \end{enumerate}
        If $H\leq G$ and $H\neq G$, then $H < G$. $H\leq G$ implies operation on $H$ is the operation on $G$ restricted to $H$. So any equation in $H$ can also be viewed as equation in $G$
        \item \bheading{The Subgroup Criterion} $H \subset G$ is a subgroup if and only if
        \begin{enumerate}
            \item $H\neq \emptyset$
            \item for all $x,y\in H$, $xy^{-1} \in H$
        \end{enumerate}
        Furthermore, if $H$ is finite, then suffice to check $H$ is nonempty and closed under multiplication
    \end{enumerate}
    \begin{itemize}
        \item \heading{examples}
        \begin{itemize}
            \item $G\leq G$ and $\pc{1}\leq G$ (latter is called the trivial subgroup)
            \item $\Z \leq \Q \leq \R$ under operation of addition 
            \item $\pc{1,r,r^2,\cdots,r^{n-1}} \leq D_{2n}$
            \item $2\Z \leq \Z$
            \item $(\Q^{\times}, \times) \not\leq (\R, +)$ (operation are different)
            \item $\Z^+ \leq \Z$ and $(\Z^+)^{\times} \not\leq \Q^{\times}$ (not closed under inverses and does not contain identity)
            \item $D_6 \not\leq D_8$ ($D_6 \not\subset D_8$)
        \end{itemize}
        \item \bheading{theorem} subgroup is a transitive relation, i.e. $K\leq H, H\leq G$, then $K\leq G$ 
    \end{itemize}
\end{definition*}
 


\section{\linkbook{62}{Centralizers and Normalizers, Stabilizers and Kernels}}



\end{document}
