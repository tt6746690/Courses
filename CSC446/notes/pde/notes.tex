\documentclass[11pt]{article}
\input{\string~/.macros}
\usepackage[a4paper, total={6in, 8in}, margin=0.7in]{geometry}
\usepackage{graphicx}
\usepackage{url}
\usepackage{bm}
\usepackage{hyperref}
\hypersetup{colorlinks=true, linktoc=all, linkcolor=blue}
\newcommand{\linkbook}[3][../../a_first_coures_in_numerical_analysis_of_differential_equations.pdf]{
    \noindent\href[page=#2]{#1}{\urlstyle{rm}{#3}}
}

\newcommand{\heading}[1]{(#1)}
\newcommand{\bheading}[1]{\textbf{(#1)}}


\renewcommand\bf{\ensuremath{\bm{f}}}
\renewcommand\bx{\ensuremath{\bm{x}}}
\renewcommand\by{\ensuremath{\bm{y}}}
\renewcommand\be{\ensuremath{\bm{e}}}
\renewcommand\bv{\ensuremath{\bm{v}}}
\renewcommand\bz{\ensuremath{\bm{z}}}
\renewcommand\bu{\ensuremath{\bm{u}}}
\renewcommand\bb{\ensuremath{\bm{b}}}
\newcommand\bsY{\ensuremath{\boldsymbol{\mathcal{Y}}}}

\newcommand{\tu}{\tilde{u}}

\newcommand{\vD}{\boldsymbol{\varDelta}}

\newcommand{\shift}{\boldsymbol{\sE}}
\newcommand{\forw}{\boldsymbol{\varDelta}_+}
\newcommand{\back}{\boldsymbol{\varDelta}_-}
\newcommand{\cent}{\boldsymbol{\varDelta}_0}
\newcommand{\avg}{\boldsymbol{\varUpsilon}_0}
\newcommand{\diff}{\boldsymbol{\sD}}
\newcommand{\id}{\boldsymbol{\sI}}

\newcommand{\Dx}{\Delta x}
\newcommand{\Dy}{\Delta y}
\newcommand{\closure}[1]{cl\,#1}

\newcommand{\abs}[1]{\ensuremath{\left|#1\right|}}

\graphicspath{{../assets/}}

\begin{document}

\linkbook{7}{table of contents}
\linkbook{21}{flow chart}
\tableofcontents

\newpage

% \begin{center}
%     Basics
% \end{center}

% \begin{enumerate}
%     \item \bheading{Taylor Expansion} Given $f \in C^{\infty}(\R)$, the Taylor expansion of $f$ at $a$ is given by 
%     \begin{align*}
%         T(a)
%         &= \sum_{n=0}^{\infty} \frac{f^{(n)}(a)}{n!} (x-a)^n \\
%         &= f(a) + \frac{f'(a)}{1!}(x-a) + \frac{f''(a)}{2!} (x-a)^2 + \cdots
%     \end{align*}
%     \item \bheading{Taylor's Theorem} Let $k\geq 1$ and function $f:\R\to\R$ be $k$ times differentiable at a point $a\in \R$, then exists $h_k:\R\to\R$ such that 
%     \[
%         f(x)
%         = \sum_{n=0}^{k} \frac{f^{(n)}(a)}{n!} (x-a)^n + h_k(x) (x-a)^k
%     \]
%     and $\lim_{x\to a} h_k(x) = 0$, i.e. the reminder term $R_k(x) = f(x) - P_k(x)$ is asymptotically trivial. If $f$ is $k+1$ times differentiable on the open interval and $f^k$ continous on the closed interval $[a,x]$, then the Lagrange remind is given by 
%     \[
%         R_k(x) = \frac{f^{k+1}(\zeta)}{(k+1)!} (x-a)^{k+1}
%     \]
%     for soem $\zeta \in [a,x]$ by the mean value theorem
%     \item \bheading{Taylor's Theorem for multivariable function} If $f:\R^n\to \R$ are $k$ times differentiable function at point $\ba\in\R^n$ then exists $h_{\alpha}: \R^n \to \R$ such that 
%     \[
%         f(\bx) = \sum_{|\alpha| \leq k} \frac{D^{\alpha}f(\ba)}{\alpha!}(\bx-\ba)^{\alpha} + \sum_{|\alpha|=k} h_{\alpha}(\bx) (\bx - \ba)^{\alpha}
%         \quad \quad \lim_{\bx \to \ba} h_{\alpha}(\bx) = 0
%     \]
%     where 
%     \[
%         |\alpha| = \alpha_1 + \cdots + \alpha_n
%         \quad \quad 
%         \alpha! = \alpha_1!\cdots \alpha_n!
%         \quad \quad
%         \bx^{\alpha} = x_1^{\alpha_1} \cdots x_n^{\alpha_n}    
%         \quad \quad 
%         D^{\alpha}f = \frac{\partial^{|\alpha|} f}{\partial x_1^{\alpha_1} \cdots x_n^{\alpha_n}}
%     \]
%     \item \bheading{$\sO$ notation} $f(x) = \sO(g(x))$   describes asymptotic behavior of function $f$
%     \begin{enumerate}
%         \item \heading{as $x\to \infty$} if there exists $M\geq 0$ and $x_0\in\R$ such that $|f(x)| \leq M g(x)$ for all $x>x_0$. 
%         \item \heading{as $x\to a$} if there exists $M\geq 0$ and $\delta \in\R$ such that $|f(x)| \leq M g(x)$ when $0 < |x-a| < \delta$. Alternatively we can say 
%         \[
%             \lim_{x\to a} \sup \abs{\frac{f(x)}{g(x)}} < \infty
%         \]
%     \end{enumerate}
%     \item \bheading{binomial theorem} 
%     \[
%         (x+y)^n = \sum_{k=0}^n  \binom{n}{k} x^{n-k} y^k
%     \]
%     \item \bheading{power series expansions}
%     \begin{align*}
%         \frac{1}{1-x} 
%             &= \sum_{n=0}^{\infty} x^n = 1+x+x^2+x^3+\cdots \\
%         e^x 
%             &= \sum_{n=0}^{\infty} \frac{x^n}{n!} = 1+x+\frac{x^2}{2}+\frac{x^3}{3!}+\cdots \\
%         (1+x)^{\alpha}
%             &= \sum_{k=0}^{\infty} \binom{\alpha}{k} x^k \\
%         \ln(1+x)
%             &= \sum_{n=1}^{\infty} (-1)^{n+1} \frac{x^n}{n} = x - \frac{x^2}{2} + \frac{x^3}{3} \tag{convergent if $|x|<1$} \\
%         \ln(1-x)
%             &= \sum_{n=1}^{\infty} \frac{x^n}{n} = x + \frac{x^2}{2} + \frac{x^3}{3} \tag{convergent if $|x|<1$} \\
%         \sin x
%             &= \sum_{n=0}^{\infty} \frac{(-1)^n}{(2n+1)!} x^{2n+1}
%             = x - \frac{1}{3!} x^3 + \frac{1}{5!} x^5 + \cdots \\
%         \sin^2 x 
%             &= \sum_{n=1}^{\infty} \frac{(-1)^{n+1} 2^{2n-1} x^{2n}}{(2n)!}
%             = \frac{2 x^2}{2!} - \frac{8x^4}{4!} + \frac{32 x^6}{6!} + 
%     \end{align*}
%     \item \bheading{trigonometric identities}
%     \begin{align*}
%         \sin (\theta + \phi) 
%             &= \sin(\theta)\cos(\phi) + \sin(\phi)\sin(\theta) \\
%         \sin (\theta + \phi) + \sin (\theta - \phi)
%             &= 2\sin (\theta) \cos(\phi) \\
%         1 - \cos(\theta) = 2\sin^2 \p{\frac{\theta}{2}}
%     \end{align*}
% \end{enumerate}


% \section{\linkbook{25}{Euler's Method and Beyond}}

% \subsection{\linkbook{25}{Ordinary differential equations and Lipschitz condition}}


% \begin{enumerate}
%     \item \bheading{Goal} Approximate solution to 
%     \[
%         \by' = \bf ( t, \by)
%         \quad \text{with initial condition} \quad 
%         \by ( t_0 ) = \by_0    
%     \]
%     where $t > t_0$ and $\bf: [t_0, \infty) \times \R^d \to \R^d$ is a sufficiently well behaved function
%     \item \bheading{Lipschitz condition} Given $\bf$ and norm $\norm{\cdot}$, the Lipschitz condition is defined by 
%     \[
%         \norm{
%             \bf(t, \bx) - \bf(t, \by)
%         } \leq \lambda \norm{\bx - \by}
%         \quad \quad
%         \text{for all}
%         \quad \bx, \by \in \R^d \;,\; t> t_0
%     \]
%     where $\lambda \in \R$ is called Lipschitz constant. 
%     \item \bheading{Picard Lindelof theorem} Consider initial value problem 
%     \[
%         y'(t) = f(t, y(t))
%         \quad
%         y(t_0) = y_0
%     \]
%     If $f$ is uniformly Lipschitz continous in $y$ and continous in $t$, then for some $\epsilon > 0$, there exists unique solution $y(t)$ to the initial value problem on the interval $[t_0 - \epsilon, t_0 + \epsilon]$ 
%     \item \bheading{Analytic function} A function $\bf$ is an analytic function if it is a function that is locally given by a convergent power series, i.e. an infinitely differentiable function such that at any point $(t,\by_0) \in [0, \infty)\times \R^d$ in its domain, the Taylor series converges to $\bf(\bx)$ for $\bx$ in a neighborhood of $(t,\by_0)$.
%     \begin{enumerate}
%         \item \heading{example} polynomial, exponential, trigonometric, logarithm, power function
%         \item \heading{note} if $\bf$ is analytic, solution $\by$ to the initial value problem is also analytic
%     \end{enumerate}
% \end{enumerate}


% \subsection{\linkbook{26}{Euler's method}}

% \begin{definition*}
%     \bheading{Euler's Method} Given initial value problem $\by' = \bf(t, \by)$ for $t\geq t_0$ and initial value $\by(t_0) = \by_0$. If we assume $\by'(t) = \bf(t, \by(t)) \approx \bf(t_0, \by(t_0))$ for $t\in [t_0,t_0+h)$ (i.e. deriviative in $[t_n, t_{n+1}]$ is approximated by value of derivative at $t_n$) for some sufficiently small time step $h>0$, we can approximate the value of $\by(t)$ by
%     \begin{align*}
%         \by(t) 
%         &= \by(t_0) + \int_{t_0}^t \bf(\tau, \by(\tau)) d\tau  \\
%         &\approx \by_0 + (t-t_0) \bf(t_0, \by_0)
%     \end{align*}
%     Given a sequence of times $(t_n)_{n\in\N} = \p{t_0, t_0+h,\cdots}$ we have numerical approximation $(\by_n)_{n\in\N}$ by 
%     \[
%         \by_{n+1} = \by_n + h\bf(t_n, \by_n)
%     \]
%     \begin{enumerate}
%         \item \heading{intuition} euler's method is a time-stepping numerical method that covers interval by an equidistant grid and produe numerical solution at the grid points. we can show that euler's method is convergent, i.e. as $h\to 0$, grid is refined, the numerical solution tends to exact solution
%     \end{enumerate}
% \end{definition*}

% \begin{definition*}
%     \bheading{convergent method} Given a time-stepping numerical method on a compact inteval $[t_0, t_0+t^*]$, we can compute numerical solutions dependent upon $h$
%     \[
%         \by_n = \by_{n,h}
%         \quad \text{for}
%         \quad n = 0,1,\cdots,\lfloor t^*/h\rfloor    
%     \]
%     A method is said to be convergent if for every ODE with Lipschitz function $\bf$, the numerical solution tends to the true solution as the grid becomes increasingly fine. More rigorously, if every ODE with Lipschitz function $\by$ and for every $t^*>0$, then following holds 
%     \[
%         \lim_{h\to 0^+} \max_{n=0,1,\cdots,\lfloor t*/h\rfloor    } \norm{\by_{n,h} - \by(t_n)} = 0
%     \]
% \end{definition*}

% \begin{theorem*}
%     \bheading{Euler's method is convergent} 
%     \begin{proof}
%         Assume $\bf$ and therefore also $\by$ is analytic, i.e. convergent Taylor expansion. Let $\be_{n,h} = \by_{n,h} - \by(t_n)$ be the numerical error. Show $\lim_{h\to 0^+} \max_n \norm{\be_{n,h}} = 0$. By Taylor's theorem 
%         \[
%             \by(t_{n+1})
%             = \by(t_n) + h\by'(t_n) + \sO(h^2)
%             = \by(t_n) + h\bf(t_n,\by(t_n)) + \sO(h^2)
%         \]
%         given $\by$ continously differentiable, $\sO(h^2)$ can be bounded uniformly for all $h>0$ by a term $ch^2$ for some $c>0$. Subtract previous from iterative formula of euler's method 
%         \[
%             \be_{n+1,h} = 
%             \be_{n,h} + h\p{
%                 \bf(t_n, \by(t_n) + \be_{n,h}) - \bf(t_m, \by(t_n))
%             } + \sO(h^2)
%         \]
%         By triangle inequality and Lipschitz condition 
%         \begin{align*}
%             \norm{\be_{n+1, h}}
%             &\leq \norm{\be_{n,h}} + h\norm{\bf(t_n, \by(t_n) + \be_{n,h}) - \bf(t_n, \by(t_n))} + ch^2 \\
%             &\leq (1+h\lambda) \norm{\be_{n,h}} + ch^2
%         \end{align*}
%         for $n=0,1,\cdots,\lfloor t^*/h \rfloor - 1$. By induction on $n$, we can show $\norm{\be_{n,h}} \leq \frac{c}{\lambda} h \p{(1+h\lambda)^n - 1}$. Since $1+h\lambda < e^{h\lambda}$ we have $(1+h\lambda)^n < e^{nh\lambda} < e^{\lfloor t*/h\rfloor h\lambda} \leq e^{t^* \lambda}$. Therefore 
%         \[
%             \norm{\be_{n,h}}   \leq \frac{c}{\lambda} (e^{t^*\lambda} - 1)h
%         \]
%         for $n=0,1,\cdots,\lfloor t*/h\rfloor$. This is an upper bound on the error that is independent of $h$, hence $\lim_{h\to 0} \norm{\be_{n,h}} = 0$. from which we can infer that error decays globally as $\sO(h)$
%     \end{proof}
% \end{theorem*}

% \begin{definition*}
%     \bheading{order $p$ method} Given arbitrary time-stepping method 
%     \[
%         \by_{n+1} = \bsY_n (\bf, h. \by_0, \by_1, \cdots, \by_n) \quad \quad n = 0,1,\cdots   
%     \]
%     for initial value problem, it is of order $p$ if 
%     \[
%         \by(t_{n+1}) - \bsY_n (\bf, h, \by(t_0), \by(t_1), \cdots, \by(t_n)) = \sO(h^{p+1})
%     \]
%     for every analytic $\bf$ and $n=0,1,\cdots$. Intuitively, a method is of order $p$ if it recovers exactly every polynomial oslution of degrees $p$ or less.
%     \begin{enumerate}
%         \item \heading{intuition} order of a method gives information about local behavior, i.e. advancing from $t_n$ to $t_{n+1}$ where $h>0$ is sufficiently small, we are incurring an error of $\sO(h^{p+1})$. Generally want the the global (convergence) behavior of the method instead.
%         \item \bheading{fact} euler's method is order 1
%         \begin{proof}
%             Euler's method can be written as $\by_{n+1} - \p{\by_n + h\bf(t_n, \by_n)} = 0$. Replace $\by_k$ by $\by(t_k)$ and expand terms of Taylor series about $t_n$ we have 
%             \[
%                 \by(t_{n+1}) - \p{\by(t_n) + h\bf(t_n, \by(t_n))}
%                  = \p{
%                      \by(t_n) + h\by'(t_n) + \sO(h^2)
%                  } - \p{
%                      \by(t_n) + h\by'(t_n)
%                  } = \sO(h^2)
%             \]
%         \end{proof}
%     \end{enumerate}
% \end{definition*}



% \subsection{\linkbook{30}{The trapezoidal rule}}


% \begin{definition*}
%     \bheading{Trapezoidal Rule} Instead of approximating derivative by a constant in $[t_n, t_{n+1}]$, namely by its value at $t_n$, the trapezoidal rule approximates the value of the derivate by average of values at the endpoints. We can approximate solution $\by(t)$ by 
%     \begin{align*}
%         \by(t) 
%         &= \by(t_n) + \int_{t_n}^t \bf(\tau, \bf(\tau)) d\tau \\
%         &\approx \by(t_n) + \frac{1}{2} (t-t_n) \p{\bf(t_n, \by(t_n)) + \bf(t, \by(t))}
%     \end{align*}
%     Given a sequence of times $(t_n)_{n\in\N} = \p{t_0, t_0+h,\cdots}$ we have numerical approximation $(\by_n)_{n\in\N}$ by 
%     \[
%         \by_{n+1} = \by_n + \frac{1}{2} h \p{
%             \bf(t_n, \by_n) + \bf(t_{n+1}, \by_{n+1})
%         }
%     \]
%     \begin{enumerate}
%         \item \bheading{theorem} order of trapezoidal rule is 2
%         \begin{proof}
%             Compute by performing Taylor expansion on $\by(t_{n+1})$ and $\by'(t_{n+1})$ about $t_n$
%             \[
%                 \by(t_{n+1}) - \pc{
%                     \by(t_n) + \frac{1}{2}h \pc{
%                         \bf(t_n, \by(t_n)) + \bf(t_{n+1}, \by(t_{n+1}))
%                     }
%                 } = \sO(h^3)
%             \]
%         \end{proof}
%         \item \bheading{theorem} trapezoidal rule is convergent
%         \begin{proof}
%             Detail of proof \linkbook{31}{here}. We can show error is bounded by 
%             \[
%                 \norm{\be_{n,h}}\leq \frac{ch^2}{\lambda} exp \p{
%                     \frac{t^* \lambda}{1 - \frac{1}{2}h\lambda}
%                 }
%             \]
%             from which we can infer that error decays globally as $\sO(h^2)$
%         \end{proof}
%         \item \heading{note} euler's method is explicit, since we can compute $\by_{n+1}$ with a few arithmetic operations by computing $\bf$, a function of a known $\by_n$. Trapezoidal rule is implicit, i.e. finding $\by_{n+1}$ is not trivial and $\bf$ is a function of both $\by_n$ and $\by_{n+1}$. We might need to solve a nonlinear equation of $\by_{n+1}$
%         \[
%             \by_{n+1} - \frac{1}{2}h \bf(t_{n+1}, \by_{n+1}) = \bv
%         \]
%         where $\bv = \by_n + \textstyle \frac{1}{2} h \bf(t_n, \by_n)$ can be evaluated easily from assumptions.
%     \end{enumerate}
% \end{definition*}



% \subsection{\linkbook{35}{The theta method}}

% \begin{definition*}
%     \bheading{theta method} is a generalization of Euler's method ($\theta=1$) and the trapezoidal rule ($\theta = 1/2$), whereby the derivates are assumed to be piecewise constant and provided by a linear combination of derivatives at the endpoints of each interval. The numerical approximates are,
%     \[
%         \by_{n+1} = \by_n + h\p{
%             \theta \bf(t_n, \by_n) + (1-\theta) \bf(t_{n+1}, \by_{n+1})
%         }
%         \quad \quad n = 0,1,\cdots
%     \]
%     for some fixed $\theta \in [0,1]$
%     \begin{enumerate}
%         \item \heading{fact} theta method is explicit for $\theta=1$ and implicit otherwise
%         \item \bheading{theorem} theta method is of order 2 for $\theta=1/2$ and order 1 otherwise.
%         \item \bheading{theorem} theta method is convergent for every $\theta \in [0,1]$
%     \end{enumerate}
% \end{definition*}



% \section{\linkbook{41}{Multistep Method}}
% \section{\linkbook{41}{Runge-Kutta Methods}}
% \section{\linkbook{41}{Stiff equations}}
% \section{\linkbook{41}{Geometric Numerical Integration}}
% \section{\linkbook{41}{Error Control}}
% \section{\linkbook{41}{Nonlinear Algebraic systems}}





% \section{\linkbook{161}{Finite Differences Schemes}}

% \subsection{\linkbook{161}{Finite differences}}


% \begin{enumerate}
%     \item \bheading{Finite difference operators} Given real sequences $\bz = \pc{z_k}_{k\in\Z} = z(kh)$ for $k\in\Z$ as discrete sampling of a function $z$ for some $h>0$. Let $x_k = kh$. We can define finite difference operators mapping the space $\R^{\Z}$ of all such sequences to itself.
%     \begin{align*}
%         (\shift \bz)_k 
%             &= z_{k+1} \tag{shift} \\
%         (\forw \bz)_k  
%             &= z_{k+1} - z_k \tag{forward difference} \\
%         (\back \bz)_k 
%             &= z_k - z_{k-1} \tag{backward difference} \\
%         (\cent \bz)_k 
%             &= z_{k+\frac{1}{2}} - z_{k - \frac{1}{2}} \tag{central difference} \\
%         (\avg \bz)_k 
%             &= \textstyle\frac{1}{2} (z_{k-\frac{1}{2}} + z_{k+\frac{1}{2}}) \tag{averaging}
%     \end{align*}
%     Finite difference operators are composed under function composition.
%     \begin{enumerate}
%         \item \bheading{fact} $\sT \in \pc{\shift, \forw, \back, \cent, \avg, \diff}$ are linear operators
%         \[
%             \sT(a\bw + b\bz) = a\sT\bw + b\sT\bz
%             \quad \text{ for } \quad
%             \bw,\bz\in\R^{\Z}\;,\;\; a,b\in\R    
%         \]
%         \item \bheading{convention} $\sT z_k$ stands for $(\sT \bz)_k$
%     \end{enumerate} 
%     \item \bheading{Differential operator} The goal is to approximate derivatives $\diff$ by expresing it with a linear combination of values along the grid.
%     \begin{align*}
%         (\diff\bz)_k &= z'(kh) \tag{differential}
%     \end{align*}
%     \item \bheading{Functions of operators} Finite difference operators are functions of $h$. Given an analytic function as Taylor series, $g(x) = \textstyle\sum_{j=0}^{\infty} a_j x^j$, we can expand $g$ about $\pc{\shift-\id, \avg - \id, \forw, \back, \cent, h\diff}$,
%     \[
%         g(\forw)\bz = \p{
%             \sum_{j=0}^{\infty} a_j \forw^j
%         } \bz = 
%         \sum_{j=0}^{\infty} a_j (\forw^j \bz)
%     \]
%     \item \bheading{Asymptotics of operators}
%     \[
%         \pc{\shift-\id, \avg - \id, \forw, \back, \cent, h\diff} \overset{h\to 0+}{\longrightarrow} O
%     \]
%     \begin{enumerate}
%         \item \bheading{example} 
%         \[
%             \forw z_k = z_{k+1} - z_k =  z(x_k + h) - z(x_k) = hz'(\eta_k) = \sO(h)
%         \]
%         by some $\eta_k \in [x_k, x_{k+1}]$ by mean value theorem
%     \end{enumerate}
%     \item \bheading{Operator $\shift^{1/2}$}
%     \[
%         (\shift^{1/2} \bz)_k = z((k+\frac{1}{2})h)
%     \]
%     by defining it with a power series expansion of $g(x) = \sqrt{1+x}$
%     \[
%         \shift^{1/2} = (\id + (\shift - \id))^{1/2} = \id + \sum_{j=0}^{\infty} \frac{ (-1)^{j-1} }{ 2^{2j - 1} } \frac{(2j - 2)!}{(j-1)!j!}  (\shift - \id)^j
%     \]
%     \item \bheading{Operator commutativity} Idea is all operator can be expressed w.r.t. $\shift$
%     \begin{align*}
%         \forw 
%             &= \shift - \id \\
%         \back
%             &= \id - \shift^{-1} \\ 
%         \cent
%             &= \shift^{1/2} - \shift^{-1/2} \\
%         \avg
%             &= \frac{1}{2} (\shift^{1/2} + \shift^{-1/2}) \\
%         \id
%             &= \shift^0 \\
%         h\diff
%             &= \ln \shift
%     \end{align*}
%     \begin{proof}
%         Rest are trivial. To show $h\diff = \ln\shift$, note 
%         \[
%             \shift z(x) = z(x+h) = 
%             \sum_{i=0}^{\infty} \frac{h^i}{i!} \frac{d^i z(x)}{dx^i}
%             = \pb{
%                 \sum_{i=0}^{\infty} \frac{1}{i!} (h\diff)^i
%             } z(x)
%             = e^{h\diff} z(x)
%         \]
%     \end{proof}
%     \item \bheading{rewrite $\diff$ in terms of $\forw, \back, \cent$} 
%     \begin{align*}
%         h\diff
%             &= \ln \p{\id + \forw} \\
%         h\diff
%             &= -\ln \p{\id - \back} \\
%         h\diff
%             &= 2\ln \p{\frac{1}{2}\cent + \sqrt{\id + \frac{1}{4}\cent^2}}
%     \end{align*}
%     \begin{proof}
%         From previous, $\shift = \id + \forw = \p{\id - \back}^{-1}$. For the last expression, consider
%         \begin{align*}
%             \cent &= \shift^{1/2} - \shift^{-1/2} \\
%             \shift^{1/2} \cent &= \shift - \id \\
%             (\shift^{1/2})^2 - \shift^{1/2}\cent - \id &= 0\\
%             \shift^{1/2} &= \frac{1}{2} \cent \pm \sqrt{\id + \frac{1}{4}\cent^2} \tag{$+$ is correct}\\
%             \shift &= \p{
%                 \frac{1}{2}\cent + \sqrt{\id + \frac{1}{4}\cent^2}
%             }^2
%         \end{align*}
%     \end{proof}
%     \item \bheading{approximate $\diff$ and its powers} To approximate $\diff$ with $\forw$, we can expand $\ln\p{\id + \forw}$ by power series expansion of $\ln(1+x) = \textstyle\sum_{i=1}^{\infty} (-1)^{n+1} \frac{x^i}{i!}$ 
%     \begin{align*}
%         \diff 
%         &= \frac{1}{h} \ln\p{\id + \forw}
%         = \frac{1}{h} \pb{
%             \forw - \frac{1}{2} \forw^2 + \frac{1}{3} \forw^3 + \sO(\forw^4)
%         } \\
%         &= \frac{1}{h} \p{
%             \forw - \frac{1}{2} \forw^2 + \frac{1}{3} \forw^3
%         } + \sO(h^3) \quad \quad h\to 0
%     \end{align*}
%     where $\forw = \sO(h)$ as $h\to 0$ shown perviously. Use binomial theorem on $\diff$ repeatedly and collect terms to $\sO(h^3)$, 
%     \[
%         \diff^s = \frac{1}{h^s}\pb{
%             \forw^s - \frac{1}{2} s \forw^{s+1} + \frac{1}{24} s(3s + 5) \forw^{s+2}
%         } + \sO(h^3)  \quad \quad h\to 0
%     \]
%     Inuititively, we can approximate $\diff^s z_k = d^s z(kh)/ dx^s$ up to $\sO(h^3)$ with $s+3$ grid points in the positive direction, i.e. $z_k, z_{k+1}, \cdots, z_{k+s+2}$. Similarly we can express $\diff$ in terms of grid points to the left with $\back$.
%     \[
%         \diff^s = \frac{(-1)^s}{h^s} \p{\ln\p{\id - \back}}^s
%         = \frac{1}{h^s} \pb{
%             \back^s + \frac{1}{2} s \back^{s+1} + \frac{1}{24} s(3s + 5) \back^{s+2}
%         } + \sO(h^3)  \quad \quad h\to 0
%     \]
%     Similarly we can express $\diff$ in terms of grid points on the left and right with $\cent$ operator. Note, only even powers of $\cent$ maps $\R^{\Z} \to \R^{\Z}$, i.e. onto grid points. ($\cent^2 z_k = z_{k+1} -2z_k + z_{k-1}$ and for any power to $2s$, $\cent^{2s} = \p{\cent^2}^{^s}$). We consider Maclaurin expansion of function $g(\xi) = \ln(\xi + \sqrt{1+\xi^2})$
%     \[
%         g(\xi) = 2 \sum_{j=0}^{\infty} \frac{(-1)^j}{2j + 1} \binom{2j}{j} \p{\frac{1}{2} \xi}^{2j+1}    
%     \]
%     Let $\xi = \textstyle\frac{1}{2} \cent$, we have power series expansion of $\diff$ in terms of $\cent$
%     \[
%         \diff = \frac{2}{h} g(\frac{1}{2} \cent)
%         = \frac{4}{h} \sum_{j=0}^{\infty} \frac{(-1)^j}{2j+1} \binom{2j}{j} \p{\frac{1}{4} \cent}^{2j+1}
%     \]
%     However powers of $\cent$ are all odd, we raise power to $2s$ to keep output of operator on the grid
%     \[
%         \diff^{2s} = \frac{1}{h^{2s}}     \pb{
%             \p{\cent^2}^s - \frac{s}{12} \p{\cent^2}^{s+1} + \frac{s(11+5s)}{1440} \p{\cent^2}^{s+2}
%         } + \sO(h^6)
%         \quad \quad  h\to 0
%     \]
%     approximates $\diff$ up to $\sO(h^6)$
%     \item \bheading{comparing $\forw$ and $\cent$ for approximating $\diff$} To attain $\sO(h^{2p})$ error, $\forw$ requires $2s+2p$ grid points and $\cent$ requires $2s+2p-1$ grid points. However $\forw$ would have a smaller error constant. (exercise 8.3)
%     \item \bheading{express $\avg$ in terms of $\cent$}
%     \[
%         \avg = \p{\id + \frac{1}{4}\cent^2}^{1/2}    
%     \]
%     \begin{proof}
%     \begin{align*}
%         \avg &= \frac{1}{2} \p{\shift^{1/2} + \shift^{-1/2}}
%         \quad &\longrightarrow \quad
%         4\avg^2 &= \shift + 2\id + \shift^{-1} \\
%         \cent &= \shift^{1/2} - \shift^{-1/2}
%         \quad &\longrightarrow \quad
%         \cent^2 &= \shift - 2\id + \shift^{-1}
%     \end{align*}
%     Therefore $4\avg - \cent^2 = 4\id$ and result follows
%     \end{proof}
%     \item \bheading{approximate odd derivatives with $\avg$ with central difference} 
%     \[
%         \diff = \frac{1}{h} \p{\avg\cent} \pb{
%             \sum_{j=0}^{\infty} (-1)^j \binom{2j}{j} \p{\frac{1}{16}\cent^2}^j
%         }\pb{
%             \sum_{i=0}^{\infty} \frac{(-1)^i}{2i + 1} \binom{2i}{i} \p{\frac{1}{16}\cent^2}^i
%         }
%     \]
%     which are constructed from even powers of $\cent$ and $\avg\cent$ which will make image of $\diff$ operator reside on the grid.
%     \[
%         \avg\cent z_k = \avg \p{z_{k+\frac{1}{2}} - z_{k - \frac{1}{2}}} = \frac{1}{2}\p{z_{k+1} - z_{k-1}}
%     \]
%     Raising powers of $\diff$ yield
%     \[
%         \diff^{2} = \frac{1}{h^2} (\avg\cent)^2 (\id - \frac{1}{3}\cent^2) + \sO(h^4)
%     \]
%     \item \bheading{practical use} instead of mixing difference operators, opt for finite difference grids. For finite grids, one-sided finite differences can be employed to evaluate $\diff$ near boundaries. 
% \end{enumerate}


% \subsection{\linkbook{169}{The five-point formula for $\nabla^2 u = f$}}


% \begin{enumerate}
%     \item \bheading{consistent} A method is consistent if the truncation error goes to 0 as step size goes to zero
%     \item \bheading{order of consistency} of $\sO(\Dx^p)+\sO(\Dy^q)$ is $p$ in $x$ and $q$ in $y$.
%     \item \bheading{theorem} If a method is consistent and stable, then it is convergent and order of convergence will be same as the order of consistency
% \end{enumerate}


% \begin{enumerate}
%     \item \bheading{Poisson Equation} 
%     \[
%         \nabla^2 u = ( \frac{\partial^2}{\partial x^2} + \frac{\partial^2}{\partial y^2}) u = f \quad \quad (x,y)\in\Omega
%     \]
%     and $f=f(x,y)$ is continuous, domain $\Omega\subset \R^2$ is bounded, open, and connected and has a piecewise-smooth boundary. Assume \textit{Dirichlet condition}, i.e. 
%     \[
%         u(x,y) = \phi(x,y) \quad \quad (x,y)\in \partial \Omega    
%     \]
%     \item \bheading{Setup} Inscribe a grid that is axis-aligned with equal spacing of $\Delta x$ in both direction, i.e. pick $\Delta x>0$, $(x_0, y_0) \in \Omega$ and let $\Omega_{\Delta x}$ be 
%     \[
%         \Omega_{\Delta x} = \pc{
%             x_0 + k\Delta x, y_0 + l \Delta x
%         } \subset \closure{\Omega}
%     \]
%     Denote 
%     \begin{align*}
%         \bI_{\Dx} &= \pc{
%             (k,l)\in\Z^2 \mid (x_0 + k\Dx, y_0 + l\Dx) \subset \closure{\Omega}
%         } \\
%         \bI_{\Dx}^{\circ} &= \pc{
%             (k,l)\in\Z^2 \mid (x_0 + k\Dx, y_0 + l\Dx) \subset \Omega
%         }
%     \end{align*}
%     and for every $(k,l) \in \bI_{\Dx}^{\circ}$, let $u_{k,l}$ be \textit{approximation} to the solution $u(x_0 + k\Dx, y_0 + l\Dx)$ of the Poisson equation at the relevant grid point. Note there is no need to approximate points in $\bI_{\Dx} \setminus \bI_{\Dx}^{\circ}$ since they lie on $\partial \Omega$ and their exact values given by $\phi$.
%     \item \bheading{internal, near-boundary, boundary points} A point on the grid $(k,l) \in \sI_{\Dx}$ whereby $(k\pm 1,l)$ and $(k, l\pm 1)$ are in $\bI_{\Dx}$ is called \textit{internal point}. A point $(k,l) \in \sI_{\Dx}$ where we can no longer employ a finite difference scheme (and so requires a special approach) is called \textit{near-boundary points}. $(k,l) \in \partial \Omega$ are called \textit{boundary points}
%     \item \bheading{Central difference approximation} given $u$ sufficiently smooth, we can approximate $\nabla^2$
%     \[
%         \nabla^2 = \frac{1}{(\Dx)^2} \p{\vD_{0,x}^2 + \vD_{0,y}^2} + \sO(\p{\Dx}^2)
%         \quad \text{where} \quad
%         \frac{\partial^2 u}{\partial x^2} = \frac{1}{\Dx^2} \vD_{0,x}^2 u_{k,l} + \sO((\Dx)^2)
%     \]
%     with central difference operators, i.e. $\vD_{0,x}, \vD_{0,y}$ along the x,y-axis.We can rewrite Poisson equation by the \textit{five point} finite difference scheme. For every internal grid point $(k,l)$, we have 
%     \[
%         \frac{1}{(\Dx)^2} (\vD_{0,x}^2 + \vD_{0, y}^2) u_{k,l} = f_{k,l}
%     \]
%     where $f_{k,l} = f(x_0 + k\Dx, y_0 + l\Dx)$. Expanding expression, we have
%     \[
%         u_{k-1,l} + u_{k+1, l} + u_{k, l-1} + u_{k, l+1} - 4u_{k,l} = (\Dx)^2 f_{k,l}    
%     \]
%     Intuitively, we have a linear combination of values of $u$ at grid point and at immediate horizontal and vertical neighbors of this point.
%     \item \bheading{properties}
%     \begin{enumerate}
%         \item \bheading{truncation error} $\sO(\Dx^2) + \sO(\Dy^2)$ (computed by substituting exact solution $\tu_{i,j} = u(x_0+\Dx,y_0+\Dy)$ to finite difference formula in place of approximate values at grid points $u_{i,j}$ to compute the truncation error)
%         \begin{align*}
%             &\frac{ \tu_{i+1,j} - 2\tu_{i,j} + \tu_{i-1,j} }{\Dx^2} + 
%             \frac{ \tu_{i, j+1} -2\tu_{i,j} + \tu_{i,j-1} }{\Dy^2} - f(x_i, y_i) \\
%             = &\frac{\partial u(x_i, y_j)}{\partial x^2} + \sO(\Dx^2) + 
%             \frac{\partial u(x_i, y_j)}{\partial y^2} + \sO(\Dy^2) - f(x_i, y_i) \\
%             = &\sO(\Dx^2) + \sO(\Dy^2)
%         \end{align*}
%     \end{enumerate}
%     \item \bheading{computational stencil / molecule} 
%     \begin{center}
%         \includegraphics[width=0.4\textwidth]{five_point_stencil}
%     \end{center}
%     \item \bheading{solving linear equation} Main idea of finite difference method is to associate every grid point having an index in $\bI_{\Dx}^{\circ}$ a single linear equation. Solving a system of linear equations whose solution is our approximation $\bu = (u_{k,l})_{(k,l)\in \bI_{\Dx}^{\circ}}$. Performance of finite difference is evaluated with 
%     \begin{enumerate}
%         \item nonsingular linear system (such that $\bu$ exists and is unique)
%         \item for $\Dx \to 0$, the convergence of $\bu$ to exact solution of Poisson equation and error
%         \item efficient and robust ways to solve sparse linear systems
%     \end{enumerate}
%     \item \bheading{A simplified grid} over a square $\Omega$. Let 
%     \[
%         \Omega = \pc{(x,y) \mid 0 < x,y < 1}    
%         \quad \quad
%         \Dx = 1/(m+1)
%         \quad \quad
%         (x_0, y_0) = 0
%     \]
%     \begin{center}
%         \includegraphics[width=0.3\textwidth]{simplified_grid}
%     \end{center}
%     \item \bheading{numerical example on Laplace equation} indicates that error decreases by a factor of 4 when number of steps $m$ increase by a factor of 2.
%     \item \bheading{discretization to a system of linear equations} Rearrange $u_{k,l}$ to $\bu \in \R^s$ where $s=m^2$ according to some permutation $\pc{\p{k_i, l_i)}}_{i=1,2,\cdots,m}$ and write 
%     \[
%         A \bu = \bb    
%     \]
%     where $A$ is $s\times s$ matrix and $\bb \in\R^s$ includes both $(\Dx)^2 f_{k,l}$ and boundary values.
%     \item \bheading{lemma (unique solution to linear system)} $A$ from previous is symmetric and the set of its eigenvalues is 
%     \[
%         \sigma(A) = \pc{
%             \lambda_{\alpha,\beta} \mid \alpha,\beta = 1,2,\cdots,m
%         }
%     \]
%     where 
%     \[
%         \lambda_{\alpha,\beta} = -4\pc{
%             \sin^2 \pb{
%                 \frac{\alpha\pi}{2(m+1)}
%             } + 
%             \sin^2 \pb{
%                 \frac{\beta\pi}{2(m+1)}
%             }
%         }
%     \]
%     \begin{proof}
%         Symmetry follows by examining matrix $A$. To find eigenvalues of $A$, find nonzero functions $(v_{k,l})_{k,l=0,1,\cdots,m+1}$ such that $v_{k,0} = v_{k,m+1}= v_{0,l} = v_{m+1,l} = 0$ where $k,l=1,2,\cdots, m$ such that 
%         \[
%             v_{k-1,l} + v_{k+1,l} + v_{k,l-1} + v_{k,l+1} - 4v_{k,l}  = \lambda v_{k,l}
%             \quad 
%             k,l = 1,2,\cdots, m   
%         \]
%         is satisfied for some $\lambda$. It follows that $(v_{k,l})$ is an eigenvector and $\lambda$ is the corresponding eignvalue of $A$. Given $\lambda_{\alpha,\beta}$ for some $\alpha,\beta$, we show that
%         \[
%             v_{k,l} = \sin\p{\frac{k\alpha\pi}{m+1}} \sin \p{\frac{l\beta\pi}{m+1}}
%             \quad 
%             k,l = 0,1,\cdots,m+1
%         \] 
%         is the corresponding eigenvector by verifying above formula. (note $\alpha,\beta$ specifies one eigenvalue / eigenvector pair and $k,l$ specifies discrete grid sampling of the eigenfunction $v$)
%     \end{proof}
%     \item \bheading{corollary} The matrix $A$ is negative definite and, therefore, nonsingular
%     \begin{proof}
%         $A$ is symmetric and from previous lemma eigenvalues are negative, therefore it is negative definite and nonsingular
%     \end{proof}
%     \item \bheading{eigenvalues of the Laplace operator} The function $v$, not identically zero, is said to be an \textit{eigenfunction} of $\nabla^2$ in a domain $\Omega$ and $\lambda$ is the corresponding eigenvalue if $v$ vanishes along $\partial \Omega$ and satisfies within $\Omega$ the equation $\nabla^2 v = \lambda v$. Note eigenvalues and eigenfunctions of the Laplace operator $\nabla^2$ over $(0,1)^2$ are \textit{related to} eigenvalues and eigenvectors of the matrix $A$. Given $\alpha,\beta$, eigenvalue of $\nabla^2$ and the corresponding eigenfunction is given by
%     \begin{align*}
%         \lambda_{\alpha,\beta} 
%             &= -(\alpha^2 + \beta^2) \pi^2 \\
%         v(x,y) 
%             &= \sin(\alpha\pi x) \sin(\beta\pi y)
%             \quad \quad 
%             x,y\in [0,1]
%     \end{align*}
%     We can easily verify that 
%     \begin{align*}
%         \nabla^2 v 
%         &= -\alpha^2 \pi \sin(\alpha\pi x) \sin(\beta \pi y) - \beta^2 \pi \sin(\alpha\pi x) \sin(\beta\pi y) \\
%         &= -(\alpha^2 + \beta^2) \pi^2 v
%     \end{align*}
%     $v$ obeys boundary conditions. Note eigenvectors $v_{k,l}$ for $A$ can be obtained by sampling of the eigenfunction $v$ at grid points 
%     \[
%         \pc{\p{
%             \frac{k}{m+1}, \frac{l}{m+1}
%         }}_{k,l=0,1,\cdots,m+1}
%     \]
%     Note $(\Dx)^{-2} \lambda_{\alpha,\beta}$ is a good approximation to $-(\alpha^2 + \beta^2) \pi^2$ provided $\alpha,\beta$ are small in comparison with $m$. Note we can expand $\sin^2$ in a power series
%     \begin{align*}
%         \frac{\lambda_{\alpha,\beta}}{\p{\Dx}^2}
%             =& -4 \p{
%                 \pc{
%                     \p{\frac{\alpha\pi}{2(m+1)}}^2 - \frac{1}{3} \p{\frac{\alpha\pi}{2(m+1)}}^4 + \cdots
%                 } + 
%                 \pc{
%                     \p{\frac{\beta\pi}{2(m+1)}}^2 - \frac{1}{3} \p{\frac{\beta\pi}{2(m+1)}}^4 + \cdots
%                 } } \\
%             =& -(\alpha^2 + \beta^2) \pi^2 + \frac{1}{12} (\alpha^4 + \beta^4) \pi^4 (\Dx)^2  + \sO((\Dx)^4)
%     \end{align*}
%     \item \bheading{theorem (convergence)} Subject to sufficient smoothness of the function $f$ and the bounordary conditions, there exists a number $c>0$, independent of $\Dx$, such that 
%     \[
%         \norm{\be} \leq c(\Dx)^2
%         \quad \quad \Dx \to 0   
%     \]
%     or equivalently 
%     \[
%         \lim_{\Dx \to 0} \norm{\be}_{\infty} = 0
%     \]
%     where $\be \in \R^s$, $s=m^2$ in same order as that of $\bu$.
%     \begin{proof}
%         Idea is to let $\be = \pc{e_{k,l}} = \pc{u_{k,l} - \tilde{u}_{k,l}}$ for $k,l=0,1,\cdots,m+1$. We can show 
%         \[
%             A\be = \bdelta_{\Dx}
%             \quad \rightarrow \quad
%             \be = A^{-1} \bdelta_{\Dx}
%         \]
%         where $\bdelta_{\Dx} = \sO((\Dx)^4)$. Since $A$ symmetric,
%         \begin{align*}
%             \norm{A^{-1}}_2 = \rho(A^{-1})
%             &= \max_{\alpha,\beta = 1,2,\cdots,m} \frac{1}{4} \pc{
%                 \sin^2 \pb{\frac{\alpha\pi}{2(m+1)}} + 
%                 \sin^2 \pb{\frac{\beta\pi}{2(m+1)}}
%             }^{-1} \\
%             &= \frac{1}{8\sin^2 \p{\frac{1}{2} \Dx \pi}} \tag{$\lambda_{1,1}$} \\
%             &= \frac{1}{2\pi^2} \tag{$\Dx\to 1$}
%         \end{align*}
%         so for any $c_1 > 1/(2\pi^2)$, and any $c_2 > 0$ such that $\norm{\bdelta_{\Dx}} \leq c_2 (\Dx^4)$, we have 
%         \begin{align*}
%             \norm{\be}
%             &= \norm{A^{-1}} \norm{\bdelta_{\Dx}}  \\
%             &\leq (c_1 (\Dx)^{-2}) (c_2 (\Dx)^4) \\
%             &= c (\Dx)^2
%         \end{align*}
%         where $c=c_1 c_2$
%     \end{proof}
%     \item \bheading{spectral radius} The spectral radius of a square matrix or a bounded linear operator is the largest absolute value of its eigenvalues
%     \[
%         \rho(A) = max \pc{
%             \abs{\lambda} \mid \lambda \text{  is an eigenvalue of } A
%         }    
%     \]
%     If $A$ is symmetric, then 
%     \[
%         \norm{A} = \rho(A)
%     \]
%     (in lecture proved this for $\norm{\cdot}_2$)
%     \item \bheading{matrix norm} is a vector norm in a vector space whose elements are matrices. Given vector norm $\norm{\cdot} \in\R^d$ and any $m\times n$ matrix $A$ induces a linear operator from $K^n$ to $K^m$ and defines a corresponding induced norm on space $K^{m\times n}$ as follows 
%     \begin{align*}
%         \norm{A} 
%         &= \max_{\bx \neq 1} \norm{Ax} \\
%         &= \max_{\bx \neq 0} \frac{\norm{Ax}}{\norm{x}} 
%     \end{align*}
%     \begin{enumerate}
%         \item \heading{example} $\norm{A}_{1} = \max_{1\leq j \leq n} \sum_{i=1}^m \abs{a_{ij}}$ (max absolute column sum)
%         \item \heading{example} $\norm{A}_{\infty} = \max_{1\leq i \leq m} \sum_{j=1}^n \abs{a_{ij}}$ (max absolute row sum)
%         \begin{align*}
%             \norm{A}_{\infty} 
%             &= \max_{\norm{x}_{\infty}=1} \norm{Ax}_{\infty} \\
%             &= \max_{\norm{x}_{\infty}=1} \max_{i=1,\cdots,n} \abs{\sum_{j=1}^n a_{ij} x_j} \\
%             &= \max_{i=1,\cdots,n} \max_{\norm{x}_{\infty}=1} \abs{\sum_{j=1}^n a_{ij} x_j} \\
%             &= \max_{i=1,\cdots,n} \abs{\sum_{j=1}^n a_{ij}}
%         \end{align*}
%         where last equality achieved by picking $x_j \in\pc{-1,1}$ to be same sign as $a_{ij}$
%         \item \heading{example} $\norm{A}_{2} = \sqrt{\rho(A^*A)}$ (spectral norm: maximum singular value of $A$) If $A$ symmetric, $\norm{A}_2 = \rho(A)$
%     \end{enumerate}
%     \item \bheading{handle near boundary grid points} approximate $z''$ at $P$ in $x$ direction as a linear combination of value of $z$ at $P,Q,T$. The coefficient of terms can be determined via Taylor expansion of $z_{x_0-\Dx},z_{x_0},z_{x_0+\tau\Dx}$ at $a=x_0$ and solve a $3\times 3$ linear system. The error of approximation is $\sO(\Dx)$
%     \begin{align*}
%         \frac{1}{(\Dx)^2} &\pb{
%             \frac{2}{\tau+1} z(x_0-\Dx) - \frac{2}{\tau} z(x_0) + \frac{2}{\tau(\tau+1)} z(x_0+\tau\Dx)
%         } \\
%         &= z''(x_0) + \frac{1}{3} (\tau-1)z'''(x_0)\Dx + \sO((\Dx)^2)
%     \end{align*}
%     To achieve $\sO((\Dx)^2)$ order for the error, we use value of $z$ at 4 grid points $V,Q,P,T$ to approximate $z''$. Coefficient to linear term can be determined with Taylor expansion and solve a $4\times 4$ linear system.
%     \begin{align*}
%         \frac{1}{(\Dx)^2} &\pb{
%             \frac{\tau-1}{\tau+2} z(x_0-2\Dx) + \frac{2(2-\tau)}{\tau+1} z(x_0-\Dx) - \frac{3-\tau}{\tau} z(x_0) + \frac{6}{\tau(\tau+1)(\tau+2)} z(x_0+\tau\Dx)
%         } \\ 
%         &= z''(x_0) + \sO((\Dx)^2)
%     \end{align*}
%     In total, a good approximation to $\nabla^2 u$ at $P$ requires 6 points. For $P$ corresponding to grid $(k,l)$, we obtain the following linear equation for constructing $A$ and $b$ for both first order and second order approximations
%     \[
%         \frac{2}{\tau+1} u_{k-1,l} + \frac{2}{\tau(\tau+1)} u_{k+\tau,l} + u_{k,l-1} + u_{k, l+1} - \frac{2+2\tau}{\tau} u_{k,l} = (\Dx)^2 f_{k,l}
%     \]
%     \[
%         \frac{\tau-1}{\tau+2} u_{k-2,l} + \frac{2(2-\tau)}{\tau+1} u_{k-1,l} + \frac{6}{\tau(\tau+1)(\tau+2)} u_{k+\tau, l} + u_{k,l-1} + u_{k,l+1} - \frac{3+\tau}{\tau} u_{k,l}
%         = (\Dx)^2 f_{k,l}
%     \]
%     Note however that the constructed matrix $A$ is not symmetric because of near boundary points. To prove that $A$ is nonsingular we need to use the Gersgorin Theorem.
%     \begin{center}
%         \includegraphics[width=4in]{near_boundary_stencil}
%     \end{center}
%     \item \bheading{The Gersgorin Criterion} Lt $A$ be complex $n\times n$ matrix and let 
%     \[
%         \bbS_i  = \pc{
%             z \in \C \bigm| \abs{ z - a_{ii} } \leq \sum_{j=1 \,:\, j\neq i}^n \abs{a_{ij}}
%         }
%         \quad \text{for} \quad
%         i = 1,\cdots, n
%     \]
%     Then 
%     \begin{enumerate}
%         \item If $\sigma(A)$ is the set of eigenvalues of $A$, then $\sigma(A) \subset \cup_{i}^n \bbS_i$
%         \item Given a partition $\pc{i_1,\cdots, i_r} \cup \pc{j_1,\cdots, j_{n-r}} = \pc{1,\cdots, n}$ where $\bbS_{i_{\alpha}} \cap \bbS_{i_{\beta}} \neq \emptyset$ for $\alpha,\beta = 1,2,\cdots, r$ and $\bbS_{i_{\alpha}} \cap \bbS_{i_{\beta}} = \emptyset$ for $i=1,\cdots, r$, $j=1,\cdots, d-r$. Then $\cup_{\alpha=1}^r \bbS_{i_{\alpha}}$ contains exactly $r$ eigenvalues of $A$ and $\cup_{\beta=1}^{n-r} \bbS_{j_{\beta}}$ contains contains exaclty $n-r$ eigenvalues of $A$ 
%         \item If $A$ is irreducible, and $\lambda \in \partial (\cup_{i=1}^n \bbS_i)$, then $\lambda \in \partial \bbS_i$ for $i=1,2,\cdots,n$
%     \end{enumerate}
%     \begin{proof}
%         \begin{enumerate}
%             \item Let $\lambda$ be given, and let $\bv\neq 0$ be the corresponding eigenvalue. Let $i$ be such that 
%             \[
%                 \abs{v_i} = \max_{j=1,\cdots,n} \abs{x_j} > 0    
%             \]
%             consider $i$th row of $A\bv$, we have 
%             \begin{align*}
%                 (A\bv)_i 
%                     &= (\lambda v)_i \\
%                 \sum_{j=1}^n a_{ij} v_j
%                     &= \lambda v_i \\
%                 (\lambda - a_{ii}) v_i 
%                     &= \sum_{j=1 \,:\, j\neq i}^n a_{ij} v_j \\
%                 \abs{\lambda - a_{ii}} 
%                     &\leq \sum_{j=1 \,:\, j\neq i}^n \abs{a_{ij}} \abs{\frac{v_j}{v_i}}
%                     \leq \sum_{j=1 \,:\, j\neq i}^n \abs{a_{ij}}
%             \end{align*}
%             Therefore $\lambda \in \bbS_i$ so $\lambda \in \cup_{i=1}^n \bbS_i$
%             \item a bit hard to prove ... 
%             \item Given any eigenvalue $\lambda \in \partial (\cup_{i=1}^n \bbS_i)$ there is a corresponding eigenvector $v$. Similar to argument from previous, we can pick $i$ such that 
%             \[
%                 \abs{v_i}= \max_{j=1,\cdots,n} \abs{v_j} > 0    
%             \]
%             wlog $\abs{v_i} = 1$ and $\abs{v_j} \leq 1$. We can show that $\lambda \in \bbS_i$. In fact, $\lambda \in \partial \bbS_i$, since otherwise $\lambda \in \cup_{i=1}^n \bbS_i$ contradicting the fact that $\lambda$ is on the boundary. By the fact that $\lambda$ is on the boundary of $\bbS_i$, we have the following equality
%             \[
%                 \abs{\lambda - a_{ii}} 
%                 = \sum_{j=1 \,:\, j\neq i}^n \abs{a_{ij}} \abs{v_{j}}
%                 = \sum_{j=1 \,:\, j\neq i}^n \abs{a_{ij}}
%             \]
%             where last equality holds if and only if $\abs{v_j} =1$ whenever $a_{ij} \neq 0$. Now we show that for any $k\in\pc{1,\cdots,n}$, we have $\lambda \in\partial \bbS_k$. By irreducibility of $A$, we can find a path $i,i_1,\cdots, i_l,k$ such that $a_{i,i_1} \neq 0$, $\cdots$, $a_{i_l,k}\neq 0$. For any edge along the path, we have $a_{i,j} \neq 0$, therefore $\abs{v_j} = 1$ using the assumption that $\abs{v_i}=1$ and (from previous argument) $\lambda \in \partial \bbS_j$. Do this repeatedly, we get $\lambda \in \partial \bbS_k$. 
%         \end{enumerate}
%     \end{proof}
%     \item \bheading{reducible matrix} A square $n\times n$ matrix $A$ is reducible if either of the following holds
%     \begin{enumerate}
%         \item exists a partition of indices $I_1,I_2 \subset \pc{1,\cdots,n}$ such that $a_{ij} = 0$ for all $i\in I_1$ and $j\in I_2$
%         \item $A$ can be placed into block upper triangular form by simultaneous row/column permutation, i.e. exists permutation $P$ such that $PAP^{-1}$ is upper triangular
%         \item the associated digraph is not strongly connected
%     \end{enumerate}
%     \item \bheading{irreducible matrix} is a matrix that is not reducible. Intuitively the associated digraph is strongly connected, i.e. any $i,j\in \pc{1,2,\cdots,n}$ is connected by a path.
%     \item \bheading{$A$ is nonsingular} Given $A\bu = \bb$ obtained by five-point formula with near boundary points. $A$ is nonsingular
%     \begin{proof}
%         Idea is for each $i$th row, $a_{i,i} < 0$ and 
%         \[
%             \sum_{j\neq i} \abs{a_{i,j}} + a_{i,i} \leq 0
%         \]
%         where inequality is sharp at near-boundary / boundary points. This means that the Gersgorin disc centers in the negative x-axis whose boundary intersects the origin (interior points) or does not reach the origin (near-boundary / boundary point). Assume $0 \in \partial(\cup_{i=1}^n \bbS_i)$, by 3rd point from Gersgorin theorem, we have $0\in \bbS_i$ for all $i$. Contradiction, since $0\not\in \bbS_i$ for near-boundary and boundary points. Therefore $0$ is not an eigenvalue of $A$ implying $A$ is nonsingular
%     \end{proof}
% \end{enumerate}


% \subsection{\linkbook{180}{Higher-order methods for $\nabla^2 u =f$}}

% \begin{enumerate}
%     \item \bheading{9 point formula}
%     \item \bheading{modified 9 point formula}
% \end{enumerate}

 

\section{\linkbook{193}{The Finite Element Method}}
\subsection{\linkbook{193}{Two-point boundary value problem}}

\begin{definition*}
    \bheading{problem setup} Linear two-point boundary value problem
    \[
        -\frac{d}{dx} \pb{a(x) \frac{du}{dx}} + b(x)u = f
        \quad \quad 0\leq x\leq 1
    \]
    where $a$,$b$, and $f$ are given functions, $a$ is differentiable and $a(x) > 0$, $b\geq 0$, $0<x<1$. Assume dirichlet boundary condition 
    \[
        u(0) = \alpha \quad \quad u(1) = \beta   
    \]
\end{definition*}


\begin{definition*}
    \bheading{4 principles of FEM}
    \begin{enumerate}
        \item \bheading{Approximate the solution in a finite-dimensional space}, i.e. approximate $u$ by a linear combination of functions in $\mathring{\mathbb{H}}_m = \text{span}\pc{
            \varphi_1, \cdots, \varphi_m
        }$ where $\varphi_0,\cdots, \varphi_m$ are linearly independent functions satisfying zero boundary condition. So we have an approximate solution $u_m$ given by 
        \[
            u_m(x) = \varphi_0(x) + \sum_{l=1}^m \gamma_l \varphi_l(x)
            \quad \quad 0\leq x \leq 1    
        \]
        \item \bheading{Choose the approximation so the \textbf{defect} is orthogonal to the space $\mathring{\mathbb{H}}_m$} Consider defect as the error of $u_m$ as approximation to $u$,
        \[
            d_m(x) = -\frac{d}{dx} \pb{a(x) \frac{du_m(x)}{dx}} + b(x)u_m(x)- f(x)
            \quad \quad 0 < x < 1
        \]
        Alternatively, we seek $u_m$ such that $d_m$ is orthogonal to all basis elements of $\mathring{\mathbb{H}}_m$, in other words, satisfy the \textbf{Galerkin equations}
        \[
            \inner{d_m}{\varphi_k} \quad \quad k = 1,2,\cdots,m    
        \]
        \item \bheading{Integrate by parts to depress the maximal extent possible the differentiability requirements of the space $\mathring{\mathbb{H}}_m$} We can expand Gerlerkin equations by substituting $u_m$ as a linear combination of basis to obtain a linear system of $m$ equations in $m$ unknown $\gamma_1,\cdots,\gamma_m$
        \[
            \sum_{l=1}^m \gamma_l \pb{
                \inner{-(a\varphi_l')'}{\varphi_k} + \inner{b\varphi_l}{\varphi_k}
            }
            = \inner{f}{\varphi_k} - \pb{
                \inner{-(a\varphi_0')'}{\varphi_k} + \inner{b\varphi_0}{\varphi_k}
            }
        \]
        for $k=1,2,\cdots,m$. Given standard Euclidean inner product over funtion spaces 
        \[
            \inner{v}{w} = \int_0^1 v(\tau) w(\tau) d\tau
        \]
        We compute integration by parts to reduce differentiability requirements for $\varphi$s
        \begin{align*}
            \inner{-(a\varphi_l')'}{\varphi_k}
            &= \int_0^1 \pc{-(a(\tau)\varphi_l(\tau)')'}\varphi_k(\tau) d\tau \\ 
            &= \pb{-a(\tau) \varphi_l'(\tau) \varphi_k(\tau)}_0^1 + \int_0^1 a(\tau) \varphi_l'(\tau) \varphi_k'(\tau) d\tau  \\
            &= \int_0^1 a(\tau) \varphi_l'(\tau) \varphi_k'(\tau) d\tau 
            = \inner{a\varphi_l'}{\varphi_k'} \quad \quad l=,0,1,\cdots,m
        \end{align*}
        The linear system becomes
        \[
            \sum_{l=1}^m a_{k,l} \gamma_l = \int_0^1 f(\tau) \varphi_k(\tau) d\tau - a_{k,0} \quad \quad k=1,2,\cdots,m    
        \]
        where 
        \[
            a_{k,l} = \int_0^1 \pb{
                a(\tau) \varphi'_l(\tau) \varphi'_k(\tau) + b(\tau) \varphi_l(\tau) \varphi_k(\tau)
            } d\tau
        \]
        With the help of integration by parts, and knowledge that value of integral is independent of values that integrand assumes on a finite set (at knots), we can choose basis functions that are piecewise once-differentiable.
        \item \bheading{Choose each function $\varphi_k$ so that it vanishes along most of $(0,1)$, thereby ensuring that $\varphi_k\varphi_l=0$ for most choices of $k,l = 1,2,\cdots,m$} Ideally, we want to choose $\mathring{\mathbb{H}}_m$ to be 
        \begin{enumerate}
            \item Dense in infintie-dimensional space inhabited by the weak solution, i.e. $\norm{u_m - u}$ is as small as possible. (spectral method) and
            \item Choose $\varphi_1,\cdots,\varphi_m$ such that evaluating $\sO(m^2)$ integrals to solve the linear system is as fast as possible.
        \end{enumerate}
        To satisfy (2), we want each function $\varphi_k$ to be supported on a relatively small set $\mathbb{E}_k \subset (0,1)$ and $\mathbb{E}_k \cap \mathbb{E}_l = \emptyset$ for as many $k,l$ as possible. One example is the chapeau (hat) function
        \[
            \varphi_k(x) = 
            \begin{cases}
                1 - k + \frac{x}{h} & (k-1)h \leq x \leq kh \\
                1 + k - \frac{x}{h} & kh \leq x \leq (k+1)h \\
                0 & \abs{x-kh} \geq h
            \end{cases}     
        \]
        which reduces the number of integral evaluation from $\sO(m^2)$ to $\sO(m)$
    \end{enumerate}
\end{definition*} 

\begin{proposition*}
    \bheading{$A$ is positive definite}
    \begin{proof}
        Idea is to show $A$ is a gram matrix (i.e. $A_{k,l} = \inner{\varphi_k}{\varphi_l}$ where $\inner{\cdot}{\cdot}$ is an inner product in $\mathring{\mathbb{H}}_m$). This is easy by verifying axioms of inner product. We claim that any gram matrix is symmetric positive definite. Let $\gamma\in\R^m$ where $\gamma \neq 0$, then 
        \[
            \inner{A\gamma}{\gamma} 
            = \sum_k \gamma_k (A\gamma)_k 
            = \sum_k \gamma_k \sum_l \gamma_l \inner{\varphi_k}{\varphi_l}
            = \inner{\sum_k \gamma_k\varphi_k}{\sum_l \gamma_l\varphi_l} 
            = \inner{u}{u}
            > 0    
        \]
        $u = \sum_k \gamma_k\varphi_k \in \mathring{\mathbb{H}}_m \neq 0$, so inequality follows from positive definiteness of inner product over $\mathring{\mathbb{H}}_m$
    \end{proof}
\end{proposition*}


\begin{definition*}
    \bheading{Weak Solutions} Consider $\mathbb{\sL} \bu = \bf$ where $\mathbb{\sL}$ is a differential operator. The classical solution is given by function $\bu$ that satisfies the equation exactly. In constrast, we can consider $\bv \in\mathring{\mathbb{H}}$, an infinite-dimensional linear space, as a \textbf{weak solution} if the Galerkin equations are satisfied, i.e. 
    \[
        \inner{\mathbb{\sL}\bv - \bf}{\bw} = 0
        \quad \quad 
        \forall \bw \in \mathring{\mathbb{H}}
    \]
    Weak and exact solution are same for ODEs, since Lipschitz condition ensures existence of unique solution. But this is not the case for boundary value problems.
\end{definition*}


\begin{definition*}
    \bheading{Variational formulation of two point BVP} Given a functional $\sJ: \mathbb{H} \to\R$, where $\BH$ is some functional space. Wish to find a function $u\in\BH$ such that 
    \[
        \sJ(u) = \min_{v\in\BH} \sJ(v)    
    \]
    Consider $a,b$, and $f$ given on $(0,1)$ where $a(x)>1$ and $b(x)\geq 1$ and any $v\in\BH$ satisfies
    \[
        \int_0^1 v^2(\tau) d\tau 
        ,\, 
        \int_0^1 \pb{v'(\tau)}^2 d\tau < \infty
    \]
    We let 
    \[
        \sJ(v) = \int_0^1 \pc{
            a(\tau) \pb{v'(\tau)}^2 + b(\tau) \pb{v(\tau)}^2 - 2f(\tau)v(\tau)
        } d\tau
        \quad \quad v\in\BH
    \]
    We can show that $u\in H$ which minimizes $J$ if and only if $u$ is the weak solution of the differential equation given at start of the section. We call the two value boundary value problem the Euler-Lagrange equation of $\sJ$
\end{definition*}


\begin{definition*}
    \bheading{Ritz Method} Let $\mathring{\BH} = \text{span}(\varphi_1,\cdots,\varphi_m)$ and choose arbitrary $\varphi_0 \in \BH$. Therefore $\mathring{\BH}$ is a m-dimensional linear space. We seek a minimum of $\sJ$ in the affine space $\varphi_0 + \mathring{\BH}$. In other word, we seem vector $\bgamma$ that minimizes 
    \[
        \sJ_m(\boldsymbol{\delta}) = \sJ \p{
            \varphi_0 + \sum_{l=1}^m \delta_l \varphi_l
        }
        \quad \quad \boldsymbol{\delta} \in \R^m
    \]
    To find the minimum of $sJ$, We let gradient of $\sJ_m$ be zero,
    \[
        \frac{1}{2} \frac{\partial \sJ(\boldsymbol{\delta})}{\delta_k} 
        = \sum_{l=1}^m \int_0^1 \p{a\varphi_l'\varphi_k' + b\varphi_l \varphi_k} d\tau 
        + \int_0^1 \p{a\varphi_0'\varphi_k' + b\varphi_0\varphi_k} d\tau
        - \int_0^1 f\varphi_k d\tau  
    \]
    Note letting $\partial \sJ_m / \partial \delta_k$ for $k=1,2,\cdots,m$ recovers the Galerkin equations! We can also show that the hessian matrix $(\partial^2 \sJ_m / \partial \partial \delta_k \partial \delta_j)_{k,j=1}^m$ be nonnegative definite. Although we get the same system of equations, we are able to ignore natural boundary conditions.
\end{definition*}
 

\begin{definition*}
    \bheading{Ritz-Galerkin method} 
    \begin{enumerate}
        \item Approximate solution in finite dimensional space $\varphi_0 + \mathring{\BH}_m \subset \BH$
        \item Retain only essential boundary conditions
        \item Choose the approximate so that the defect is orthogonal to $\mathring{\BH}_m$, or alternatively, so that a variational problem is minimized in $\mathring{\BH}_m$
        \item Integrate by parts to depress the maximal extent possible the differentiability requirements of the space $\mathring{\BH}_m$ 
        \item Choose each function in a basis of $\mathring{\BH}_m$ in such a way that it vanishes along much of the spatial domain of interest, thereby ensuring that the intersectino between the supports of most of the basis function is empty
    \end{enumerate}
\end{definition*}

\subsection{\linkbook{206}{Synopsis of FEM theory}}

\begin{definition*}
    \bheading{problem setup} Given boundary value problem
    \[
        \sL u = f
        \quad \quad \bx \in \Omega
    \]
    where $u=u(\bx)$ and $f=f(\bx)$ is bounded and $\Omega \subset\R^d$ is an open, bounded, connected set with sufficiently smooth boundary. $\sL$ is a linear differentiable operator
    \[
        \sL = \sum_{\abs{\alpha}<2\nu} c_{\alpha}(\bx) D^{\alpha}    
    \]
\end{definition*}

\begin{definition*}
    \bheading{Properties of linear differential operator} Given a linear differentiable linear operator $\sL$ and a bilinear form $\ta(\cdot,\cdot)$ such that $\ta(v,w) = \inner{\sL v}{w}$ for sufficiently smooth $v$ and $w$. Then $\sL$ is
    \begin{enumerate}
        \item \bheading{self-adjoint} if $\ta(v,w)=\ta(w,v)$ for all $v,w\in\mathring{\BH}$ 
        \item \bheading{elliptic} if $\ta(v,v) > 0$ for all $v\in\mathring{\BH} \neq 0$ 
        \item \bheading{positive definite} if it is self-adjoint and elliptic
    \end{enumerate}
    As an example, the following is a positive definite operator (genearlization of 2-point BVP in 9.1 and negative of Laplace operator $-\nabla^2$)
    \[
        \sL = - \sum_{i=1}^d \frac{\partial }{\partial x_i} \sum_{j=1}^d b_{i,j}(\bx) \frac{\partial }{\partial x_j}    
    \]
    where $(b_{i,j}(\bx))$ is symmetric positive definite for all $\bx\in\Omega$
\end{definition*}


\begin{theorem*}
    \bheading{Variational formulation of linear differentiable operator} Provided that the operator $\sL$ is positive definite. $\sL u = f$ is the Euler-Lagrange equation of the variational problem
    \[
        \sJ(v) = \ta(v,v) - 2\inner{f}{v}
        \quad\quad 
        v\in \BH    
    \]
    The weak solution of $\sL u = f$ is therefore, the \textbf{unique} minimum of $\sJ$ in $\BH$ (up to Lebesque zero)
\end{theorem*}

\begin{definition*}
    \bheading{Methods for solving $\sL u =f$} Choose $\varphi_0 \in \BH$ and let $\varphi_1,\cdots,\varphi_m$ span $\mathring{\BH}_m$. 
    \begin{enumerate}
        \item \bheading{Ritz method for $\sL u = f$} Seek $\boldsymbol{\gamma} \in \R^m$ that will minimize
        \[
            \sJ_m(\boldsymbol{\delta}) = \sJ(\varphi_0 + \sum_{l=1}^m \delta_l \varphi_l)
            \quad \quad 
            \boldsymbol{\delta} \in \R^m
        \]
        Setting $\partial \sJ_m / \partial \delta_l$ to zero for $l=1,\cdots,m$ results in a linear system,
        \[
            a_{k,l} = \ta(\varphi_k, \varphi_l)
        \]
        \item \bheading{Galerkin method} Seek $\boldsymbol{\gamma} \in \R^m$ that will satisfy
        \[
            \ta\p{
                \varphi_0 + \sum_{l=1}^m \gamma_l \varphi_l, \varphi_k
            } - \inner{f}{\varphi_k} = 0
            \quad \quad k = 1,2,\cdots, m
        \]
    \end{enumerate}
\end{definition*}

\begin{definition*}
    \bheading{Special case of Sobolev norm} Given any $v\in\BH$, let 
    \[
        \norm{v}_H = \p{
            \norm{v}^2 + \pb{\ta(v,v)}^{1/2}
        }    
    \]
    be a norm in $\BH$. $\ta$ is 
    \begin{enumerate}
        \item \bheading{bounded} if there exists $\delta>0$ such that $\abs{\ta(v,w)} \leq \delta \norm{v}_H \norm{w}_H$ for all $v,w\in H$
        \item \bheading{coercive} if there exists $\kappa > 0$ such that $\ta(v,v) \geq \kappa \norm{v}^2_H$ for every $v\in \BH$
    \end{enumerate}
\end{definition*}


\begin{theorem*}
    \bheading{Lax-Milgram theorem} Let $\sL$ be linear, bounded, and coercive and let $\BV$ be a  closed linear subspace of $\mathring{\BH}$. There exists a unique $\tu \in \varphi_0+\BV$ such that
    \[
        \ta(\tu, v) - \inner{f}{v} = 0
        \quad \quad v\in\BV    
    \]
    and the error is bounded
    \[
        \norm{\tu - u}_H \leq \frac{\delta}{\kappa} 
        \inf\pc{
            \norm{v-u}_H \mid v\in\varphi_0 + \BV
        }
    \]
    where $\varphi_0 \in \BH$ is arbitrary and $u$ is a weak solution of $\sL u =f $ in $\BH$
    \begin{enumerate}
        \item (observation) constant $\delta/\kappa$ is independent of choice of finite dimensional $\mathring{\BH}$ or of the norm $\norm{\cdot}$
        \item (observation) although the infimum on right hand side is unknown (since $u$ is not known), the distance of $u$ to $\varphi_0 + \mathring{\BH}_m$ can be bounded in terms of distance of an arbitrary member $w\in \varphi_0 + \mathring{\BH}$ from $\varphi_0 + \mathring{\BH}_m$
        \item (remark) error estimation in Galerkin method can be replaced by an appoximation problem: given a function $w\in \mathring{\BH}$ find the distance $\inf_{v\in\mathring{\BH}_m} \norm{w-v}_H$
    \end{enumerate}
\end{theorem*}


\section{\linkbook{227}{Spectral Method}}

\section{\linkbook{255}{Gaussian elimination for sparse linear equations}}

\section{\linkbook{273}{Classical iterative methods for sparse linear equations}}

\section{\linkbook{313}{Multigrid techniques}}

\section{\linkbook{331}{Conjugate Gradient}}




\end{document}
