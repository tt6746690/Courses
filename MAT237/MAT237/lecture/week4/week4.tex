\documentclass[11pt]{article}

\input{/Users/markwang/.preamble}

\begin{document}

\section*{Sequence and Completeness}

A \textbf{sequence} is an ordered collection of real numbers

$ $\\
\begin{defn*}
  Let $(x_n)_{n=1}^{\infty}$ be a sequence in $\R$. We say
  \begin{enumerate}
    \item $(x_n)$ is bounded if there exists an $M>0$ such that $|x_n| < M$ for every $n\in \N$
    \item $(x_n)$ is increasing if $x_{n+1} > x_n$ for every $n\in \N$
    \item $(x_n)$ converges with limit $L$, written as $(x_n)\xrightarrow{n\to \infty}L$, if for every $\epsilon > 0$ there exists an $N\in\N$ such that whenever $n>N$, $|x_n - L| < \epsilon$
  \end{enumerate}
\end{defn*}

\begin{proposition*}
  If $(x_n)_{n=1}^{\infty}$ is a sequence in $\R$ such that $(x_n)\to x$ and $(x_n)\to y$ then $x=y$; that is, \textbf{limits of convergent sequences} are \textbf{unique}.
  \begin{rem}
    There is also quite some limit laws for sequences on pg24
  \end{rem}
\end{proposition*}


\begin{theorem*}
  \textbf{Squeeze Theorem for Sequences}
  Let $(a_n), (b_n), (c_n)$ be sequences in $\R$, and assume that for sufficiently large $k$ we have $a_n \leq b_n \leq c_n$. If $(a_n)\ to L$ and $(c_n)\to L$, then $(b_n)\to L$
\end{theorem*}

\begin{theorem*}
  Every convergent sequences are bounded. The partial converse is the \textbf{Monotone Convergent Theorem} If $(a_n)$ is bounded from above and non-decreasing, then $(a_n)$ is convergent with its limit given by $\sup \{ a_n: n\in\N \}$
\end{theorem*}


$ $\\A sequence in $\R^m$ is any function $x: \N \to \R^m$. We write $x_n := x(n)$. If $k: \N \to \N$ then one can define a subsequence by $x(k(n)) := x_{k_n}$. If $(x_n)\in\R^m$ then $x_n = (x_n^1, \dots, x_n^m)\in \R^m$. A single component is therefore
\[
  (x_n^k)_{n=1}^{\infty} \text{ for arbitrary } k\in [1,\dots, m]
\]

\begin{defn*}
  Let $(x_n)_{n=1}^{\infty}$ be a sequence in $\R^n$. We say that $(x_n)$ converges with limit $x\in\R^n$, written as $(x_n)\to x$, if for every $\epsilon > 0$ there exists an $N\in\N$ such that whenever $n> N$ then $|| x_n - x|| < \epsilon$
  \begin{rem}
    Corresponding theorems about uniqueness of limits, the limit laws and the Squeeze Theorem all holds in $\R^n$ as well as $\R$.
  \end{rem}
\end{defn*}

\begin{theorem*}
  \textbf{Every subsequence of a convergent sequence converges to the same limit} Let $(x_n) \to x$ then every subsequence $x_{n_k} \xrightarrow{k\to \infty} x$
\end{theorem*}

\begin{proposition*}
   Let $(x_n)_{n=1}^{\infty} \in \R^m$ where $x-n = (x_n^1, \dots, x_n^m)$. Then $(x_n)$ converges if and only if $(x_n^i)$ converges for $i=1,\dots, m$
\end{proposition*}

\begin{proposition*}
   If $S\subseteq \R^n$, then $x\in \overline{S}$ if and only if there exists a convergent sequence $(x_n)$ in $S$ such that $(x_n) \to x$
\end{proposition*}

\begin{corollary*}
  \textbf{convergent sequence and closed set} A set $S\subseteq \R^n$ is closed if and only if whenever $(x_n)_{x=1}^{\infty}$ is a convergent sequence in $S$ with $(x_n)\to x$, then $x\in S$
\end{corollary*}


\section*{Completeness}

\begin{theorem*}
  Every bounded sequence in $\R$ has a convergent subsequence.
\end{theorem*}

\begin{defn*}
  A sequence $(x_n)_{x=1}^{\infty}\in \R^m$ is \textbf{bounded} if there exists $M > 0$ such that $||x_n || \leq M$ for every $n\in \N$
\end{defn*}

\begin{proposition*}
  Every bounded sequence in $\R^n$ has a convergent subsequence.
  \begin{rem}
    We prove this by taking convergent subsequence for each component, which is in $\R$, one at a time until we find limits for each individual component.
  \end{rem}
\end{proposition*}


\begin{defn*}
  A sequence $(x_n)_{n=1}^{\infty}$ is a \textbf{Cauchy Sequence} if for every $\epsilon > 0$ there exists a $N\in\N$ such that if $n,k>N$ then
  \[
    d(x_n, x_k) = ||x_n - x_k|| < \epsilon
  \]
  \begin{rem}
    Cauchy sequence gets closer together the further they travel into the sequence. Cauch sequence encapsulates basic behaviour of convergent sequence without a priori knowledge of the limit itself.
  \end{rem}
\end{defn*}

\begin{proposition*}
  Every Cauchy sequence is \textbf{bounded}
\end{proposition*}

\begin{proposition*}
  Every \textbf{convergent sequence} $(x_n) \to x$ is a \textbf{Cauchy sequence}. \textbf{The converse is true only in complete space} In summary, If $(x_n)_{n=1}^{\infty}$ is a sequence in $\R^n$, then $(x_n)$ is Cauchy if and only if $(x_n)$ is convergent.
  \begin{rem}
    The proof $\leftarrow$ involves proving that Cauchy sequence is bounded and therefore has a convergent subsequence. And claim that the sequence itself converges to the same limit. In a way, we can say that being \textbf{convergent} is stronger than being \textbf{cauchy}
  \end{rem}
\end{proposition*}

\begin{defn*}
  $S\subseteq \R^n$ is \textbf{complete} if every Cauchy sequence converges.
\end{defn*}

\section*{Continuity}

\begin{defn}
  \label{limit defn of continuity}
  \textbf{Limit definition of continuity} Let $f:\R^n \to\R^m$ with $c\in\R^n$ and $L\in\R^m$. we say
  \[
    \lim_{x\to c} f(x) = L
  \]
  if for every $\epsilon > 0$ there exists a $\delta > 0$ such that whenever $0< || x-c || < \delta$ then $|| f(x)-L || < \epsilon$ \\
  In other wards, the function $f$ is continuous if
  \[
    \lim_{x\to c} f(x) = f(c)
  \]
  If $f$ is continuous at every point in its domain, we say that $f$ is continuous
  \begin{rem}
    A limit exists iff the limit is the same regardless of the path taken to get to $c$. To prove that limit does not exists we simply take an arbitrary path and prove that limit converges to different points. To prove that limit is convergent, we use the Squeeze theorem.
  \end{rem}
\end{defn}


\begin{theorem}
  \label{Multivariable Squeeze Theorem}
  \textbf{Multivariable Squeeze Theorem} Let $f,g,h: \R^n \to \R$ be functions and $c\in\R^n$. Assume that in some neighborhood of $c$, such that $f(x)\leq g(x)\leq h(x)$ for all $x$ in that neighborhood. If
  \[
    \lim_{x\to c} f(x) = \lim_{x\to c} h(x) = L \text{   then   }  \lim_{x\to c}g(x) = L
  \]
  \begin{rem}
    \[
      -| f(x) | \leq f(x) \leq |f(x)| \text{ then } |f(x)|\to 0 \Rightarrow f(x)\to 0
    \]
  \end{rem}
\end{theorem}

\begin{theorem*}
  \label{Sequence definition of continuity}
  \textbf{Sequence Definition of continuity: maps convergent sequence to convergent sequence} A function $f:\R^n\to \R^m$ is continuous if and only if whenever $(a_n)_{n=1}^{\infty} \to a$ is a convergent sequence in $\R^n$, then $(f(a_n))_{n=1}^{\infty} \to f(a)$ is a convergent sequence in $\R^m$
\end{theorem*}

\begin{theorem*}
  \label{Topological definition of continuity}
  \textbf{Topological definition of continuity} A function $f:\R^n \to \R^m$ is continuous if and only if whenever $U\subseteq \R^m$ is an open set, then $f^{-1}(U) \subseteq \R^n$ is also an open set.
  \begin{proof}
    $ $\\
    $\Rightarrow$ Let $x\in f^{-1}(U)$, then $f(x)\in U$. Since $U$ open, then $B_{\epsilon}(f(x))\in U$. Since $f$ continuous, let $\delta$ for the correspondinfg $\epsilon$. We claim that $B_{\delta}(x)\subseteq f^{-1}(U)$. We prove this by taking a point $y\in B_{\delta}(x)$, hence $||y-x|| < \delta$. Then by continuity $||f(y)-f(x)|| < \epsilon$. Then $f(y)\in B_{\epsilon}(f(x))\subseteq U$. Then by definition of inverse functions $y\in f^{-1}(U)$. $B_{\delta}(x)\in U$ is true and thereore $f^{-1}(U)$ is open.\\
    $\Leftarrow$ Assume preimage of an open set is open for which we show $f$ is continuous. Let $\epsilon > 0$ be given. Let $U = B_{\epsilon}(f(x))$ then $x\in f^{-1}(U)$. Since open set, exists $\delta > 0$ such that $B_{\delta}(x) \subseteq f^{-1}(U)$. Let $y\in B_{\delta}(x)$ then $|| y - x|| < \delta$ Since $f(y) \in f(B_{\delta}(x)) \subseteq U = B_{\epsilon}$. Then $||f(y)-f(x)|| < \epsilon$ as required.
  \end{proof}
  \begin{rem}
    We can use topological definition to prove a set is open by finding a continuous function that transforms the set into an open set.
  \end{rem}
\end{theorem*}

\begin{theorem*}
  If $f:\R^n \to \R^m$ is continuous at $c$ and $g:\R^m \to \R^k$ is continuous at $f(c)$, then $g\circ f: \R^n \to \R^k$ is continuous at $c$.
\end{theorem*}

\begin{proposition*}
  $f:\R^n \to \R^m$ is continuous if and only if whenever $V\subseteq \R^m$ is closed then $f^{-1}(V)$ is closed.
\end{proposition*}


\begin{defn*}
  Let $D\subseteq \R^n$ and $f: D \to \R^m$, We say that $f$ is \textbf{uniformly continuous} if for every $\epsilon > 0$, there exists a $\delta > 0$ such that \textit{for every} $x,y \in D$ satisfying $|| x-y|| < \delta$ then $||f(x) - f(y)||<\epsilon$
  \begin{rem}
    Uniform continuity is a global condition whereas continuity is a local one. We are looking at pairs of points rather than single points in this case. The key point is that in uniform continuity the value of $\delta$ depends only on $\epsilon$ and not on the point in the domain - we can find a $\delta$ that works for every point $x$; whereas $\delta$ in the case continuity is specified after both $\epsilon$ and $x$ such that $\delta(\epsilon, x)$ is a function of both terms. \\
    For comparison sake, continuity
    \[
      \forall \epsilon > 0 \forall x\in D \exists \delta>0 \forall y\in D (||y-x|| < \delta \Rightarrow ||f(y) - f(x)|| < \epsilon)
    \]
    and uniform continuity
    \[
      \forall \epsilon > 0 \exists \delta>0 \forall x\in D \forall y\in D (||y-x|| < \delta \Rightarrow ||f(y) - f(x)|| < \epsilon)
    \]
    Thus for continuity at each point, one takes an arbitrary point $x$, and then there must exist a distance $\delta$; while for uniform continuity a single $\delta$ must work uniformly for all points $x$ (and $y$).\\
    To prove that a function is not uniformly continuous, assume function is uniformly continuous for contradiction. Choose arbitrary say $\epsilon = 1$ and pick a $\delta$ guaranteed by uniform continuity. Then choose $y$ as a function of $x$ such that precondition for uniform continuity is satisfied, i.e. $|x-y| < \delta$ and show that $| f(y) - f(x)| > 1 = \epsilon$. We usually find the appropriate edge case for $x, y$ by deriving backward to find a range of potential $x$ as a function of $\delta$

    \end{rem}
\end{defn*}

\begin{theorem*}
  If $D\subseteq \R^n$ is compact set and $f: D\to \R^m$ is continuous, then $f$ is uniformly continuous. More precisely, continuous functions with compact domains are uniformly continuous.
  \begin{rem}
    Using this theorem we can deduce that some continuous function is automatically uniformly continuous as long as domain is compact.
  \end{rem}
\end{theorem*}


\begin{theorem*}
  If $f$ is uniformly continuous, then if $(x_n)$ is cauchy, $(f(x_n))$ is also cauchy
\end{theorem*}

\end{document}
