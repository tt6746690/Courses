\documentclass[10.5pt]{article}
\input{\string~/.preamble}
\geometry{margin=0.5in}
\usepackage{mathrsfs} % https://www.ctan.org/pkg/mathrsfs

% arg1=pdfurl arg2=pagenum arg3=text
\newcommand{\linkbook}[3][../topology_2ed_james_munkres.pdf]{
    \noindent\href[page=#2]{#1}{\urlstyle{rm}{#3}}
}
% {ipad pagenum}-4 -= {.pdf pagenum}

\newcommand{\Z}{\mathbb{Z}}
\newcommand{\calA}{\mathcal{A}}
\newcommand{\calB}{\mathcal{B}}
\newcommand{\calC}{\mathcal{C}}
\newcommand{\calD}{\mathcal{D}}
\newcommand{\calE}{\mathcal{E}}
\newcommand{\calT}{\mathcal{T}}
\newcommand{\calP}{\mathcal{P}}
\newcommand{\ba}{\mathbf{a}}
\newcommand{\bx}{\mathbf{x}}
\newcommand{\by}{\mathbf{y}}
\newcommand{\Q}{\mathbbm{Q}}
\renewcommand{\norm}[1]{\left\lVert#1\right\rVert}
\DeclareMathOperator{\mdash}{{\hbox{-}}}


\begin{document}

\tableofcontents
\newpage


\begin{enumerate}
    \item \textbf{Functions}
    \begin{itemize}
        \item \textbf{injective} $f:X\to Y$ injective or one-to-one if $\forall a,b\in X$, $f(a) = f(b) \Rightarrow a=b$
        \item \textbf{surjective} $f:X\to Y$ is surjective if any $y \in Y \Rightarrow y = f(x)$ for some $x\in X$
        \item composition of injective/surjective/bijective functions are injective/surjective/bijective
        \item If $f$ is injective with range $Y$, then its inverse function $f^{-1}: Y\to X$ is a bijective function
    \end{itemize}
    \item \textbf{Set Relations}
    \begin{itemize}
        \item \textbf{De Morgan's Law}
        \[
            X \setminus \left( \bigcup_{\alpha \in I} A_{\alpha} \right)
            = \bigcap_{\alpha \in I} \left( X \setminus A_{\alpha} \right)
            \qquad 
            X \setminus \left( \bigcap_{\alpha \in I} A_{\alpha} \right)
            = \bigcup_{\alpha \in I} \left( X \setminus A_{\alpha} \right)
        \]
        \item \textbf{how functions acts on sets} Let $A,B\subseteq X$ and $C,D\subseteq Y$
        \begin{itemize}
            \item $f$ well-behaved for union
            \[
                f(A\cup B) = f(A) \cup f(B)    
            \]
            \item $f$ not well-behaved for intersection, difference
            \begin{align*}
                f(A\cap B) &\subseteq f(A) \cap f(B) \\
                f(A) \setminus f(B) &\subseteq f(A\setminus B) \\
            \end{align*}
            \item $f^{-1}$ well-behaved with union, intersection, difference, and set complement
            \begin{align*}
                f^{-1} (C\cup D) 
                &= f^{-1}(C) \cup f^{-1}(D) \\
                f^{-1} (C\cap D) 
                &= f^{-1}(C) \cap f^{-1}(D) \\
                f^{-1} (C\setminus D) 
                &= f^{-1} (C) \setminus f^{-1}(D) \\
                f^{-1} (Y\setminus C) 
                &= X\setminus f^{-1}(C)
            \end{align*}
            \item $f$ and $f^{-1}$ mixed 
            \begin{align*}
                A &\subseteq f^{-1}(f(A)) \tag{with equality if $f$ is injective} \\ 
                f(f^{-1}(C)) &\subseteq C  \tag{with equality if $f$ surjective}\\
            \end{align*}
        \end{itemize}
    \end{itemize}
\end{enumerate}

\begin{rem}
    Read
    \begin{enumerate}
        \item chapter 1, 1-8
        \item chapter 2, 12-22
        \item chapter 3, 23-29
        \item chapter 4. 30-35
    \end{enumerate}
\end{rem}

\newpage

\section{General Topology}

\subsubsection{\linkbook{31}{3 Relation}}

\begin{enumerate}
    \item A relation on a set $A$ is a subset $C$ of cartesian product $A\times A$. $xCy$ means $(x,y)\in C$
    \item \textbf{equivalence relation} is a relation if it satisfies reflexivity, symmetry, transitivity
    \item \textbf{equivalence class} a subset of $A$ determined by some $x\in A$, i.e. $E = \{y\mid y\sim x\}$
    \item \textbf{partition} of a set $A$ is a collection of disjoint nonempty subsets of $A$ whose union is all of $A$
    \item \textbf{order relation} is a relation if it satisfies comparability (any $x\neq y\in A$ either $xCy$ or $yCx$ but not both) nonreflexivity ($xCx$ does not hold for any $x\in A$) and transitivity
    \item \textbf{dictionary order relation} Let $A,B$ bet sets and $<_{A}$ and $<_{B}$ be order relations. The order relation on $A\times B$ is defined by $a_1 <_A a_2$ or if $a_1 = a_2$ and $b_1 <_B b_2$
    \item \textbf{least upper bound property} An ordered set $A$ has the property if every nonempty subset $A_0$ of $A$ that is bounded above ($\exists b\in A$ s.t. $x\leq b$ for all $x\in A_0$) has a least upper bound (all bounds of $A_0$ has a smallest element)
    \begin{itemize}
        \item $\R$ and $(-1,1)$ has least upper bounde property
        \item $B=(-1,0) \cup (0,1)$ does not heave least upper bound property, $\{ - \rfrac{1}{2n} \mid n\in \Z_{+} \}$ is bounded above by any $b\in (0,1)$ but its least upper bound $0 \not\in B$
    \end{itemize}
\end{enumerate}

\subsubsection{\linkbook{46}{5 Cartesian Product}}
\begin{enumerate}
    \item \textbf{indexed family of sets} Let $\calA$ be nonempty collection of sets, let $f:J\to \calA$ be a surjective indexing function. $(\calA, f)$ is called indexed family of sets, denoted by $\{\calA_{\alpha}\}_{\alpha \in J} = \{\calA_{\alpha}\}$ where $f(\alpha) = \calA_{\alpha}$
    \item \textbf{m-tuple} Let $m\in \Z_+$, Given a set $X$, define m-tuple of $X$ to be a function $\bx: \{1, \cdots, m\} \rightarrow X$ and denote $\bx = (x_1,\cdots, x_m)$
    \item \textbf{cartesian product} Let $\calA = \{A_1, \cdots, A_m\}$ be indexed family of sets, let $X = \cup_{i=1}^m A_i$. Then cartesian product of $\calA$ is 
    \[
        X^m = \prod_{i=1}^m A_i \qquad A_1 \times \dots \times A_m    
    \]
    to be the set of all m-tuples $\bx$ of elements of $X$ such that $x_i\in A_i$ for each $i$
    \item \textbf{$\omega$-tuple} Given a set $X$, define $\omega$-tuple of elements of $X$ be a function $\bx: \Z_+ \to X$. $\bx$ is an \textit{infinite sequence}, of elements of $X$. Denote $x_i = \bx(i)$ as i-th coordinate of $\bx$. Denote $\bx$ itself by $(x_1, x_2, \cdots)$ or $(x_n)_{n\in \Z_+}$
    \item \textbf{cartesian product (infinite)} Let $\calA = \{A_1, A_2, \cdots \}$ be indexed family of sets and $X$ be union of sets in $\calA$, the cartesian product of $\calA$
    \[
        X^{\omega} = \prod_{i\in \Z_+} A_i 
        \qquad
        A_1 \times A_2 \times \cdots    
    \]
    is defined to be the set of all $\omega$-tuples $(x_1,x_2,\cdots)$ of elements of $X$ such that $x_i\in A_i$ for each $i$
\end{enumerate}


\subsubsection{\linkbook{78}{11 The Maximum Principle}}

\begin{defn*}
    \textbf{(Strict Partial Ordering)} Given a set $A$, and a relation $\prec$ on $A$ is a strict partial order if 
    \begin{enumerate}
        \item (nonreflexivity) $a\prec a$ never holds
        \item (tranasitivity) If $a\prec b$ and $b\prec c$, then $a\prec c$
    \end{enumerate}
    Idea is not all $x,y \in A$ is comparable, i.e. either $x\prec y$ or $y\prec x$. Although a subset $S\subset A$ can be simply ordered.
\end{defn*}

\begin{theorem*}
    \textbf{(The Maximum Principle)} Let $A$ be a set and $\prec$ be a strict partial order on $A$. Then exists a maximal simply ordered subset $B$ of $A$, i.e. does not exsits $B'\subset A$ such that $B\subsetneq B'$ and $B'$ simply ordered.
\end{theorem*}

\begin{defn*}
    \textbf{(upper bound and maximal element)} Let $A$ be a set and $\prec$ be a strict partial order on $A$. If $B\subset A$, an \textbf{upper bound} on $B$ is $c\in A$ such that $b=c$ or $b\prec c$ for all $b\in B$. A \textbf{maximal element} of $A$ is an element $m\in A$ such that for no element $a\in A$ does  $m\prec a$ hold.
\end{defn*}

\begin{lemma*}
    \textbf{(Zorn's Lemma)} Let $A$ be a strictly partially ordered set. If every ordered subset of $A$ has an upper bound in $A$, then $A$ has a maximal element.
\end{lemma*}


\newpage

\section{Topological Spaces and Continuous Functions}

\subsubsection{\linkbook{85}{12 Topological Spaces}}

\begin{defn*} (\textbf{Topology})
    Topology on a set $X$ is a collection $\calT$ of subsets of $X$ having properties 
    \begin{enumerate}
        \item $\emptyset, X\in \calT$
        \item Arbitrary union of subcollection of $\calT$ is in $\calT$ (If $\forall \alpha \in I$, $U_{\alpha} \in \calT$, then $\cap_{\alpha\in I} U_{\alpha} \in \calT$)
        \item Finite intersection of subcollection of $\calT$ is in $\calT$ (If $\forall 1\leq i \leq n$, $U_i \in \calT$, then $\cap_{i=1}^n U_i \in \calT$)
    \end{enumerate}
    A topological space is a pair $(X,\calT)$, where $\calT$ are the open sets.
    \begin{itemize}
        \item Standard topology on $\R^n$ is $\calT_0 = \calT_{std} = \{U\subset \R^n \mid \forall x\in U, \exists \epsilon > 0 \, B_{\epsilon}(x)\subset U\}$
        \item Standard topology on $\R$ is generated by $\calB_{std} = \{(a,b) = \{x \mid a < x < b \}$ \textit{the open intervals}
        \item Lower limit topology on $\R$ is generated by $\calB_{l.l.} = \{x \mid a \leq x < b\}$ \textit{the half-open intervals}
        \item Discrete topology $\calT_1 = \calT_{disc} = \calP(X)$ \textit{all subsets are open}
        \item Trivial topology $\calT_2 = \calT_{triv} = \{\emptyset, X\}$ \textit{only empty set and $X$ are open}
        \item Finite complement topology $\calT_{f.c.} = \{U\subseteq X \mid X - U \text{ is finite or all of } X\}$
        \item Countable complement topology $\calT_{c} = \{U\subseteq X \mid X - U \text{ is countable or all of } X\}$
    \end{itemize}
\end{defn*}

\begin{lemma*}(\textbf{Arbitrary intersection of topologies is a topology})
    $\forall \alpha \in I$ $\calT_{\alpha}$ is a topology, so is $\cap_{\alpha\in I} \calT_{\alpha}$
\end{lemma*}

\begin{defn*} (\textbf{Compare topology})
    If $\calT' \subset \calT$, then $\calT'$ is coarser / weaker / smaller than $\calT$, $\calT$ is finer / stronger / larger than $\calT'$. $\calT$ and $\calT'$ are \textit{comparable} if $\calT \subset \calT'$ or $\calT' \subset \calT$
    \[
        \calT_{triv} \subset \calT_{f.c.} \subset \calT_{std} \subset \calT_{disc}
        \qquad
        \calT_{std} \subset \calT_{l.l.}
    \]
\end{defn*}


\subsubsection{\linkbook{88}{13 Basis for a Topology}}

\textit{A terser representation of $\calT$}

\begin{defn*} (\textbf{Basis})
    If $X$ is a set, a basis for $(X,\calT)$ is a collection $\calB$ (\textit{basis elements}) of subsets of $X$ s.t.
    \begin{enumerate}
        \item For each $x\in X$, exists at least one basis element $B\in \calB$ containing it
        \item If $x\in B_1 \cap B_2$, then exists $B_3\in \calB$ s.t. $x\in B_3 \in B_1 \cap B_2$
    \end{enumerate}
    A topology $\calT_{\calB}$ \textbf{generated by} $\calB$ is defined as 
    \[
        \calT_{\calB} = \{U\subset X \mid \forall x\in U,\,\, \exists B\in \calB, \,\, x\in B\subset U\}
    \]
    \begin{itemize}
        \item $\calT_{\calB}$ is the unique minimal topology containing $\calB$
        \[
            \calT_{\calB} = \textstyle \bigcap_{\calT \in \mathbbm{T}} \calT 
        \]
        where $\mathbbm{T} = \{\calT \mid \calT \supset \calB \text{ and } \calT \text{is a topology}\}$
        \item For any $X$, all one point sets of $X$ is a basis for $\calT_{disc}$
    \end{itemize}
\end{defn*}

\begin{lemma*} ($\calB \rightarrow \calT_{\calB}$)
    $\calT_{\calB}$ equals the collections of all unions of elements of $\calB$
    \[
        \calT_{\calB} = \left\{ 
            \textstyle \bigcup_{\alpha\in I} B_{\alpha} \mid B_{\alpha} \in \calB \quad \forall \alpha \in I   
        \right\}
    \]
\end{lemma*}

\begin{lemma*} ($\calT_{\calB} \rightarrow \calB$)
    Let $(X, \calT)$ be topological space. Let $\calC$ be a collection of open sets of $X$ such that
    \[
        \forall U\subset X,\quad \forall x\in U,\quad \exists C\in \calC \text{ s.t. } x\in C\subset U   
    \]
    Then $\calC$ is a basis for $\calT$ (handy in deciding $\calB_Y = \{B\cap Y \mid B\in \calB\}$ is a basis for $Y\subset X$)
\end{lemma*}

\begin{lemma*} (\textbf{compare topology by basis})
    Let $\calB$ and $\calB'$ be bases for $\calT$ and $\calT'$, respectively, on $X$. Then following equivalent
    \begin{enumerate}
        \item $\calT' \supset \calT$ ($\calT'$ is finer than $\calT$)
        \item For each $x\in X$ and each $B\in \calB$ containing $x$, there is a basis element $B'\in \calB'$ such that $x\in B' \subset \calB$
    \end{enumerate}
\end{lemma*}


\subsubsection{\linkbook{95}{14 Order Topology}}


\begin{defn*} (\textbf{Order Topology}) Let $X$ be a set with simple order relation. Let $\calB$ be a collection of all sets of the following type
    \begin{align*}
        \calB 
        &= \{(a,b) \mid a < b \quad a,b\in X \} \\
        &\bigcup \{ [a_0, b) \mid a_0 \text{ is minimal element (if any) of $X$ } b\in X \quad b\neq a_0 \} \\
        &\bigcup \{ (a,b_0] \mid b_0 \text{ is maximal element (if any) of $X$ } a\in X \quad a\neq b_0\} 
    \end{align*}
    Generated topology $\calT_{\calB}$ is called order topology
    \begin{itemize}
        \item In $\R$, $\calT_{ord} = \calT_{std}$
        \item In $\Z_+$, $\calT_{ord} = \calT_{disc}$ (since any $\{n\} = (n-1,n+1) \in \calT_{ord}$)
        \item In $\{1,2\} \times \Z_+$ in $\calT_{dict}$ is not in $\calT_{disc}$ (althogh most single point set are open, $2\times 1$ is not open)
        \item In $\R^2$, both $\calB$ and $\calB'$ generates $\calT_{dict}$
        \[
            \calB = \{ (a\times b, c\times d) \mid a < c \lor (a=c \land b < d) \}
            \quad
            \calB' = \{ (a\times b, a\times d) \mid b < d \}
        \]
    \end{itemize}
\end{defn*}

\subsubsection{\linkbook{97}{15 Product Topology}}

\begin{defn*} (\textbf{Product Topology})
    Let $X,Y$ be topological spaces, the product topology on $X\times Y$ is generated by the basis
    \[
        \calB = \{ U\times V \mid U\in \calT_X \; V\in \calT_Y\}    
    \]
    Alternatively define product topology with basis. If $\calB$, $\calC$ are basis for $X$ and $Y$ respectively. Then
    \[
        \calD = \{B \times C \mid B\in \calB \; C\in \calC \}    
    \]
    is a basis for topology of $X\times Y$
    \begin{itemize}
        \item $X\times \{ y \} \cong X$
        \item $X\times Y \cong Y\times X$
        \item $(X\times Y) \times Z \cong X \times (Y\times Z)$ 
        \item Product spaces does not work well with order topology.
        \begin{itemize}
            \item Consider $X=\R^2$ and $Y=[0,1] \times [0,1]$, then $\{0.5\} \times [0,1]$ is not open in $\calT_{ord}$ but is open in $\calT_{subspace}$
        \end{itemize}
    \end{itemize}
\end{defn*}

\begin{defn*}
    (\textbf{Projection}) Let $\pi_1 : X\times Y \to X$ be defined by $\pi_1(x,y) = x$. $\pi_1$ is a projection of $X\times Y$ \textbf{onto} the first factor. (note projections are surjective)
\end{defn*}

\begin{defn*}
    \textbf{(Product Topology by Continuity of Functions)} Given $X,Y$, there exists unique topology on $X\times Y$ such that 
    \begin{enumerate}
        \item projections $\pi_X$ and $\pi_Y$ are continuous
        \item If $f: Z \to X$ and $g: Z \to Y$, hen $f\times g: Z \to X\times Y$ is continuous
    \end{enumerate}
    \begin{proof}
        Define $\calB = \{ U\times V \mid U\subset X \text{ open }\; V\subset Y \text{ open } \}$. Show $\calT_{\calB}$ satisfies the above 2 conditions. Prove uniqueness by showing $id: (X, \calT') \to (X, \calT'')$ is a homeomorphism utilizing the above 2 conditions.
    \end{proof}
\end{defn*}

\subsubsection{\linkbook{98}{16 Subspace Topology}}

\begin{defn*}
    (\textbf{Subspace Topology}) Let $(X,\calT)$. Let $Y\subset X$, then 
    \[
        \calT_Y = \{Y\cap U \mid U \in \calT\}
    \]
    is the subspace topology on $Y$. Alternatively, define using basis. If $\calB$ generates $\calT$, then 
    \[
        \calB_Y = \{B\cap Y \mid B \in \calB \}    
    \]
    is a basis for $\calT_Y$. 
    \begin{itemize}
        \item \textbf{(lemma)} $Y\subset X$. $U$ open in $Y$ and $Y$ open in $X$, then $U$ open in $X$
        \item \textbf{(lemma)} A subspace of a subspace is a subspace
        \item \textbf{\underline{(theorem)}} A product of subspaces is a subspace of the product \textit{(subspace and product topology work well)}
        \item $X$ ordered and $Y\subset X$. order topology on $Y$ may not be same as order topology of $Y$ inherited as a subspace of $X$ (\textit{subspace and order topology does not work well})
        \begin{itemize}
            \item Let $Y=[0,1] \subset \R$. $\calB_Y$ are of the form $(a,b) \cap [0,1]$. Note
            \begin{enumerate}
                \item $[0,b)$ where $b\not\in [0,1]$ is open in $[0,1]$ but not in $\R$
                \item $\calT_{ord} \cong \calT_{subspace}$ since the basis elements of the same form
            \end{enumerate}
            \item Let $Y=[0,1) \cup \{2\} \subset \R$. 
            \begin{enumerate}
                \item $\{2\}$ open in $\calT_{subspace}$ since $(1.5,2.5) \cap Y = \{2\}$. 
                \item  $\{2\}$ not open in order topology since any basis of the form $\{x \in Y \mid a < x \leq 2 \; a\in Y\}$ contains some other point other than $\{2\}$
                \item $\calT_{ord} \not\cong \calT_{subspace}$
            \end{enumerate}
        \end{itemize}
        \item \textbf{\underline{(theorem)}} \textit{subspace and order topology works well if the subspace is convex}
    \end{itemize}
\end{defn*}

\begin{defn*}
    \textbf{(convex)} Given ordered $X$ and subset $Y\subset X$ is convex if $(a,b) \subset X$ lies in $Y$ completely
\end{defn*}

\begin{theorem*} \textbf{(subspace and order topology works well if the subspace is convex)}  \\
    Let $X$ be ordered and $Y\subset X$ be convex. Then order topology on $Y$ same as topology $Y$ inherits as a subspace of $X$.
\end{theorem*}

\begin{defn*}
    \textbf{(Subspace Topology by Continuity of Functions)} Let $X$ be topological space, $Y\subset X$, there exists unique topology on $Y$ such that 
    \begin{enumerate}
        \item inclusion $i_Y : Y\hookrightarrow X$ is continuous
        \item If $f:Z\to Y$ such that $i_Y \circ f : Z \to X$ is continuous, then $f$ is continuous
    \end{enumerate}
\end{defn*}


\subsubsection{\linkbook{101}{17 Closed Sets and Limit Points}}

\begin{defn*}
    \textbf{(Closed)} A subset $A$ of $X$ is closed if $X - A$ is open.
    \begin{itemize}
        \item In $\calT_{f.c.}$, all finite subsets and $X$ are closed
        \item In $\calT_{disc}$ every set is closed.
        \item In $Y = [0,1] \cup (2,3)$, $[0,1]$ and $(2,3)$ are open and closed in subspace topology of $Y$
        \item $C$ closed in $Y$ does not imply $C$ is closed in $X$. However a closed set in a closed subspace is closed overall, i.e. in $X$ (Let $Y$ be a subspace of $X$. If $A$ is closed in $Y$ and $Y$ is closed in $X$, then $A$ is closed in $X$. )
    \end{itemize}
\end{defn*}

\begin{theorem*}
    \textbf{(Topology by closed sets)}
    \begin{enumerate}
        \item $\emptyset$ and $X$ are closed
        \item Arbitrary intersection of closed sets are closed
        \item Finite unions of closed sets are closed
    \end{enumerate}
\end{theorem*}

\begin{theorem*}
    \textbf{(Closedness in Subspace)} Let $Y$ be a subspace of $X$. Then $A$ is closed in $Y$ if and only if it equals the intersection of a closed set $X$ with $Y$ ($Y\subset X$, then $A\subset Y$ closed in $Y$ if exists $K\subset X$ s.t. $A=K\cap Y$)
\end{theorem*}


\begin{defn*}
    \textbf{(Closure and Interior)} If $A\subset X$
    \begin{enumerate}
        \item \textbf{Interior} $Int_X A = \mathring{A}$ is
        \begin{itemize}
            \item the union of all open sets in $X$ contained in $A$, i.e. $Int_X A = \cup_{U\in \calT_X : U\subset A} U$
            \item maximal open subests of $A$ in $X$
        \end{itemize}
        \item \textbf{Closure} $Cl_X A = \overline{A}$ is
        \begin{itemize}
            \item the intersection of all closed sets containing $A$, i.e. $Cl_X A = \cap_{U\in \calT_X : U\supset A} U$
            \item minimal closed set containing $A$
        \end{itemize}
    \end{enumerate}
    \begin{itemize}
        \item Set relationship
        \[
            Int A \subset A \subset \overline{A}    
        \]
        \item \underline{(theorem)}. Let $A\subset Y\subset X$, Let $\overline{A}$ denote closure of $A$ in $X$. Then the closure of $A$ in $Y$ is $\overline{A}\cap Y$
        \begin{itemize}
            \item In $\R$, $Y=(0,1]$, let $A=(0,0.5) \subset Y$. $Cl_{\R} A = [0, 0.5]$, $Cl_{Y} A = Cl_{\R} A \cap Y = (0, 0.5]$
        \end{itemize}
        \item If $A$ open, then $Int(A) = A$; If $A$ closed, then $\overline{A} = A$
        \item If $A\subset X$, then $(\mathring{A})^c = \overline{(A^c)}$ (\textit{Complement of interior is closure of the complement})
        \item In $\R$, let $A = \Q$ or $A = \R - \Q$, $int(A) = \emptyset$ and $\overline{A} = \R$ ($int(A)=\emptyset$ since no $(a,b)$ contained fully in $\Q$ or $\R-\Q$)
    \end{itemize}
\end{defn*}

\begin{defn*}
    \textbf{(neighborhood)} $U$ is an open set containing $x$ is equivalent to $U$ is a neighborhood of $x$ 
\end{defn*}

\begin{defn*}
    \textbf{(intersects)} $A$ intersects $B$ if and only if $A\cap B\neq \emptyset$
\end{defn*}

\begin{theorem*} \textbf{(define closure using neighborhoods)}
    Let $A$ be a subset of $X$,
    \begin{enumerate}
        \item then $x\in \overline{A}$ if and only if every neighborhood of $x$ intersects $A$
        \item Let $\calB$ be basis of $X$, then $x\in \overline{A}$ if and only if every basic neighborhood $B$ of $x$ intersects $A$
    \end{enumerate}
    proof by contraposition. Following are examples which uses this theorem to test/determine the closure
    \begin{itemize}
        \item If $A=(0,1]$ then $\overline{A}=[0,1]$ (since every neighborhood of $\{0\}$ intersects $A$)
        \item If $B = \{1/n \mid n\in \Z_+\}$, then $\overline{B} = \{0\} \cup B$
        \item If $C=\{0\} \cup (1,2)$ then $\overline{C} = \{0\} \cup [1,2]$
        \item In $\R$, $\overline{\Q} = \R$ since every neighborhood of $x\in \R$ contains some rational number, so intersects $\Q$
        \item In $\Z_+$, $\overline{\Z_+} = \Z_+$
    \end{itemize}
\end{theorem*}

\begin{defn*}
    \textbf{(Limit Point)} Let $A\subset X$ and $x\in X$, $x$ is a limit point of $A$ if every neighborhood of $x$ intersects $A$ in some point other than $x$ itself. In other words,
    \[
        x \in \overline{A - \{x\}}
    \]
    \begin{itemize}
        \item In $\R$, $A=(0,1]$, then any $x\in [0,1]$ is a limit point of $A$ and no other point in $\R$ is a limit point
        \item In $\R$, $B = \{1/n \mid n\in \Z_+\}$, 0 is the only limit point of $B$ (any other $x\in \R$ has neighborhood that does not intersect $B$ or intersects at $x$ itself)
        \item In $\R$, $C=\{0\} \cup (1,2)$, all $x\in [1,2]$ are limit points of $C$
        \item In $\R$, every $x\in \R$ is a limit point of $\Q$
        \item In $\R$, no point is a limit point of $\Z_+$
    \end{itemize}
\end{defn*}

\begin{theorem*}
    \textbf{(define closure using limit point)} Let $A\subset X$, let $A'$ be set of all limit points, then
    \[
        \overline{A} = A\cup A'    
    \]
    \begin{itemize}
        \item (corollary) $A\subset X$ is closed if and only if it contains all its limit points, i.e. $A' \subset A$
    \end{itemize}
\end{theorem*}

\begin{rem}
    ways to prove $A$ is closed 
    \begin{enumerate}
        \item show $A^c$ is open
        \item show $\overline{A} = A$ by proving that every $x\in A^c$ has open neighborhood that does not intersect $A$
        \begin{itemize}
            \item $A=\{x_0\} \subset \R$ closed since every point different from $x_0$ has neighborhood not intersecting $\{x_0\}$
        \end{itemize}
    \end{enumerate}
\end{rem}

\begin{defn*}
    \textbf{(Sequence Convergence)} A sequence $(x_n)$ converges to a point $x\in X$, denoted $x_n \to x$ or $\lim_{n\to \infty} x_n = x$ if 
    \[
        \forall \text{ neighborhood } U \text{ of } x\; \exists N\in \N \text{ such that } \forall n\geq N \; x_n \in U
    \]
    For metric spaces
    \[
        \forall \epsilon > 0 \; \exists N\in \N \; \forall n\geq N \; |x_n - x| < \epsilon
    \]
\end{defn*}

\begin{defn*}
    \textbf{(Separated)} Let $x,y\in X$, $x$ and $y$ can be separated if each lies in a neighborhood which does not contain the other point. (neighborhood not necessarily disjoint)
\end{defn*}

\begin{defn*}
    \textbf{($T_1$ Space)} $X$ is $T_1$ if any two distinct points in $X$ are separated.
    \begin{itemize}
        \item \underline{(theorem)} Let $A\subset X$ $T_1$. $x\in A'$ if and only if every neighborhood of $x$ contains infinitely many points of $A$.
        \item \underline{(theorem)} Every finite point set, specifically one-point set, in a $T_1$ space is closed
    \end{itemize}
\end{defn*}

\begin{defn*}
    \textbf{($T_2$ Hausdorff Space)} 
    A topological space $X$ is called Hausdorff space if for each pair $x_1,x_2$ of distinct points of $X$, there exists neighborhoods $U_1$ and $U_2$ of $x_1$ and $x_2$, respectively that are disjoint.
    \[
        \forall x \neq y \in X \; \exists \text{ neighborhoods } U, V \text{ of x and y respectively}  \text{ s.t. } U\cap V = \emptyset
    \]
    \begin{itemize}
        \item \textbf{(motivation)} Generally, one point set not always closed; Sequences converges can converge to more than limit. 
        \item \textbf{\underline{(theorem)}} Every finite point set, specifically one-point set, in a Hausdorff space is closed
        \item \textbf{\underline{(theorem)}} If $X$ is $T_2$, then a sequence of points of $X$ converges to at most 1 point of $X$
        \item \textbf{\underline{(theorem)}} Every simply ordered set is $T_2$ in the order topology. Product of two $T_2$ space is $T_2$; Subspace of a $T_2$ space is $T_2$ (order/product/subspace topology well behaved with $T_2$)
        \item \textbf{(examples)}
        \begin{itemize}
            \item $\R^n_{std}$, $X_{disc}$ are $T_2$
            \item $X_{triv}$ not $T_2$ except when $|X_{triv}| =1$
            \item $X_{f.c.}$ not $T_2$ when $X$ is infinite (since any $x,y\in X_{f.c.}$ are infinite and intersects)
        \end{itemize}
    \end{itemize}
\end{defn*}


\subsubsection{\linkbook{112}{18 Continuous Functions}}

\begin{defn*}
    \textbf{(Continuous)} A function $f:X\to Y$ is continuous if each open subset $V\subset Y$, the set $f^{-1}(V)$ is open. Alternatively formulated with basis, $f$ is continuous if every basis element $B\in \calB$, $f^{-1}(B)$ is open
    \begin{itemize}
        \item $id:\R_{std} \to \R_{l.l}$ is not continuous; $id:\R_{l.l.} \to \R_{std}$ is continuous
    \end{itemize}
\end{defn*}


\begin{theorem*}
    \textbf{(TFAE for continuous function)} $f:X\to Y$
    \begin{enumerate}
        \item $f$ is continuous
        \item For every subset $A$ of $X$, $f(\overline{A}) \subset \overline{f(A)}$ (convergence: $x_n \to x \Rightarrow f(x_n)\to f(x)$ to $x\in \overline{A} \Rightarrow f(x)\in \overline{f(A)}$)
        \item For every closed set $B$ of $Y$, the set $f^{-1}(B)$ is closed
        \item (generalized $\epsilon$-$\delta$) $\forall x\in X$ and $\forall$ neighborhood $V$ of $f(x)$, $\exists$ neighborhood $U$ of $x$ such that $f(U)\subset V$
    \end{enumerate}
    \begin{itemize}
        \item In metric space, 4 can be reformulated with $\epsilon$-$\delta$ definition. A function $f:\R^n \to \R^m$ is continuous if 
        \begin{align*}
            \forall x_0 \in \R^n, \quad \forall \epsilon > 0 \quad \exists \delta > 0 \text{ s.t. } & |x-x_0|<\delta \Rightarrow |f(x)-f(x_0)|<\epsilon \\ 
            \text{ i.e. } & (x\in B_{\delta}(x_0) \Rightarrow f(x)\in B_{\epsilon}(f(x_0)))
        \end{align*}
    \end{itemize}
\end{theorem*}

\begin{defn*}
    \textbf{(Homeomorphism)} Let $f:X\to Y$ be a bijection If $f$ and the inverse function $f^{-1}:Y\to X$ are continuous, then $f$ is called a homeomorphism. (continuous bijection)
    \begin{itemize}
        \item \underline{(theorem)} If $X\cong Y$, then $X$ and $Y$ share topological property, i.e. property expressed in terms of topology only.
        \item (example) $(-1,1) \cong \R$ since $f: (0,1) \to \R$ by $f(x) = \tan(x)$ is a homeomorphism
        \item (example) A function can be continuous but not homeomorphic. Consider $S^1 = \{x\times y\mid x^2 + y^2 = 1\}\subset \R^2$ the unit circle. Let $f:[0,1) \to S^1$ by $f(t) = (\cos(2\pi t), \sin(2\pi t))$. $f$ is bijective and continuous but $f^{-1}$ not continuous
        \item If $\calT_1 \subset \calT_2$, then $Id: (X,\calT_2) \to (X,\calT_1)$ is continuous
        \item If $\calT_1 = \calT_2$, then $Id: (X, \calT_1) \to (X,\calT_2)$ is a homeomorphism
    \end{itemize}
\end{defn*}


\begin{defn*}
    \textbf{(Imbedding)} An injective map $f:X\to Y$ is a topological imbedding of $X$ in $Y$ if $f':X\to Z$ is a homeomorphism (note the image set $Z = f(X)$ carries subspace topology inherited from $Y$)
    \begin{itemize}
        \item Intuitively, an imbedding $f:X\to Y$ let us treat $X$ as a subspace of $Y$.
        \item \underline{(theorem)} Every map that is injective, continuous, and either open or closed is am imbedding.
        \item (example) $f:[0,1) \to \R$ by $f(t) = (\cos(2\pi t), \sin(2\pi t))$ maps to $S^1$. $f$ is a continuous injective map but not am imbedding.
    \end{itemize}
\end{defn*}


\begin{theorem*}
    \textbf{(Constructing Continuous Functions)} Given $X,Y,Z$
    \begin{enumerate}
        \item (constant function) If $f:X\to Y$ by $f(X) = \{y_0\}$, then $f$ is continuous
        \item (inclusion) If $A\subset X$ with subspace topology, inclusion $i_A:A\hookrightarrow X$ is continuous
        \item (composition) If $f:X\to Y$ and $g:Y\to Z$ are continous, then $g\circ f: X\to Z$ is continuous
        \item (restricting domain) If $f:X\to Y$ is continous and $A\subset X$, then $f|_A : A\to Y$ is continuous ($f|_A = f \circ i_A$)
        \item (restricting or expanding range) Let $f:X\to Y$ continuous. If $f(X)\subset A\subset Y \subset B$, then $g:X\to A$ obtained by restricting range of $f$ is continuous. $h:X\to B$ obtained by expanding range of $f$ also continuous ($h = i_Y \circ f$)
        \item (local formulation of continuity) $f:X\to Y$ is continuous if $X = \cup_{\alpha\in I} U_{\alpha}$ such that $f|_{U_{\alpha}}$ is continuous for each $\alpha \in I$
        \item (map to products) Let $f:A\to X\times Y$ given by $f(t) = (f_1(t), f_2(t))$ and $f_1:A\to X$, $f_2:A\to Y$. Then $f$ continuous if and only if $f_1$ and $f_2$ are continuous
        \item (Algebraic Operations) If $f,g:X\to \R$ continous, then $f+g$, $f-g$, $f \cdot g$, $f/g$ ($g(x) \neq 0$) all continous 
        \item (Uniform Limit Theorem) If a sequence of continuous real-valued function of a real variable converges \textbf{uniformly} to a limit function, then the limit function is continuous 
    \end{enumerate}
\end{theorem*}

\begin{theorem*}
    \textbf{(the pasting lemma)} Let $X = A\cup B$, where $A$ and $B$ are both closed or both open in $X$. Let $f:A\to Y$ and $g:B\to Y$ be continuous. If $f(x) = g(x)$ for every $x\in A\cap B$, set $h:X\to Y$
    \[
        h(x) = 
        \begin{cases}
            f(x) & x\in A \\
            g(x) & x\in B \\ 
        \end{cases}    
    \]
    then $h$ is continuous
    \begin{itemize}
        \item $h(x) = x$ for $x\leq 0$ and $h(x) = x/2$ for $x\geq 0$ is continous
    \end{itemize}
\end{theorem*}


\subsubsection{\linkbook{122}{18 The Product Topology}}

\begin{defn*}
    \textbf{(J-tuple)} Let $J$ be index set. Define \textbf{J-tuple} of elements of $X$ be a function $\bx: J \to X$. Given  $\alpha\in J$, denote \textbf{$\alpha$th coordinate} as $\bx$ at $\alpha$ by $x_{\alpha}$ instead of $\bx(\alpha)$. Denote $\bx$ as $(x_{\alpha})_{\alpha \in J}$. Denote the \textbf{set of all $J$-tuples} of elements of $X$ by $X^J$
\end{defn*}

\begin{defn*}
    \textbf{(Cartesian Product)} Let $\{A_{\alpha}\}_{\alpha\in J}$ be an indexed family of sets; Let $X = \cup_{\alpha\in J} A_{\alpha}$. The cartesian product of this indexed family is given by 
    \[
        \prod_{\alpha\in J} A_{\alpha} = 
        \{
            (x_{\alpha})_{\alpha \in J}\in X \mid x_{\alpha} \in A_{\alpha}
        \}
        = \{
            \bx: J \to \bigcup_{\alpha \in J} X \mid \forall \alpha\in J \; \bx(\alpha) \in A_{\alpha}
        \}
    \]
    is the set of all J-tuples $(x_{\alpha})_{\alpha \in J}$ of elements of $X$ such that $x_{\alpha} \in A_{\alpha}$ for each $\alpha\in J$. Equivalently, the set of all functions $\bx: J \to \cup_{\alpha \in J} X$ such that $\bx(\alpha) = A_{\alpha}$ for each $\alpha\in J$
\end{defn*}

\begin{defn*}
    \textbf{(Projection)} The projection mapping associated with index $\beta$ is defined by
    \[
        \pi_{\beta} = \prod_{\alpha\in J} X_{\alpha} \to X_{\beta}
        \qquad
        \pi_{\beta}((x_{\alpha})_{\alpha\in J}) = x_{\beta}
    \]
\end{defn*}

\begin{defn*}
    \textbf{(Box Topology)} Let $\{X_{\alpha}\}_{\alpha\in J}$ be indexed family of topological spaces. Basis
    \[
        \calB_{box} = 
        \left\{
            \prod_{\alpha\in J} U_{\alpha} \mid U_{\alpha} \text{ open in } X_{\alpha} \; \forall \alpha \in J
        \right\}    
    \]
    generates box topology
    \begin{itemize}
        \item \underline{(theorem)} Given basis $\calB_{\alpha}$ for each $X_{\alpha}$, $\textstyle \prod_{\alpha\in J} \calB_{\alpha}$ where $\textstyle B_{\alpha}\in \calB_{\alpha}$ is a basis for $\textstyle\prod_{\alpha\in J} X_{\alpha}$
    \end{itemize}
\end{defn*}

\begin{defn*}
    \textbf{(Product Topology)} Let $\{X_{\alpha}\}_{\alpha\in J}$ be indexed family of topological spaces. Basis
    \begin{align*}
        \calB_{prod}
        &= \left\{
            \pi_{\beta_1}^{-1}(U_{\beta_1}) \cap \cdots \cap \pi_{\beta_n}^{-1}(U_{\beta_n}) \mid U_{\beta_i} \in \calT_{X_{\beta_i}} \; \beta_1 ,\cdots, \beta_n \in J
        \right\} \\
        &= \left\{
            \prod_{\alpha\in J} U_{\alpha} \mid \forall \alpha \in J \; U_{\alpha} \in \calT_{X_{\alpha}} \text{  for almost all } \alpha \; U_{\alpha} = X_{\alpha}
        \right\}
    \end{align*}
    generates product topology
    \begin{itemize}
        \item \underline{(theorem)} Let $\textstyle f:A\to \prod_{\alpha\in J} X_{\alpha}$ defined by $f(a) = (f_{\alpha}(a))_{\alpha\in J}$ where $f_{\alpha}: A\to X_{\alpha}$ for each $\alpha$, Then $f$ is continuous if and only if each function $f_{\alpha}$ is continuous
        \item \underline{(theorem)} For finite products, $\calT_{box} = \calT_{prod}$
        \item (example where product topology works while box topology does not) In $\R^{\omega} = \textstyle \prod_{n\in \Z_+} \R$, countably infinite product of $\R$. Define $f:\R\to \R^{\omega}$ by $f(t) = (t,t,\cdots)$. Each $f_{\alpha} = \pi_{\alpha} \circ f$ continuous so $f$ continuous in product topology. However $f$ not continuous in box topology. Consider $B = \{ (-1/n, 1/n) \mid n\in \Z_+ \} \in \calB_{box}$ open,
        \[
            f^{-1}(B) 
            = \{ t \mid (t,t,\cdots) \in B \}
            = \bigcap_{n\in \Z_+} (-\frac{1}{n}, \frac{1}{n})
            = \{0\}
        \]
        not open in $\R$
    \end{itemize}
\end{defn*}

\begin{defn*}
    \textbf{(product topology by continuity of functions)} There is a unique topology $\calT_{prod}$ on $\textstyle X = \prod_{\alpha\in J} X_{\alpha}$ with
    \begin{enumerate}
        \item $\pi_{\beta}: X \to X_{\beta}$ continuous for every $\beta\in J$
        \item Given $\textstyle f:Z \to \prod_{\alpha\in J} X_{\alpha}$. If $f_{\alpha} = \pi_{\alpha} \circ f$ is continuous for every $\alpha$, then $f$ is continuous
    \end{enumerate}
    Note $\calB_{prod}$ satisfies the two condition
    \begin{proof}
        1. by design. proof for 2 as follows 
        \[
            f^{-1}\left(\bigcap_{i=1}^n \pi_{\beta_i}^{-1}(U_{\beta_i})\right)
            = \{z\in Z \mid \forall i \; f_{\beta_i} \in U_{\beta_i}\}
            = \bigcap_{i=1}^n f^{-1}_{\beta_i} (U_{\beta_i})
        \]
        is open as finite intersection of open sets
    \end{proof}
\end{defn*}

\begin{theorem*}
    \textbf{(Both box and product topology works well with subspace/$T_2$/closure)}
    \begin{enumerate}
        \item Let $A_{\alpha}\subset X_{\alpha}$. Then $\textstyle \prod A_{\alpha}$ is a subspace of $\textstyle \prod X_{\alpha}$
        \item If $X_{\alpha}$ is $T_2$, then $\textstyle \prod X_{\alpha}$ is $T_2$
        \item Let $A_{\alpha}\subset X_{\alpha}$, then $\textstyle \prod \overline{A_{\alpha}} = \overline{\prod A_{\alpha}}$
    \end{enumerate}
    \begin{proof}
        \textbf{proof of 2.} Let $\bx = (x_{\alpha})$ and $\by = (y_{\alpha})$ where $\bx \neq \by$. So exists $\beta\in J$ such that $x_{\beta} \neq y_{\beta}$. Since $X_{\beta}$ is $T_2$, exsits neighborhoods $U_{\beta},V_{\beta}$ for $x_{\beta}$ and $y_{\beta}$ s.t. $U_{\beta} \cap V_{\beta} = \emptyset$. Note $\pi^{-1}(U_{\beta})$ and $\pi^{-1}(V_{\beta})$ are disjoint neighborhoods of $\bx,\by$. \textbf{proof of 3.} Mainly use the definition of closure using neighborhoods. ($\Rightarrow$) Let $\bx = (x_{\alpha}) \in \textstyle \prod \overline{A_{\alpha}}$. Need to show $\bx \in \overline{\textstyle \prod A_{\alpha}}$. Let $U = \textstyle \prod U_{\alpha}$ be basic neighborhood of $\bx$ in $\calT_{box}$ or $\calT_{prod}$. For each $\alpha$, since $x_{\alpha}\in A_{\alpha}$, can find $y_{\alpha} \in U_{\alpha} \cap A_{\alpha}$. Note $\by = (y_{\alpha}) \in U \cap \textstyle \prod A_{\alpha}$. Since $U$ arbitrary,$\bx \in \overline{\textstyle\prod A_{\alpha}}$. ($\Leftarrow$) Let $\bx = (x_{\alpha}) \in \textstyle \overline{\prod A_{\alpha}}$. Want to show $x_{\beta}\in \overline{A_{\beta}}$ for all $\beta$. Let $V_{\beta}$ be arbitrary neighborhood of $x_{\beta}$. Consider $\pi^{-1}(V_{\beta})$, which is open in both $\calT_{box}$ and $\calT_{prod}$. By definition of closure, exsits $\by = (y_{\alpha}) \in \pi^{-1}(V_{\beta}) \cap \textstyle \prod A_{\alpha}$. Hence $y_{\beta}\in V_{\beta} \cap A_{\beta}$. Hence, $x_{\beta}\in \overline{A_{\beta}}$
    \end{proof}
\end{theorem*}


\subsubsection{\linkbook{129}{19 The Metric Topology}}


\begin{defn*}
    \textbf{(Metric)} A metric on $X$ is a function
    \[
        d: X\times X \to \R    
    \]
    with properties
    \begin{enumerate}
        \item (non-negativity) $d(x,y)\geq 0$ for all $x,y\in X$, wth equality if $x=y$
        \item (symmetry) $d(x,y) = d(y,x)$ for all $x,y\in X$
        \item (triangle inequality) $d(x,y) + d(y,z) \geq d(x,z)$ for all $x,y,z\in X$
    \end{enumerate}
\end{defn*}

\begin{defn*}
    \textbf{($\epsilon$-ball centered at $x$)}
    \[
        B_d(x, \epsilon) = \{y \mid d(x,y) < \epsilon\}
    \]
\end{defn*}

\begin{defn*}
    \textbf{(norm)} Given $\bx = (x_1,\cdots, x_n)\in \R^n$, the norm of $\bx$ defined by $\norm{x} = (x_1^2 + \cdots + x_n^2)^{1/2}$.
\end{defn*}

\begin{defn*}
    \textbf{(Metric Topology)}
    Given $X$ and a metric $d$, basis 
    \[
        \calB_{d} = 
        \{
            B_d(x,\epsilon) \mid x\in X \; \epsilon > 0
        \}
    \]
    generates the metric topology $\calT_d$ induced by $d$. Therefore 
    \[
        \calT =
        \left\{
            U\subset X \mid \forall x\in U \; \exists \epsilon > 0 \; B(x,\epsilon) \subset U
        \right\}
    \]
    \begin{enumerate}
        \item \textbf{discrete metric} $d_{disc}$, defined by $d_{disc}(x,y)=1$ if $x\neq y$ and $d_{disc}(x,y)=0$ if $x=y$ induces $\calT_{disc}$
        \item \textbf{standard metric} on $\R$ defined by $d(x,y) = |x-y|$ induces $\calT_{ord}$
        \item \textbf{(diamond)} $d_1 = \textstyle \sum_i |x_i - y_i|$
        \item \textbf{euclidean metric (circle)} on $\R^n$, $d_2 = d(\bx, \by) = \norm{x-y} = \sqrt{(x_1-y_2)^2 + \cdots + (x_n - y_n)^2}    $
        \item \textbf{square metric (square)} on $\R^n$, $d_{\infty} = \rho(\bx, \by) = max \{ |x_1 - y_1| ,\cdots , |x_n - y_n| \}$ (furthest coordinate within $\epsilon$)
        \item \underline{(theorem)} In $\R^n$, $\calT_{d_1} = \calT_{d_2} = \calT_{\infty}$ induces same topology $\calT_{std}$ (proof by showing basis elements nests!)
        \item \underline{(theorem)} If $X$ is metrizable, then $X$ is $T_2$
        \item \underline{(theorem)} Subspaces of metric space behaves well $d|_{A\times A}$ induces subspace topology for $A\subset X$.
    \end{enumerate}
\end{defn*}


\begin{defn*}
    \textbf{(Metrizable)} If $X$ is topological space, $X$ is \textbf{metrizable} if there exists metric $d$ that induces the topology of $X$. A \textbf{metric space} is a metrizable space $X$ together with a specific metric $d$ that gives the topology of $X$
    \begin{itemize}
        \item Metrizability is a topological property
        \item (not every topology comes with a metric) consider $X_{triv}$ where $|X|\geq 2$, $X$ not metrizable since it's not $T_2$
        \item $\R^n$ is metrizable ($d_1,d_2,d_{\infty}$ induces $\R^n_{std}$)
        \item $\R^{\omega}$ is metrizable under product topology but not box topology
        \item $\R^J$ where $J$ uncountable is not metrizable
        \item \underline{(theorem)} countable products of metrizable spaces is metrizable
        \item (by sequence lemma) sequences are sufficient to describe metrizable spaces
    \end{itemize}
\end{defn*}


\begin{defn*}
    \textbf{(Bounded and Diameter)} Given $(X,d)$, $A\subset X$ is \textbf{bounded} if there is some $M$ such that
    \[
        d(a_1,a_2) \leq M    
    \]
    for all $a_1,a_2\in A$. The \textbf{diameter} of $A$ is defined
    \[
        diam\, (A, d) = \sup\{d(a_1,a_2) \mid a_1,a_2\in A\}    
    \]
    \begin{itemize}
        \item boundedness is not a topological property, since it depends on a specific $d$ (consider $d$ and $\overline{d}$)
    \end{itemize}
\end{defn*}

\begin{defn*}
    \textbf{(Standard Bounded Metric)} Given $(X,d)$, define $\overline{d}: X\times X\to \R$ by 
    \[
        \overline{d}(x,y) = \min \{ d(x,y), 1\}    
    \]
    to be the standard bounded metric corresponding to $d$
    \begin{itemize}
        \item \underline{(theorem)} $d$ and $\overline{d}$ induces the same topology, i.e. $\calT_{d} = \calT_{\overline{d}}$
        \item (trick) by above theorem, we can say \underline{$diam(A) \leq 1$ without loss of generality} by replacing $d$ with $\overline{d}$
    \end{itemize}
\end{defn*}

\begin{defn*}
    \textbf{(Uniform Metric and Uniform Topology)} Generalize square metric to $\R^{J}$. Given $\bx = (x_{\alpha})_{\alpha\in J}$, $\by = (y_{\alpha})_{\alpha\in J} \in \R^J$. Define metric $\overline{\rho}$ by
    \[
        \overline{\rho}(\bx, \by) = \sup \{ \overline{d}(x_{\alpha}, y_{\alpha}) \mid \alpha\in J\}
    \]
    is called \textbf{uniform metric} on $\R^J$ inducing \textbf{uniform topology}.
\end{defn*}

\begin{theorem*}
    \textbf{(Relationship of Topologies on $\R^J$)}
    \[
        \calT_{prod} \subset \calT_{uniform} \subset \calT_{box}    
    \]
    where the three topologies are all different if $J$ is infinite.
\end{theorem*}

\begin{theorem*}
    \textbf{(Metric inducing product topology)} If $\bx, \by \in \R^{\omega}$, define
    \[
        D(x,y) = \sup \left\{
            \frac{\overline{d}(x_i, y_i)}{i}
        \right\}   
    \]
    is a metric that induces product topology on $\R^{\omega}$
    \begin{proof}
        Show $D$ is a metric. Then show $D$ gives product topology. Now show $\calT_D = \calT_{prod}$. To show $\calT_{prod}$ is finer, show there exists basis element in product topology that contains in a basis element of the metric topology. Let $B_D(\bx, \epsilon)$ be basic neighborhood of $\bx$. Let $N$ be such that $1/N < \epsilon$. Consider $V\subset \calT_{prod}$ defined as
        \[
            V = (x_1 - \epsilon, x_1 + \epsilon) \times \cdots \times (x_N - \epsilon, x_N + \epsilon) \times \R \times \R \cdots
        \]
        Show $V\subset B_D(\bx, \epsilon)$. Note for any $\by \in \R^{\omega}$, $\overline{d}(x_i, y_i) / i \leq 1/N$ for $i\geq N$. Therefore, 
        \[
            D(\bx, \by) = \sup \left\{
                \frac{\overline{d}(x_1, y_1)}{1}, \cdots, \frac{\overline{d}(x_N, y_N)}{N}, \frac{1}{N}
            \right\}    
        \]
        If $\by \in V$, then $\overline{d}(x_i, y_i) < \epsilon$ for all $i < N$. So $D(\bx, \by) < \epsilon$. Hence $V\subset B_D(\bx, \epsilon)$. Conversely, want to show $\calT_D$ is finer. The key here is recognize that product topology $\textstyle \prod U_i$ where each component is metrizable and induced by $d$. Let $U = \textstyle \prod_{i\in I} U_i \times \prod_{i\not\in I} \R$ be a basis element in $\calT_{prod}$ where $I$ is finite. Let $\bx \in U$, want to find a basic neighborhood $V \in \calT_{D}$ such that $\bx \in V \subset U$. For each $i\in I$, find an interval $(x_i - \epsilon_i, x_i + \epsilon_i)$ such that $i \epsilon < \epsilon_i$ and $\epsilon \leq 1$. We can achieve this by setting $\epsilon = \min \{\epsilon_i /i \mid i\in I\}$. Now we claim that $\bx \in B_D(\bx, \epsilon)\subset U$. Let $\by \in B_D(\bx, \epsilon)$, then for all $i$,
        \[
            \frac{\overline{d}(x_i, y_i)}{i} \leq D(\bx, \by) < \epsilon    
        \]
        We need to show that $\by \in U$. We only care about $i\in I$ since $y_i \in \R$ for all $i\not\in I$. When $i\in I$, $\overline{d}(x_i, y_i) < i \epsilon < \epsilon_i \leq 1$. Therefore, $d(x_i, y_i) < \epsilon_i$ implying $\by \in \textstyle U$
    \end{proof}
\end{theorem*}


\subsubsection{\linkbook{138}{21 The Metric Topology Continued}}


\begin{defn*}
    \textbf{(Continuity in Metric Spaces)} Let $f:X\to Y$ where $(X,d_X)$ and $(Y,d_Y)$ are metrizable. Then $f$ continuous if and only if $\forall x\in X$ and $\forall \epsilon > 0$, $\exists \delta > 0$ such that 
    \[
        d_X(x,y) < \delta \;\Rightarrow \; d_Y(f(x), f(y)) < \epsilon
    \]
\end{defn*}

\begin{defn*}
    \textbf{(almost always)} means all but finitely many
\end{defn*}

\begin{defn*}
    \textbf{(Convergence)} $(x_n) \to x$ if for all neighborhood $U$ of $x$, almost always $x_n \in U$.
\end{defn*}

\begin{defn*}
    \textbf{(Uniform Convergence)} Let $f_n:X\to Y$ be a sequence of functions where $Y$ is metrizable. Let $d$ be metric for $Y$. The sequence $(f_n)$ \textbf{converges uniformly} to the function $f:X\to Y$ if given $\epsilon > 0$, there exists $N > 0$ such that 
    \[
        d(f_n(x), f(x)) < \epsilon
    \]
    for all $n > N$ and all $x\in X$.
    \begin{itemize}
        \item depends on $\calT_Y$ and metric $d$
        \item stronger than point-wise convergence
        \item (uniform limite theorem) Given $f_n:X\to Y$ where $Y$ metrizable. If $(f_n)$ converges uniformly to $f$, then $f$ is continous.
    \end{itemize}
\end{defn*}


\begin{defn*}
    \textbf{(Sequantial Closure)} Given $A\subset X$, sequential closure is given by 
    \[
        seq\mdash cl(A) = \{x\in X \mid \exists\; (x_n) \to x \; x_n \in A \}
    \]
    \begin{itemize}
        \item (fact) $A\subset seq\mdash cl(A) \subset \overline{A}$
        \item (the sequence lemma) $seq\mdash cl(A) \subset cl(A)$ and with equality if $X$ is metrizable.
    \end{itemize}
\end{defn*}

\begin{lemma*}
    \textbf{(The sequence lemma)} Let $X$ be a topological space and let $A\subset X$. If there is a sequence of points of $A$ converging to $x$, then $x\in \overline{A}$. The converse holds if $X$ is metrizable.
    \begin{proof}
        ($\Rightarrow$) If $x_n \to x$ where $x_n\in A$. Let $U$ be any neighborhood of $x$. By definition of convergent sequence, $\exists N$ such that $\forall n\geq N$, $x\in U$. Since $U$ arbitrary, every neighborhood of $x$ contains some $x_n \in A$, hence $x\in \overline{A}$. Conversely, let $d$ be metric inducing topology of $X$. Consider neighborhood $B_d(x,1/n)$ for all $n\in \Z_+$, pick $n$-th term for the sequence as $x_n \in B_d(x,1/n) \cap A$. We claim that $(x_n)_{n\in \Z_+}$ is convergent. Indeed, take $B_d(x,\epsilon)$ be arbitrary basic neighborhood of $x$, take $N>0$ such that $1/N < \epsilon$. Therefore, for all $i\geq N$, $x_i \in B_d(x,\epsilon)$ by construction.
    \end{proof}
\end{lemma*}

\begin{theorem*}
    \textbf{(Sequence Continuity)} Let $f:X\to Y$. If $f$ continuous, then every convergent sequence $x_n \to x$ in $X$, the sequence $f(x_n) \to f(x)$. The converse is true if $X$ is metrizable.
\end{theorem*}

\begin{defn*}
    \textbf{(First countability axiom)} A space $X$ that has a countable basis at each point satisfies first countability axiom. A space $X$ said to have \textbf{countable basis at the point} $x$ if there is a countable collection $\{U_n\}_{n\in \Z_+}$ of neighborhoods of $x$ such that any neighborhood $U$ of $x$ contains at least one of the sets $U_n$.
    \begin{itemize}
        \item used to prove the above lemma/theorem; metrizability is not necessary.
    \end{itemize}
\end{defn*}

\begin{defn*}
    \textbf{Spaces that are not metrizable}
    \begin{enumerate}
        \item $\R^{\omega}$ in box topology is not metrizable
        \item $\R^{J}$ where $J$ uncountable is not metrizable in product topology
    \end{enumerate}
    \begin{proof}
        Generally, to show that a space $X$ is not metrizable, we can show that the space does not satisfy the sequence lemma. Specifically, if $X$ is metrizable, then $seq\mdash cl(A) = cl(A)$, to prove $X$ not metrizable, find $A\subset X$ and $x\in X$ such that $x\in cl(A)$ and $x\not\in seq\mdash cl(A)$
        \textbf{(Point 1)}
        Consider $A\subset \R^{\omega}$ be points with only positive coordinate values, i.e $A = \{(x_1,x_2,\cdots) \mid x_i > 0 \; i\in \Z_+\}$. Now we claim that $\mathbf{0} = (0, 0, \cdots)$ is in $cl(A)$ but not $seq\mdash cl(A)$. $\mathbf{0}\in cl(A)$: any neighborhood $B = (a_1,b_1) \times (a_2,b_2) \times \cdots$ of $\mathbf{0}$ intersects $A$ at point $(\rfrac{1}{2}b_1, \rfrac{1}{2}b_2,\cdots)$. Now we show there is no sequence of $A$ converging to $\mathbf{0}$. Assume for contradiction that $(\bx_n) \to \mathbf{0}$ where $\bx_n = (x_{n,k})_{k=1}^{\infty}$. Let $U = \textstyle \prod_n (-1, x_{n,n}) \subset \calT_{box}$ be a neighborhood of $\mathbf{0}$. However not almost always $\bx_n \in U$: in fact $U$ contains no elements of the sequence $(\bx_n)$ since $n$-th coordinate $x_{n,n} \not\in (-1, x_{n,n})$. Contradicts assumption that $(\bx_n) \to \mathbf{0}$. Therefore $\mathbf{0} \not\in seq\mdash cl(A)$.
        \textbf{(Point 2)} Let $A = \{(x_{\alpha}) \mid x_{\alpha} = 1\text{ except for finitely many } \alpha \in I\}$. Let $\mathbf{0} \in \R^J$ be origin. Now we show $\mathbf{0} \in cl(A)$. Let $U = \textstyle \prod U_{\alpha}$ be neighborhood of $\mathbf{0}$ where $U_{\alpha} \neq \R$ for all $\alpha \in \{\alpha_1, \cdots, \alpha_n\}$ in product topology. Now we show $U\cap A \neq \emptyset$. Consider $\by = (y_{\alpha}) \in \R^J$ where $y_{\alpha} = 0$ for all $\alpha\in \{\alpha_1, \cdots, \alpha_n\}$ and $y_{\alpha} = 1$ otherwise. Note $\by \in A$ since all but finitely many $y_{\alpha}$ is 1; $\by \in U$ since $y_{\alpha} \in U_{\alpha}$ for $\alpha_1, \cdots, \alpha_n$ and $y_{\alpha} \in \R = U_{\alpha}$ otherwise. Hence $\by \in U\cap A$ and therefore $\mathbf{0} \in cl(A)$. Now we show $\mathbf{0} \not\in seq\mdash cl(A)$. Let $\ba_n$ be a sequence of $A$. Let $J_n \subset J$ such that $\ba_n(\alpha) \neq 1$ for all $\alpha \in J_n$. $J$ finite by of $A$. Let $J' = \textstyle \cup_{n\in \Z_+} J_n$. Note $J'$ is a countable union of finite set and hence countable. Since $J$ uncountable, exists $\beta\in J$ such that $\beta \not\in J'$ and therefore every point in the sequence $\ba_n (\beta) = 1$ for all $n\in \Z_+$. Let $U_{\beta} = (-1,1) \in \R$. Consider a neighborhood $U = \pi^{-1}(U_{\beta}) \subset R^J$ of $\mathbf{0}$ that contains no points in $\ba_n$. Therefore $\ba_n$ does not converge to $\mathbf{0}$. Therefore $\mathbf{0} \not\in seq\mdash cl(A)$.
    \end{proof}
\end{defn*}

\subsubsection{\linkbook{146}{22 The Quotient Topology}}

\begin{defn*}
    \textbf{(Quotient Map)} Given $X,Y$ and $p:X\to Y$ be a surjective map. $p$ is a quotient map if a subset $U\subset Y$ is open if and only if $p^{-1}(U)$ is open in $X$.
    \begin{itemize}
        \item \underline{(theorem)} $p$ is a surjective continuous map that is either open or closed, then $p$ is a quotient map
        \item (example) projection maps $\pi_1 : X\times Y \to X$ is surjective, continuous, open and therefore a quotient map. However $\pi_1$ is not a closed map (since $\pi_1(\{x\times y\mid xy=1\}) = \R - \{0\}$ not closed)
    \end{itemize}
\end{defn*}

\begin{defn*}
    \textbf{(Quotient Topology and Quotient Space)} Let $X$ be a space and $A$ be a set. If $p:X\to A$ is a surjective map, then there uniquely exists one topology $\calT$ on $A$ relative to which $p$ is a quotient map; $\calT$ is called the \textbf{quotient topology} induced by $p$.
    \[
        \calT = \{U \subset Y \mid p^{-1}(U) \in \calT_X \}    
    \]
    As a special case. Givecn $\sim$ be an equivalence relation on $X$ and $Y=\rfrac{X}{\sim} = \{ \{y:y\sim x_0\} \mid x_0 \in X \} \subset \calP(X)$ are equivalence classes of $X$. Then $p:X \to Y$ exists and is a surjection. The space $Y$ with the quotient topology is called a \textbf{quotient space} of $X$.
\end{defn*}

\begin{defn*}
    \textbf{define quotient topology with continuity of functions} Given topological space $X$ and $\pi:X\to Y$ a surjection, there is a unique topology on $Y$ satisfying 
    \begin{enumerate}
        \item $\pi: X\to Y$ continuous
        \item If $Z$ is a topological space, $g:Y\to Z$ is a function. If $g \circ \pi$ is continuous, then $g$ is continuous.
    \end{enumerate}
\end{defn*}

\newpage

\section{Connectedness and Compactness}

\textit{theorems about continuous functions}
\begin{itemize}
    \item (intermediate value theorem) If $f:[a,b]\to \R$ continuous and $f(a)<r<f(b)$, then exists $c\in [a,b]$ s.t. $f(c)=r$
    \item (maximum value theorem) If $f:[a,b]\to \R$ continuous then exists $c\in [a,b]$ such that $f(x)\leq f(c)$ for all $x\in [a,b]$
    \item (uniform continuity theorem) If $f:[a,b]\to \R$ continuous, then given $\epsilon>0$ exists $\delta>0$ such that $|f(x_1)-f(x_2)|<\epsilon$ for every pair $x_1,x_2\in [a,b]$ for which $|x_1-x_2|<\delta$
\end{itemize}

\subsubsection{\linkbook{156}{23 Connected Spaces}}

\begin{defn*}
    \textbf{(Separation and Connected)} Given topological space $X$. A \textbf{separation} of $X$ is a pair $U,V$ where $U\cap V=\emptyset$ and $X=U\cup V$ and $U,V$ both nonempty. The space $X$ is \textbf{connected} if there does not exists a separation of $X$. Equivalently, $X$ is connected if the only clopen sets are $\emptyset$ and $X$.
    \begin{itemize}
        \item (fact) connectedness is a topological property
        \item \underline{(theorem)} Image of a connected space under a continuous map is connected
        \item \underline{(theorem)} Finite product $\textstyle \prod X_{\alpha}$ connected if and only if $X_{\alpha}$ connected for all $\alpha$
    \end{itemize}
\end{defn*}

\begin{defn*}
    \textbf{(Separation and Connected for subspaces)} Given $Y\subset X$. A \textbf{separation} of $Y$ is a pair of disjoint nonempty sets $A$ and $B$ whose union is $Y$, neither of which contains a limit point of the other (i.e. $cl_Y(A) \cap B = \overline{A} \cap B = \emptyset$). The space $Y$ is connected if there exits no separation of $Y$
    \begin{proof}
        Suppose $A,B$ forms a seperation of $Y$. Then $cl_Y(A) = \overline{A}\cap Y$. Since $A$ closed in $Y$, $A=cl_Y(A) = \overline{A}\cap Y$. Since $A,B$ disjoint, $\overline{A} \cap B = \emptyset$. Since $\overline{A}$ contains all its limit point, $B$ has no limit points of $A$. Conversely, given assumption we have $\overline{A} \cap B = \emptyset$ and $A\cap \overline{B} = \emptyset$. Hence $\overline{A} \cap Y = A$ and $\overline{B} \cap Y = B$. Then $A,B$ are closed in $Y$. Since $A,B$ partitions $Y$, $A,B$ are open in $Y$ as well.
    \end{proof}
    \begin{itemize}
        \item (examples)
        \begin{itemize}
            \item $\{a,b\}$ with $\calT_{triv}$ is connected
            \item $Y=[-1,0)\cup (0,1] \subset \R$ is connected since $[-1,0)$ and $(0,1]$ is a seperation. ($0$ is a limit point to both, but does not matter since $0$ is not contained in $[-1,0)$ or $(0,1])$
            \item $X=[-1,1]\subset\R$. $[-1,0]$ and $(0,1]$ is not a seperation ($0$ is a limit point of $(0,1]$ but contained in $[-1,0]$)
            \item $\Q$ not connected. Only one point subsets of $\Q$ are connected.
            \item $X = \{(x,y) \mid y=0\} \cup \{(x,y) \mid y = 1/x \} \subset \R^2$ not connected (neither contain a limit point of each other)
        \end{itemize}
        \item (lemma) If $C,D$ forms a separation of $X$ and $Y$ is a connected subspace of $X$, then $Y$ lies entirely in $C$ or $D$
        \item \underline{(theorem)} Union of connected subspaces of $X$ with a commont point is connected  \\ 
        ($\textstyle \cap A_{\alpha} \neq \emptyset$ where $A_{\alpha}$ connected for all $\alpha$ then $\textstyle \cup A_{\alpha}$ connected)
        \item \underline{(theorem)} If $A\subset X$ be a connected subspace. If $A\subset B \subset \overline{A}$, then $B$ also connected
    \end{itemize}
\end{defn*}

\begin{defn*}
    \textbf{(Connectedness for Infinite Products)}
    \begin{enumerate}
        \item $\R^{\omega}$ is not connected in box topology
        \item $\R^{\omega}$ is connected in product topology
    \end{enumerate}
    \begin{proof}
        \textbf{(Point 1)} Enough to find a separation of $\R^{\omega}$. Interprete $\R^{\omega}$ as the collection of all real numbered sequences and is partitioned by the set of all bounded sequences of real numbers $A$ and the set of all unbounded sequences of real numbers $B$. Now we show $A,B$ both open. Consider $\ba \in \R^{\omega}$, we can find a neighborhood of $\ba$ in the box topology by $U = (a_1-1,a_1+1) \times (a_2-1,a_2+1) \times \cdots$. If $\ba$ is bounded, $U$ consists of only bounded sequencs so $\ba\in U\subset A$. If $\ba$ is unbounded, then $U$ consists of only unbounded sequences and $\ba\in U\subset B$. Therefore $\R^{\omega}$ not connected in box topology. 
        \textbf{(Point 2)} To show $\R^{\omega}$ in product topology is connected, we find some connected $C\subset \R^{\omega}$ where $C\subset \R^{\omega} \subset \overline{C}$ and use the lemma to show that $\R^{\omega}$ is connected. Consider $\tilde{\R}^n \subset \R^{\omega}$ defined to be the set of all sequences fixed to 0 beyond $n$: $\bx \in \R^{\omega}$ such that $x_i > 0$ for all $i>n$. Since $\tilde{\R}^n \cong \R^n$ and $\R^n$ is connected, $\tilde{\R}^n$ is connected. Since each $\tilde{\R}^n$ is connected and $\textstyle \cap_{n} \tilde{\R}^n = \{0, 0, \cdots\} = \mathbf{0}$, we have  $\R^{\infty} = \textstyle \cup_{n} \tilde{\R}^n$ connected. To complete the proof, we show $\overline{\R^{\infty}} = \R^{\omega}$. Consider $\ba = (a_1,a_2,\cdots) \in \R^{\omega}$ and $U=\textstyle \prod U_{\alpha}$ be neighborhood of $\ba$ in box topology. We show $U\cap \R^{\infty} \neq \emptyset$. Let $N$ be such that $U_i = \R$ for all $i>N$. Consider $\bx = \{a_1,a_2,\cdots, a_N, 0, 0, \cdots\} \in \R^{\infty}$. $\bx \in U$ since $x_i \in U_i$ for all $i\leq N$ and $x_i \in \R = U_i$ for all $i>N$. Therefore $\bx \in U\cap \R^{\infty}$ hence $\overline{\R^{\infty}} = \R^{\omega}$
    \end{proof}
\end{defn*}


\subsubsection{\linkbook{162}{24 Connected Subspaces of the Real Line}}

\begin{defn*}
    \textbf{(convex)} $Y\subset X$ is convex if every $a<b\in Y$, $[a,b] \in Y$
\end{defn*}

\begin{defn*}
    \textbf{(Linear Continuum)} A simply ordered set $L$ having more than 1 eleent is called a linear continuum if 
    \begin{enumerate}
        \item $L$ has least upper bound property
        \item If $x<y$, there exists $z$ such that $x<z<y$
    \end{enumerate}
    \begin{itemize}
        \item (fact) condition for connectedness to hold on $\R$
        \item \underline{(theorem)} If $L$ is linear continuum in order topology, then $L$ is connected, and so are intervals and rays in $L$
        \item \underline{(theorem)} If $Y\subset \R$, then $Y$ is connected if and only if $Y$ is convex and nonempty
        \item (corollary) $\R$, intervals and rays in $\R$ are all connected
    \end{itemize}
\end{defn*}

\begin{theorem*}
    \textbf{(Unit interval in $I=[0,1] \subset \R$ is connected)}
    \begin{proof}
        Idea is to show any separation $(A,A^c)$ gives $A = I$. Let $A \subset I$ be clopen. without loss of generality let $0\in A$. Define $G = \{x\in I \mid [0,x] \in A \}\subset A$ and $g = \sup G$. Goal is to show $1 = g \in G$ such that $A = G$ and therefore $I - A = \emptyset$ which contradicts assumption of separation. \textbf{First} we show $g>0$, note $0\in A$ where $A$ is open, hence $[0,\epsilon)\in A$ and therefore $\epsilon/2 \in G$. So $g=\sup G \geq \epsilon/2 > 0$. \textbf{Second} we show $g \not< 1$. We first show $g\in A$. Since $G\subset A$, we have $\overline{G} \subset \overline{A}=A$. Then $g = \sup G = \overline{G} \in A$. Hence $g\in A$. Since $A$ open, can find $(g-\epsilon, g+\epsilon)\in A$. Easily, $[0, g + \epsilon/2] \in A$ and hence $g+\epsilon/2 \in G$ which contradicts $g =\sup G$. \textbf{Third} we show $g\in G$. Same as before we show $g\in A$. Can find open neighborhood $(1-\epsilon, 1] \in A$. Easily $[0,1] \in A$ hence $A=I$. To conclude $(A,A^c)$ is not a separation. 
    \end{proof}
\end{theorem*}


\begin{theorem*}
    \textbf{(Intermediate Value Theorem)} Let $f:X\to Y$ be continuous map, where $X$ is connected and $Y$ is in order topology. If $a,b \in X$ and $r\in Y$ such that $f(a) <  r < f(b)$, then there exists $c\in X$ such that $f(c) = r$.
    \begin{proof}
        Let $A = f(X) \cap (-\infty, r)$ and $B = f(X) \cap (r, \infty)$. Note $A\cap B=\emptyset$ and neither are empty since $f(a)\in A$ and $f(b)\in B$. Note $A,B \subset f(X)$ are open in the subspace topology by definition. If no $c\in X$ such that $f(c) = r$, then $f(X) = A \cup B$ then $(A,B)$ constitutes a separation of $f(X)$, contradicting the fact that image of a connected space under a continuous map $f(X)$ is connected.
    \end{proof}
\end{theorem*}

\begin{defn*}
    \textbf{(Path Connectedness)} Let $x,y\in X$, a \textbf{path} in $X$ from $x$ to $y$ is a continuous map $f:[a,b] \to X$ such that $f(a) = x$ and $f(b) = y$. A space $X$ is \textbf{path-connected} if every pair of points in $X$ can be joined by a path.
    \[
        \forall x,y\in X \; \exists \text{ continuous } f:[0,1] \to X \; f(0)=x \; f(1)=y
    \]
    \begin{itemize}
        \item \underline{(theorem)} continuous image of path-connected space is path-connected
        \item \underline{(theorem)} path-connected space is connected. (converse not always true: see topologist's sine curve)
        \item (proposition) connectedness and path-connected subsets of $\R$ are the same
        \item \underline{(theorem)} If $X_{\alpha}$ path-connected, then $\textstyle \prod X_{\alpha}$ is also path-connected (in product topology) 
        \item (example) unit ball $B^n = \{\bx \mid \norm{\bx} \leq 1 \} \subset \R^n$ is path-connected ($f:[0,1]\to \R^n \; t \to (1-t)\bx + t\by$ for any $\bx,\by \in \R^n$)
        \item (example) punctured euclidean space $\R - \{0\}$ is path-connected
        \item (example) unit sphere $S^{n-1} \{\bx \mid \norm{\bx} = 1\} \subset \R^n$ is path-connected ($g: \R^n \setminus \{0\} \to S^{n-1} \; \bx \to \bx / \norm{\bx}$ is continuous)
        \item (example) Let $S = \{x\times \sin (1/x) \mid 0 < x \leq 1 \} \subset \R^2$. The \textbf{topologist's sine curve} $\overline{S}$ is connected but not path-connected
        \[
            \overline{S} = \left(\{0\} \times [-1,1]\right) \;\bigcup\; \left\{x\times \sin (1/x) \mid 0 < x \leq 1 \right\} \subset \R^2
        \]
    \end{itemize}
\end{defn*}


\begin{theorem*}
    \textbf{(Topologist's sine curve is connected but not path-connected.)}
    \begin{proof}
        Let $S'=\left(\{0\} \times [-1,1]\right)$ and let $S= \left\{x\times \sin (1/x) \mid 0 < x \leq 1 \right\} \subset \R^2$ and hence $\overline{S}=S'\cup S$. Since $S$ is connected and $\overline{S}$ is image of a connected set $(0,1]$ under continuous map, $\overline{S}$ is also connected. Now we show $\overline{S}$ is not path-connected. Let $f:[a,c]\to \overline{S}$, $a<0$ be a path connecting $(0,0)$ and $(1,0)$. Since $S'$ is closed, $f^{-1}(S')$ is closed and has a largest element $b$. Therefore $f':[b,c]\to \overline{S}$ where $f'$ maps $b$ to $S'$ and rest of points to $S$. Replace $[b,c]$ with $[0,1]$ for convenience. Let $f(t)=  (x(t), y(t))$ which has to be continuous. Then $x(0) = 0$ and $x(t) > 0$ and $y(t) = \sin (1/x(t))$ for $t>0$. There exists $t_n$ a sequence such that $y(t_n) = (-1)^n$ does not converge, contradicting continuity of $f$. We construct $t_n$ as follows. For each $n$, pick $u$ in range $0<u<x(1/n)$ such that $\sin (1/u) = (-1)^n$. Use intermediate value theorem to find $t_n$ suc that $x(t_n) = u$.
    \end{proof}
\end{theorem*}


\subsubsection{\linkbook{172}{26 Compact Spaces}}

\begin{defn}
    \textbf{(Cover and Compact)} A collection $\calA \subset\calP(X)$ \textbf{cover} $X$, or be a \textbf{covering} of $X$, if $\textstyle \cup_{A\in \calA} A = X$. $\calA$ is open cover, if it is a cover and all $A\in \calA$ are open. A space $X$ is said to be \textbf{compact} if every open cover $\calA$ contains a finite subcollection that also covers $X$, i.e. if $\{A_{\alpha}\}$ is an open cover, exists $I = \{\alpha_1,\cdots,\alpha_n\}$ such that $\textstyle \cup_{\alpha\in I} A_{\alpha} = X$ 
    \begin{itemize}
        \item (examples)
        \begin{itemize}
            \item $\R$ is not compact ($\calA = \{(n, n+2) \mid n\in \Z_+\}$ does not have a finite subcover)
            \item $(0,1]$ (and similarly $(0,1)$) is not compact ($\calA = \{ (1/n,1]\mid n\in \Z_+\}$ emits no finite cover)
            \item $X = \{0\} \cup \{1/n \mid n\in \Z_+\} \subset \R$ is compact. (idea: $U \in \calA$ covering 0 covers all but finitely many points of $1/n$. Take $U$ and $U_{i}\in \calA$ for all $1/i$ not in $U$ forms a finite cover for $X$)
            \item $X$ with finitely many points is compact (all covers are finite)
            \item $[0,1]$ is compact 
        \end{itemize}
        \item \underline{(theorem)} Image of a compact space under a continuous map is compact
        \item \underline{(theorem)} Let $f:X\to Y$ be bijective continuous map. If $X$ is compact and $Y$ is $T_2$, then $f$ is a homeomorphism (idea: $K\subset X$ closed. Since $X$ compact, $K$ also compact, $f(K)$ is then compact, which is closed in $T_2$ space $Y$)
        \item \underline{(theorem)} Product of finitely many compact spaces is compact (tube lemma)
        \item \underline{(theorem)} Product of infinitely many compact spaces in product topology is compact (Tychnoff theorem)
        \item \underline{(theorem)} A continuous function $f:X\to \R$ where $X$ is compact is bounded. (idea: $X = \cup_{i=1}^{\infty} f^{-1}((-\epsilon_i,\epsilon_i))$, by compactness $X = \cup_{i=1}^n ((-\epsilon_i, \epsilon_i)) = f^{-1}((-M, M))$ where $M=\max_{i=1}^n \epsilon_i$)
    \end{itemize}
\end{defn}

\begin{defn*}
    \textbf{(Compactness for subspace)}  Let $Y\subset X$. $Y$ is compact (i.e. every covering of $Y$ by open sets in $Y$ has a finite subcover) if and only if every covering of $Y$ by sets open in $X$ contains a finite subcollection covering $Y$
    \begin{itemize}
        \item \underline{(theorem)} Closed subspace of a compact space is compact, i.e. $Y\subset X$ where $X$ is compact and $Y$ is closed, then $Y$ is compact (idea: Given covering $\calA$ of $Y$ by open sets in $X$, $\calA \cup \{X - Y\}$ emits finite subcover for $X$, and therefore for $Y$ by definition of compactness for subspace)
        \item (example) Given $[a,b]\in\R$ compact, any closed subset of $[a,b]$ is also compact.
        \item \underline{(theorem)} Compact subspace of a $T_2$ space is closed, i.e. $Y\subset X$ where $X$ is $T_2$ and $Y$ is compact, then $Y$ is closed (idea: fix $x_0\in X\setminus Y$, find $U$ s.t. $x_0\in U\subset X\setminus Y$. By $T_2$, exists disjoint open pair $U_{y},V_{y}$ separating $x_0,y$ for all $y\in Y$. By compactness of $Y$, $\{V_y\mid y\in Y\}$ emits finite cover $V =V_{y_1}\cup \cdots \cup V_{y_n}$ which are disjoint from $U=U_{y_1}\cap\cdots\cap U_{y_n}$. $X\setminus Y$ open hence $Y$ closed)
        \item (lemma) If $X$ is $T_2$ and $Y\subset X$ is compact and $x_0\not\in Y$, exists disjoint $U,V\in\calT_X$ that covers $x_0$ and $Y$, respectively
        \item (example) $(a,b] \in \R$ not compact since its not closed. (In $T_2$, closedness is a necessary condition for compactness)
    \end{itemize}
\end{defn*}

\begin{defn*}
    \textbf{($T_3$ Space)} A space $X$ is $T_3$ if
    \begin{enumerate}
        \item it is $T_1$ (singletons are closed)
        \item if $A \subset X$ is closed and $y \in A^c$, then there exists disjoint open $U,V\in\calT_X$ such that $A\subset U$ and $y\in V$
    \end{enumerate}
    \begin{itemize}
        \item \underline{(theorem)} $X$ is compact and $T_2$, then $X$ is $T_3$ (straight from previous lemma)
    \end{itemize}
\end{defn*}

\begin{lemma*}
    \textbf{The Tube Lemma} Consider $X\times Y$, where $Y$ is compact. If $N \subset X\times Y$ is open and contains the slice $\{x_0\} \times Y$, then $N$ contains some \textbf{tube} $W\times Y$ about $\{x_0\}\times Y$, where $W$ is a neighborhood of $x_0\in X$
    \begin{proof}
        Idea is try to cover the slice $\{x_0\} \times Y$ with basis elements $U\times V \in N$, which happens to cover some tube about the slice. Since $\{x_0\}\times Y \cong Y$ and $Y$ compact, then the slice is compact. Hence can cover $\{x_0\}\times Y$ with finitely basis elements $\calA = \{U_1\times V_1, \cdots, U_n \times V_n\}$ where $U_i \times V_i \in N$. Note $W = U_1\cap \cdots \cap U_n$ is open and contains $x_0$. We claim that $\calA$ actually covers not only the slice but also the tube $W\times Y$. Let $x\times y\in W\times Y$. Some $V_i \supset y$ and since $x\in \cap_j U_j$, $x\in U_i$. Hence $x\times y\in U_i \times V_i$. 
    \end{proof}
    \begin{itemize}
        \item (example) $N=\{ x\times y \mid |x| < 1/(y^2 + 1)\}$ is an open set containing $\{0\}\times \R$ but it contains no tube about the slice. (since $\R$ is not compact, tube lemma does not hold)
    \end{itemize}
\end{lemma*}

\begin{theorem*}
    \textbf{(Product of finitely many compact spaces is compact)}
    \begin{proof}
        Show $X\times Y$ compact given $X,Y$ compact and do induction from here. Let $\calA$ be any open cover for $X\times Y$. For any $x_0\in X$, $\{x_0\} \times Y \subset X\times Y$ is compact and therefore covered by finitely many open sets of $X\times Y$, specifically $\{A_1, \cdots, A_m\} \in \calA$. Let $N = \{A_1,\cdots,A_m\}$. Since $N \supset \{x_0\} \times Y$ and $N$ is open, there exists tube $W\times Y$ where $W$ is a neighborhood of $x_0$. For each $x\in X$, we can find such neighborhood $W_x$ of $x$ such that the tube $W_x\times Y$ is covered by finitely many elements of $\calA$. Since $X$ is compact, there is a finite covering $\{W_1, \cdots, W_k\}$ of $X$. Union of the tubes covers $X\times Y$, i.e. $\cup_{j=1}^k W_j \times Y = X\times Y$. Since each tube covered by finitely many elements of $\calA$ and there is finitely many tubes, we can cover $X\times Y$ with finitely many elements of $\calA$
    \end{proof}
\end{theorem*}

\begin{defn*}
    \textbf{(Finite intersection property - FIP)} A collection $\calC \subset\calP(X)$ is said to have \textbf{finite intersection property} if intersection of every finite subcollection is nonempty
    \[
        \{C_1,\cdots,C_n\} \subset \calC \quad \Rightarrow \quad \bigcap_{i=1}^n C_i \neq \emptyset
    \]
\end{defn*}

\begin{theorem*}
    \textbf{(define compactness using finite intersection property)} $X$ is compact if and only if every collection of closed sets having the finite intersection property has nonempty intersection
    \[
        X \text{ compact } 
        \quad \iff \quad
        \text{$\forall \calC$ of closed sets in $X$ with FIP $\Rightarrow$ $\cap_{C\in\calC} C \neq \emptyset$}
    \]
    \begin{proof}
        Given $\calA \subset\calP(X)$, let $\calC = \{X - A \mid A\in \calA\}$. Then the following holds
        \begin{enumerate}
            \item $\calA$ is a collection of open sets if and only if $\calC$ is a collection of closed sets
            \item $\calA$ covers $X$ if and only if intersection $\textstyle \cap_{C\in \calC} C$ of all elements of $\calC$ is empty (by DeMorgan's Law)
            \item $\{A_1, \cdots, A_n\} \subset \calA$ covers $X$ if and only if intersection of corresponding elements $C_i = X-A_i$ of $\calC$ is empty
        \end{enumerate}
        Idea is to take contraposiitive of normal definition of compactness: Given any $\calA$ of open sets, if no finite subcollection of $\calA$ covers $X$, then $\calA$ does not cover $X$. Let $\calC$ be defined as above, apply above 3 points. We get: Given any $\calC$ of closed sets (1), if every finite collection of elements of $\calC$ is nonempty (3), then the intersection of all the elements of $\calC$ is nonempty (2)
    \end{proof}
    \begin{itemize}
        \item (example) nested sequence of closed sets $C_1 \supset C_2 \supset \cdots$, where each $C_n \neq \emptyset$, then $\calC = \{C_n\}_{n\in \Z_+}$ has finite intersection property and the intersection $\cap_{n\in\Z_+} C_n \neq \emptyset$
    \end{itemize}
\end{theorem*}


\subsubsection{\linkbook{182}{27 Compact Subspaces of the Real Line}}


\begin{theorem*}
    \textbf{(Compactness for Ordered Set with Least Upper Bound Property)} Let $X$ be a simply ordered set having least upper bound property. In order topology, each closed interval in $X$ is compact.
\end{theorem*}

\begin{theorem*}
    \textbf{(Characterize Compact Subspace of $\R^n$)}
    \begin{enumerate}
        \item ($\R$) every closed interval in $\R$ is compact (follows from previous theorem)
        \begin{itemize}
            \item $[0,1] \in\R$ is compact
        \end{itemize}
        \item ($\R^n$) $A\subset \R^n$ is compact if and only if it is closed and bounded in euclidean $d_2$ or square $d_{\infty}$ metric
        \begin{itemize}
            \item (note) compact sets in metric space is \textbf{not} equivalent to the set of closed and bounded sets. boundedness depends on metric $d$ whereas compactness is purely a topological property 
            \item Unit sphere and \underline{closed} unit ball $S^{n-1}, B^n \subset \R^n$ are compact (closed and bounded)
            \item $A=\{x\times 1/x \mid 0 < x \leq 1 \} \subset \R^2$ is not compact (closed but not bounded)
            \item $S=\{x\times \sin(1/x)\}$ is not compact (bounded but not closed since does not contain limit points $S'=\{0\}\times [-1,1]$)
        \end{itemize}
    \end{enumerate}
    \begin{proof}
        \textbf{(Part 1)} proof similar to how we proved $I=[0,1]$ is connected: 3 cases, by contradiction. \textbf{(Part 2)} Note $d_{\infty}(\bx,\by)\leq d_{2}(\bx,\by) \leq \sqrt{n} d_{\infty}(\bx,\by)$. So $A$ is bounded under $d_{2}$ if and only if it is bounded under $d_{\infty}$. So we need to consider square metric $d_{\infty}$ only. ($\Rightarrow$) Assume $A$ compact. Since $A\subset \R^n$ is compact and $\R^n$ is $T_2$, $A$ is closed. To show it is bounded, consider an open cover $\calA = \{B_{d_{\infty}}(\mathbf{0}, m) \mid m\in \Z_+\}$ for $\R^n$. Since $A\subset \R^n$ is compact, some finite collections covers $A$. So $A\subset B_{d_{\infty}}(\mathbf{0}, M)$ for some $M<\infty$. So any $\bx,\by\in A$ has $d_{\infty}(\bx,\by) \leq 2M$. Hence $A$ is bounded. ($\Leftarrow$) Idea is to find a compact superset of $A$, when combined with the fact that $A$ is closed, gives $A$ is compact. Assume for any $\bx,\by\in A$, $d_{\infty}(\bx,\by)\leq N$. Let $\bx_0\in A$ and let $b = d_{\infty}(\bx_0,\mathbf{0})$. Then for any $\bx\in A$, we have $d_{\infty}(\mathbf{0}, \bx) \leq d_{\infty}(\mathbf{0}, \bx_0) + d_{\infty}(\bx_0, \bx) \leq N + b = P$. Therefore $A\subset \textstyle \prod_{i=1}^n [-P,P]$ which is compact since $[-P,P]\subset\R$ is compact and finite product of compact set is compact. Since $A$ is closed and $A\subset [-P,P]^n$, $A$ is also compact.
    \end{proof}
\end{theorem*}

\begin{theorem*}
    \textbf{(Extreme Value Theorem)} Let $f:X\to Y$ be continuous, $Y$ is in order topology. If $X$ is compact, then $f$ attains its maximum and minimum, i.e. $\exists c,d\in X$ such that $f(c) = \textstyle\inf_{x\in X} f(x)$ and $f(d) =\textstyle\sup_{x\in X} f(x)$
    \begin{proof}
        Now show $f$ attains its maximum, proof for minimum is similar.
        ($\Rightarrow$) Since $f$ continuous and $X$ compact, then $A = f(X)$ is compact. Enough to show that $A$ contains its largest element $M$, and since $M\in A$, $M=f(d)$ for some $d\in X$. By contradiction assume $A$ has no largest element, the $\calA = \{(-\infty, a) \mid a\in A \}$ forms an open covering of $A$. Since $A$ compact, $\{(-\infty, a_1),\cdots,(-\infty, a_n)\}$ covers $A$. Let $a = \max \{a_1, \cdots, a_n\}$. Note $a\not\in (-\infty, a_i)$ for all $1\leq i \leq n$. Since $a\in A$, $a$ is not covered by $\calA$ which is a contradiction. (Alternatively, simply use the fact that $f(X)$ is a compact subspace of $T_2$ space $Y$ so closed, and therefore $\textstyle\sup_{x\in X} f(x) \in f(X)$)
    \end{proof}
\end{theorem*}
 
\begin{defn*}
    \textbf{(point-set distance)} Let $(X,d)$ be metric, let $A\subset X$ be nonempty. For each $x\in X$, define \textbf{distance from $x$ to $A$} by $d(x,A) = \inf \{d(x,a)\mid a\in A\}$. 
    \begin{itemize}
        \item (lemma) Fixing $A$, $d: X\to \R \; x \mapsto d(x,A)$ is a continuous function 
    \end{itemize}
\end{defn*}

\begin{defn*}
    \textbf{(diameter)} For metric space $(X,d)$ and $A\subset X$,  $diam(A,d) = \sup \{d(a_1,a_2) \mid a_1,a_2 \in A\}$
\end{defn*}

\begin{lemma*}
    \textbf{(Lebesgue Number Lemma)} Let $\calA$ be open covering of metric space $(X,d)$. If $X$ is compact, there is a $\delta > 0$ such that for each subset of $X$ having diameter less than $\delta$, there exists an element of $\calA$ containing it. The number $\delta$ is a \textbf{Lebesgue number} for the covering $\calA$. In other words, 
    \[
        X \text{ compact } \quad \Rightarrow \quad
        \exists \delta > 0 \;\; \forall B\subset \calP(X) \;\; ( diam(B,d) < \delta  \Rightarrow  \exists A\subset \calA \;  B\subset A )
    \]
    \begin{proof}
        Let $\calA$ be a cover for $X$. If $X \in \calA$, then $\delta>0$ could be any since $X$ contains any subsets in $X$. Now assume $X\not\in \calA$. Choose finite $\{A_1, \cdots, A_n\} \subset \calA$ covering $X$. For each $i$, let $C_i = X \setminus A_i$. Define $f:X\to\R$ by 
        \[
            f(x) = \frac{1}{n} \sum_{i=1}^n d(x, C_i)    
        \]
        Key idea is $f(x) \geq 0$. Consider any $x\in X$, pick $\epsilon$ such that $B_{\epsilon}(x)\subset A_i$, then $d(x,C_i) \geq \epsilon$, so then $f(x) \geq \epsilon /n$. Note $f$ is continuous and by extreme value theorem attains its minimum $\delta$. Now we show $\delta$ is the Lebesgue number for $\calA$. Let $B\subset X$ having diameter less than $\delta$, let $x_0\in B$ be arbitrary. Note $B\subset B_{\delta}(x_0)$. By definition of $f$ as the average distance to all $C_i$s, since $f(x_0) \geq \delta$, we have $d(x_0, C_m) \geq \delta$ for some $C_m$. Therefore $B\subset X \setminus C_m = A_m \subset \calA$.
    \end{proof}
\end{lemma*}


\begin{defn*}
    \textbf{(Uniform Continuity)} Given $(X,d_X)$ and $(Y,d_Y)$ metric, then $f:X\to Y$ \textbf{uniformly continuous} if
    \[
        \forall \epsilon > 0 \; \exists \delta > 0 \; \forall x_0,x_1\in X \qquad
        \left(
            d_X(x_0, x_1) < \delta \Rightarrow d_Y(f(x_0), f(x_1)) < \epsilon
        \right)
    \]
    \begin{itemize}
        \item (note) Idea is one $\delta > 0$ works for all $x,y\in X$ nearby such that $f(x),f(y)\in Y$ are also nearby
        \item \underline{(theorem)} Given continuous $f:X\to Y$, where $X,Y$ are metric and $X$ is compact, then $f$ is uniformly continuous
    \end{itemize}
\end{defn*}


\begin{defn*}
    \textbf{(continuous function on compact space is uniformly continuous)}
    \begin{proof}
        Let $\epsilon >0$. take open covering of $Y$ by balls $B(y, \epsilon/2)$. Let $\calA = \{ f^{-1}(B(y, \epsilon/2)) \mid y\in Y \}$ be open covering for $X$. Let $\delta$ be Lebesgue Number for $\calA$ on compact $X$. Let $x_0, x_1\in X$ such that $d_X(x_0, x_1) \leq \delta$. Then $diam(\{x_0, x_1\}) \leq \delta$, and by Lebesgue Number Lemma, $\exists A \in\calA$ such that $\{x_0, x_1\} \in A = f^{-1}(B(y, \epsilon/2))$ for some $y\in Y$. Therefore, $d_Y(f(x_0), f(x_1)) < \epsilon$
    \end{proof}
\end{defn*}


\subsubsection{\linkbook{188}{28 Limit Point Compactness}}
\newpage



\section{Countability and Separation Axioms}


\newpage
\section{The Tychonoff Theorem}
\subsubsection{\linkbook{240}{37 The Tychnoff Theorem}}

\begin{lemma*}
    \textbf{(Get Maximal $\calD\subset\calP(X)$ with FIP using Zorn's Lemma)} Let $X$ be a set; $\calA \subset \calP(X)$ having finite intersection property. Then there exists a collection $\calD \subset\calP(X)$ such that
    \begin{enumerate}
        \item $\calD \supset \calA$
        \item $\calD$ has finite intersection property
        \item no collection $\calD'\subset\calP(X)$ where $\calD' \supsetneq \calD$ has finite intersection property
    \end{enumerate}
    Call $\calD$ as maximal with respect to FIP
    \begin{proof}
        Let $\mathbbm{C}$ be a superset whose elements are collections of subsets of $X$, i.e. $\calA,\calD\in \mathbbm{C}$. Let 
        \[
            \mathbbm{A} = \{ \calB\subset \calP(X) \mid \calB \supset \calA \;\text{ and }\; \calB \text{ has FIP } \}  
        \]
        Use $\subsetneq$ as strict ordering on $\mathbbm{A}$. In order to apply Zorn's lemma on $(\mathbbm{A}, \subsetneq)$, we show if $\mathbbm{B} \subset \mathbbm{A}$ that is simply orderd by $\subsetneq$, then $\mathbbm{B}$ has an upper bound in $\mathbbm{A}$. Let $\calC = \cup_{\calB \in \mathbbm{B}} \calB$ and we show that $\calC$ is an element of $\mathbbm{A}$, and then it is the required upper bound for $\mathbbm{B}$. To show $\calC \in \mathbbm{A}$. We show
        \begin{enumerate}
            \item ($\calA\subset\calC$) obvious since any $\calB\in\mathbbm{B}$ has $\calA \subset \calB$
            \item ($\calC$ has FIP) Let $C_1,\cdots,C_n\subset\calC$ a finite subset, we want to show that their intersection is nonempty. Since $\calC = \cup_{\calB\in\mathbbm{B}} \calB$, $\calC_i \in\calB_i$ for each $i$. Hence, $\{\calB_1, \cdots, \calB_n\} \subset\mathbbm{B}$. Since $\mathbbm{B}$ is assumed to be simply ordered, there exists $\calB_k$ such that $\calB_i \subset\calB_k$ for all $i\neq k$. Therefore $\{C_1, \cdots, C_n\}\subset \calB_k$. Since $\calB_k$ has FIP, $\cap_{i=1}^n C_i \neq \emptyset$ as desired.
        \end{enumerate}
        Hence the upper bound $\calC$ is inside $\mathbbm{A}$
    \end{proof}
\end{lemma*}

\begin{lemma*}
    \textbf{(Intersection on Maximal $\calD$ w.r.t FIP)} Let $X$ be a set and $\calD$ be a collection of subsets of $X$ that is maximal with respect to the finite intersection property. Then,
    \begin{enumerate}
        \item Any finite intersection of elements of $\calD$ is an element of $\calD$
        \item If $A\subset X$ that intersects every element of $\calD$, then $A$ is an element of $\calD$
    \end{enumerate}
    \begin{proof}
        \textbf{(Part 1)} Let $B$ be intersection of finitely many elements of $\calD$. Let $\calE = \calD \cup \{B\}$. To prove $B\in \calD$, we show $\calE$ has FIP so by maximality of $\calD$, $\calE=\calD$ and $B\in \calD$. Now take finitely many elements of $\calE$, if none of them is $B$, then their intersection is nonempty since they are a subset of $\calD$ which has FIP. Now if one of finitely many elements of $\calE$ is $B$, then the intersection $D_1 \cap \cdots \cap D_m \cap B$ is nonempty since $B$ is the intersection of finitely many elements of $\calD$. \textbf{(Part 2)} Let $X\subset X$ s.t. $A\cap D \neq \emptyset$ for all $D\in \calD$. Similar to idea of part 1, define $\calE = \calD \cup \{A\}$. We show that $\calE$ has FIP. Now take finitely many elements of $\calE$. If none of them is $A$, then they are nonempty by FIP of $\calD$. Otherwise, $D_1\cap \cdots \cap D_m \cap A$ is nonempty since $A\cap D_j$ for all $1\leq j\leq m$ by assumption.
    \end{proof}
\end{lemma*}

\begin{theorem*}
    \textbf{(Tychonoff Theorem)} Arbitrary product of compact spaces is compact in product topology
    \begin{proof}
        Let $X = \textstyle \prod_{\alpha\in J} X_{\alpha}$ where each space $X_{\alpha}$ is compact. Let $\calA\subset\calP(X)$ be arbitrary and have FIP, now we show that $\cap_{A\in\calA} A$ is nonempty, thereby proving that $X$ is compact. We apply lemma 1 to find a maximal collection $\calD$ such that $\calA\in\calD$ and $\calD$ has FIP. Now we show that $\cap_{D\in\calD} \overline{D}$ is nonempty. Let $\pi_{\alpha}: X\to X_{\alpha}$ be projection map. For each $\alpha$, consider $\{\pi_{\alpha}(D) \mid D\in\calD \} \subset X_{\alpha}$ which has FIP since $\calD$ does. By compactness, we can chooose $x_{\alpha} \in \cap_{D\in\calD} \overline{\pi_{\alpha}(D)}$, which is possible by FIP. Let $\bx = (x_{\alpha})_{\alpha\in J} \in X$. To complete the proof, we show that $\bx \in \cap_{D\in\calD} \overline{D}$. Consider any coordinate index $\beta$ and any $D\in\calD$, let $U_{\beta} \subset X_{\beta}$ be a neighborhood of $x_{\beta}$. Since $x_{\beta} \in \overline{\pi_{\beta}(D)}$, $U_{\beta}$ intersects $\pi_{\beta}(D)$. Let $\pi_{\beta}(\by) \in U_{\beta} \cap \pi_{\beta}(D)$ for some $\by \in D$. Hence $\by \in \pi_{-1}(U_{\beta}) \cap D$. Since $D$ is arbitrary, by lemma 2.2, $\pi^{-1}(U_{\beta}) \in\calD$. Since $\beta$ arbitrary, this holds for all $\beta\in J$. Then any basis element containing $\bx$ as finite intersection of subbasis $\pi^{-1}(U_{\beta}) \in\calD$, then by lemma 2.1, is also in $\calD$. Now use $\calD$ FIP again, $B\cap D \neq \emptyset$ where $D\in\calD$ and $B\in\calD$ is any basic element containing $\bx$. $\bx \in \cap_{D\in\calD} \overline{D}$
    \end{proof}
\end{theorem*}



\end{document}
