\documentclass[11pt]{article}
\input{"../../preamble"}
\begin{document}

\section{Continuity}

\begin{defn}
  \label{limit defn of continuity}
  \textbf{Limit definition of continuity} Let $f:\R^n \to\R^m$ with $c\in\R^n$ and $L\in\R^m$. we say
  \[
    \lim_{x\to c} f(x) = L
  \]
  if for every $\epsilon > 0$ there exists a $\delta > 0$ such that whenever $0< || x-c || < \delta$ then $|| f(x)-L || < \epsilon$
  \begin{rem}
    A limit exists iff the limit is the same regardless of the path taken to get to $c$. To prove that limit does not exists we simply take an arbitrary path and prove that limit converges to different points. To prove that limit is convergent, we use the Squeeze theorem.
  \end{rem}
\end{defn}


\begin{theorem}
  \label{Multivariable Squeeze Theorem}
  \textbf{Multivariable Squeeze Theorem} Let $f,g,h: \R^n \to \R$ be functions and $c\in\R^n$. Assume that in some neighborhood of $c$, such that $f(x)\leq g(x)\leq h(x)$ for all $x$ in that neighborhood. If
  \[
    \lim_{x\to c} f(x) = \lim_{x\to c} h(x) = L \text{   then   }  \lim_{x\to c}g(x) = L
  \]
  \begin{rem}
    \[
      -| f(x) | \leq f(x) \leq |f(x)| \text{ then } |f(x)|\to 0 \Rightarrow f(x)\to 0
    \]
  \end{rem}
\end{theorem}

\begin{theorem}
  \label{Sequence definition of continuity}
  \textbf{Sequence Definition of continuity: maps convergent sequence to convergent sequence} A function $f:\R^n\to \R^m$ is continuous if and only if whenever $(a_n)_{n=1}^{\infty} \to a$ is a convergent sequence in $\R^n$, then $(f(a_n))_{n=1}^{\infty} \to f(a)$ is a convergent sequence in $\R^m$
\end{theorem}

\begin{theorem}
  \label{Topological definition of continuity}
  \textbf{Topological definition of continuity} A function $f:\R^n \to \R^m$ is continuous if and only if whenever $U\subseteq \R^m$ is an open set, then $f^{-1}(U) \subseteq \R^n$ is also an open set.
  \begin{proof}
    $ $\\
    $\Rightarrow$ Let $x\in f^{-1}(U)$, then $f(x)\in U$. Since $U$ open, then $B_{\epsilon}(f(x))\in U$. Since $f$ continuous, let $\delta$ for the correspondinfg $\epsilon$. We claim that $B_{\delta}(x)\in U$. We prove this by taking a point $y\in B_{\delta}(x)$, hence $||y-x|| < \delta$. Then by continuity $||f(y)-f(x)|| < \epsilon$. Then $f(y)\in B_{\epsilon}(f(x))\subseteq U$. Then by definition of inverse functions $y\in f^{-1}(U)$. $B_{\delta}(x)\in U$ is true and thereore $f^{-1}(U)$ is open.\\
    $\Leftarrow$ Assume preimage of an open set is open for which we show $f$ is continuous. Let $\epsilon > 0$ be given. Let $U = B_{\epsilon}(f(x))$ then $x\in f^{-1}(U)$. Since open set, exists $\delta > 0$ such that $B_{\delta}(x) \subseteq f^{-1}(U)$. Let $y\in B_{\delta}(x)$ then $|| y - x|| < \delta$ Since $f(y) \in f(B_{\delta}(x)) \subseteq U = B_{\epsilon}$. Then $||f(y)-f(x)|| < \epsilon$ as required.
  \end{proof}
  \begin{rem}
    We can use topological definition to prove a set is open by finding a continuous function that transforms the set into an open set.
  \end{rem}
\end{theorem}

\begin{theorem}
  If $f:\R^n \to \R^m$ is continuous at $c$ and $g:\R^m \to \R^k$ is continuous at $f(c)$, then $g\circ f: \R^n \to \R^k$ is continuous at $c$.
\end{theorem}

\end{document}
