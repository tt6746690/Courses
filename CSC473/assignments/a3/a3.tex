\documentclass[11pt]{article}
\input{/Users/markwang/.preamble}

\geometry{margin=1in}

\begin{document}


\renewcommand{\E}[1]{\mathsf{E}\left\{#1\right\}}
\newcommand{\V}[1]{\mathsf{V}\left\{#1\right\}}
\renewcommand{\P}[1]{\mathsf{P}\left(#1\right)}
\newcommand{\abs}[1]{\left|\,#1\,\right|}



\begin{lemma*}
    \textbf{5: Farkas's Lemma} For any $m\times n$ matrix $A$ and $m\times 1$ vector $b$, exactly one of following is true 
    \begin{enumerate}
        \item There exists $x\in \R^n$ such that $x\geq 0$ and $Ax = b$ 
        \item There exists $y\in \R^m$ such that $A^T y \leq 0$ and $b^T y > 0$ 
    \end{enumerate}
\end{lemma*}

\begin{lemma*}
    \textbf{7: Farkas's Lemma variant} For any $m\times n$ matrix $A$ and $m\times 1$ vector $b$, exactly one of following is true 
    \begin{enumerate}
        \item There exists $x\in \R^n$ such that $x\geq 0$ and $Ax \leq b$ 
        \item There exists $y\in \R^m$ such that $y\geq 0$, $A^T y \geq 0$ and $b^T y < 0$ 
    \end{enumerate}
\end{lemma*}

$ $\\

\textbf{Question 1: LP Theory} 

\begin{enumerate}
\item use lemma 5 prove lemma 7
    \begin{proof}
        Equivalent to proving $(\textbf{7.1}) \iff \neg (\textbf{7.2})$ 
        \begin{enumerate}
            \item $(\Rightarrow)$ Assume (\textbf{7.1}) true, 2 cases 
            \begin{enumerate}
                \item Exists $x\in \R^m$ such that $x\geq 0$ and $Ax = b$, then (\textbf{5.1}) is true and (\textbf{5.2}) is false. By contradiction, assume (\textbf{7.2}) is true, let $y' \in\R^m$ such that $y'\geq 0$, $A^T y' \geq 0$ and $b^T y' < 0$. Let $y = -y'$, therefore, there exists $y \in\R^m$ such that $A^T y \leq 0$ and $b^T y > 0$, which is saying (\textbf{5.2}) is true, hence contradiction. Therefore (\textbf{7.2}) is false  
                \item Otherwise, exists $x\in \R^m$ such that $x\geq 0$ and $Ax < b$, then (\textbf{5.1}) is false and (\textbf{5.2}) is true. Let $y\in \R^m$ be any vector satisfying (\textbf{5.2}), we claim $y > 0$. By contradiction assume $y \leq 0$, then $b < 0$ since $b^T y > 0$ (\textbf{5.2}). Use $b^T y > 0$ again, $y\neq 0$, so $y < 0$. Therefore $Ax < b < 0$. Multiply $y < 0$ to both sides, we have
                \[
                    y^t (Ax) > 0 
                    \quad \rightarrow \quad 
                    (A^T y)^T x > 0
                \]
                Contradiction since $x \geq 0$ and $A^T y \leq 0$ (\textbf{5.2}). Therefore $y> 0$ for all $y$ satisfying (\textbf{5.2}). Therefore, no $y\in \R^m$ exists satisfying (\textbf{5.2}) for which $y \leq 0$, i.e. 
                \[
                    \nexists y\in \R^m \, , \, y \leq 0 \quad A^T y\leq 0 \quad  b^T y > 0
                \]
                therefore 
                \[
                    \nexists y\in \R^m \, ,\, y\geq 0 \quad A^T y\geq 0 \quad b^T y < 0  
                \]
                which is equivalent to (\textbf{7.2}) false
            \end{enumerate}
            \item $(\Leftarrow)$
            Idea is that given (\textbf{7.2}) false, we want to construct matrices $\tilde{A}$ and $\tilde{b}$ satisfying constraints for (\textbf{7.2}) as well as (\textbf{5.2}), 
            \[
                \underset{m\times (m+n)}{\tilde{A}} = 
                \begin{pmatrix}
                    -A & -I \\ 
                \end{pmatrix}  
                \qquad 
                \underset{m\times 1}{\tilde{b}} = 
                \begin{pmatrix}
                    -b
                \end{pmatrix}  
            \]
            Therefore, $(\textbf{7.2})$ can be rewritten as,
            \[
                \nexists y\in \R^m  \, ,\, y\geq 0 \quad A^T y\geq 0 \quad b^T y < 0
                \qquad \rightarrow \qquad 
                \nexists y\in \R^m  \, ,\, \tilde{A}^T y \leq 0 \quad \tilde{b}^T y > 0
            \]
            where the latter formulation is simply saying $(\textbf{5.2})$ is false, therefore $(\textbf{5.1})$ is true, i.e. 
            \[
                \exists \tilde{x} \in\R^{m+n} \, ,\, \tilde{x} \geq 0 \quad \tilde{A}\tilde{x} = \tilde{b} 
            \]
            let $\tilde{x} = (x,x')$ where $x\in\R^n, x'\in\R^m$, note $x,x'\geq 0$, also 
            \[
                \begin{pmatrix}
                    -A & -I \\ 
                \end{pmatrix}  
                \begin{pmatrix}
                    x \\ x' \\
                \end{pmatrix}
                = 
                \begin{pmatrix}
                    -b
                \end{pmatrix}  
                \qquad \iff \qquad 
                -Ax - x' = -b 
            \]
            therefore $Ax = b-x' \leq b$ since $x' \geq 0$, therefore,
            \[
                \exists x\in\R^n \, ,\, x\geq 0 \quad Ax \leq b
            \]
            hence $(\textbf{7.1})$ is true. 
        \end{enumerate}
    \end{proof}
    \item Use lemma 7, prove for any $m\times n$  matrix $A$ and $m\times 1$ matrix $b$ such that $\{x: Ax \leq b\, ,\, x\geq 0\} \neq \emptyset$, and any $n\times 1$ vector $c$, the system of inequalities 
    \[
        c^T x \geq 1 \qquad Ax \leq b \qquad x \geq 0     
    \]
    is infeasible if and only if there exists a $y\in\R^m$, $y\geq 0$, such that $A^T y\geq c$ and $b^T y < 1$
    \begin{proof}
        Define 
        \[
            \tilde{A} =
            \begin{pmatrix}
                A \\
                -c^T \\
            \end{pmatrix}    
            \qquad 
            \tilde{b} = 
            \begin{pmatrix}
                b \\ 
                -1 \\ 
            \end{pmatrix}
        \]
        Then proving the following suffices
        \[
            \nexists x\in\R^n \, ,\, \tilde{A}x \leq \tilde{b} \quad x \geq 0
            \qquad \iff \qquad 
            \exists y \in\R^m \, ,\, y \geq 0 \quad A^T y \geq c \quad b^T y < 1
        \]
        \begin{enumerate}
            \item ($\Leftarrow$) let $\tilde{y} = \begin{pmatrix} y & 1 \\ \end{pmatrix}$, then, 
            \[
                \exists \tilde{y} \in\R^{m+1} \, ,\, \tilde{y} \geq 0 \quad \tilde{A}^T \tilde{y} \geq 0 \quad \tilde{b}^T \tilde{y} < 0
            \]
            lhs follows by Farkas's lemma.
            \item $(\Rightarrow)$ Assume lhs true, by Farkas's lemma, 
            \[
                \exists \tilde{y} \in\R^{m+1} \, ,\, \tilde{y} \geq 0 \quad \tilde{A}^T \tilde{y} \geq 0 \quad \tilde{b}^T \tilde{y} < 0 
            \]
            Let $\tilde{y} = \begin{pmatrix} y & a \end{pmatrix}$ satisfying the condition above. If $a\neq 0$, let $\tilde{y}' = \rfrac{1}{a} \tilde{y} = \begin{pmatrix} \rfrac{1}{a} y & 1 \end{pmatrix}$. Since $a > 0$,
            \[
                \tilde{A} \tilde{y}' = \frac{1}{a} \tilde{A}\tilde{y} \geq 0 
                \qquad 
                \tilde{b}^T \tilde{y}' = \frac{1}{a} \tilde{b}^T \tilde{y} < 0 
            \]
            Since $\tilde{A} \tilde{y}' = A^T y  - c$ and $\tilde{b}^T \tilde{y}' = b^T y - 1$, then $A^T y \geq c$ and $b^T y < 1$. This proves rhs true. Now we consider the case where $a=0$. We claim $a=0$ is not possible. Since if $a=0$, then we note $y$ in $\tilde{y}$ statifies $ A^T y \geq 0$ and $b^T y \leq 0$
            \[
                \exists y\in\R^m \, ,\, y\geq 0 \quad A^T y \geq 0 \quad b^T y \leq 0
                \qquad \overset{Farkas}{\rightarrow} \qquad 
                \nexists x\in \R^n \, ,\, x \geq 0 \quad Ax \leq b
            \]
            which contradicts the assumption that $\{x: Ax\leq 0 \quad x\geq 0\} \neq \emptyset$
        \end{enumerate}
    \end{proof}
\end{enumerate}


$ $\\
\textbf{Question 2: Primal-Dual Algorithm} Given directed $G = (V,E)$ where $w_e > 0$. Given $s,t \in V$ and a flow from $s$ to $t$. weight of flow $w(f) = \textstyle\sum_{e\in E} w_e f_e$. Objective to send one unit of flow from $s$ to $t$ so that weight of flow minimized 

\begin{enumerate}
\item Primal 
\begin{align*}
    \min \quad & \sum_{e\in E} w_e f_e \\ 
    st \quad  & \\
    \forall u\in V \setminus \{s,t\} \quad & \sum_{(v,u)\in E} f_{vu} - \sum_{(u,v)\in E} f_{uv} = 0 \\
    & \sum_{(v,s)\in E} f_{vs} - \sum_{(s,v)\in E} f_{sv} = -1 \\ 
    & \sum_{(v,t)\in E} f_{vt} - \sum_{(t,v)\in E} f_{tv} = 1 \\ 
    \forall e\in E \quad & f_e \geq 0
\end{align*}
Dual is given by 
\begin{align*}
    \max \quad & y_t - y_s \\
    st \quad & \\
    \forall (u,v)\in E \quad &  y_v - y_u \leq w_{uv}
\end{align*}
\item Given feasible $y$, characterize when there exists feasible flow $f$ such that $y,f$ satisfies complementary slackness conditions \\ 
Feasible $f,y$ are optimal if following complementary slackness condition holds
\[
    \forall (u,v)\in E \quad (w_{uv} + y_u - y_v)f_{uv} = 0
\]
Therefore, \textit{ $f,y$ optimal if and only if there exists a flow $f$ such that $f_{uv} \neq 0$ for any $w_{uv}+y_u-y_v = 0$, i.e. flow only operates on tight edges  }

\item Give a primal-dual algorithm which finds a pair of feasible solutions primal and dual linear programs \\

\begin{algorithm}[H]
    $y\leftarrow 0$ \\
    $S \leftarrow \{s\}$ \\ 
    \While{$t \not\in S$}{
        $\delta \leftarrow \min \{ w_{uv} + y_u - y_v : u\in S \,\land\, v\not\in S \,\land\, (u,v) \in E \} $ \\
        $Q = \{ v\not\in S: (u,v)\in E \,\land\, u\in S \,\land\, w_{uv} + (y_u - \delta) - y_v = 0 \}$ \\ 
        \For{$u\in S$}{
            $y_u \leftarrow y_u - \delta$ \\ 
        }
        $S \leftarrow S \cup Q$ \\
    }
    $ p \leftarrow$ a path from $s$ to $t$ consists of edges in $S$ \\
    $ f_e \leftarrow 
    \begin{cases}
        1 &  e \in p \\
        0 & otherwise \\ 
    \end{cases} $\\ 
    \Return{ $(f,y)$}
\end{algorithm} 

\newpage

We will prove the following points 
\begin{enumerate}
\item $y$ remain feasible in each iteration 
\begin{proof}
    For all $u\in V$, $y_u = 0$ before loop starts and hence is a feasible solution to the dual. At any iteration, assume $y$ is feasible we prove that $y_u' = y_u - \delta$ ($u\in S$) $y_u' = 0$ ($u\not\in S$) remains feasible. For any $(u,v)\in E$ such that $u,v\in S$, we have 
    \[
        w_{uv} + y_u' - y_v'  = w_{uv} + y_u - \delta - y_v + \delta = w_{uv} + y_u - y_v \geq 0 
    \]
    For any $(u,v)\in E$ such that $u\in S$ and $v\not\in S$, then 
    \begin{align*}
        w_{uv} + y_u' - y_v' = 
        w_{uv} + y_u - \delta - y_v 
        &= (w_{uv} + y_u  - y_v)  \\
        & \quad - \min \{ w_{uv} + y_u - y_v : u\in S \,\land\, v\not\in S \,\land\, (u,v) \in E \} \geq 0 \\ 
    \end{align*}
    For any $(u,v)\in E$ such that $u\not\in S$, $v\in S$, then 
    \[
        w_{uv} + y_u' - y_v' = w_{uv} + y_u - y_v + \delta \geq \delta \geq 0
    \]
    For any $(u,v)\in E$ such that $u,v\not\in S$, $y_u' = y_v' = 0$ satisfies dual constraints. 
\end{proof}


\item complementary slackness is maintained, i.e. \textit{for all $e\in S$, $w_{uv} + y_u - y_v = 0$}
\begin{proof}
    Assume $S$ consists of only 'tight' edges, we now prove that edges in $S$ remains tight after updating $y_u$. Let $(u,v)\in S$, then $w_{uv} + y_u - y_v = 0$, updating $y_u' = y_u - \delta$ and $y_v' = y_v = \delta$, we have 
    \[
        w_{uv} + y_v' - y_v' = w_{uv} - \delta - (y_v - \delta) = w_{uv} + y_u - y_v = 0
    \]
    Also note $Q$ consists of edges that becomes 'tight' after updating $u\in S$ by $-\delta$. Therefore $S \cup A$ contains 'tight' edges only
\end{proof}
\item dual objective strictly increases in each iteration 
\begin{proof}
    Inside the loop $y_t' = y_t = 0$, and $y_s' = y_s - \delta$. Therefore 
    \[
        y_t' - y_s' = y_t - y_s + \delta
    \]
    the dual increases by $\delta$ during each iteration
\end{proof}
\item algorithm terminates
\begin{proof}
    The loop terminates because in each iteration $|S|$ increases by $|Q| \geq 1$ since we always pick a $\delta$ that makes at least one edge $(u,v)\in E$, $u\in S$, $v\not\in S$ 'tight'. The loop body runs in polynomial time, therefore the algorithm terminates in polynomial time 
\end{proof}
\item the algorithm is correct 
\begin{proof}
    The returned primal solution is a feasible solution in that it represent a flow that pushes one unit from $s$ to $t$ while satisfying conservation constraint. The corresponding dual solution $y$ is also feasible and $f,y$ satisfies the complementary slackness constraint and hence $f,y$ are optimal solution. Note $f$ is an integral solution 
\end{proof}
\item the algorithm has runtime of $O(nm)$
\begin{proof}
    The loop runs at most $n$ iterations. At each iteration of the for loop, finding $\delta$ and $Q$ requires $O(m)$ and updating $y$ requires $O(n)$, hence $O(m+n)$ for each iteration. In total, the algorithm termiantes in $O(n(m+n))$ 
\end{proof}


\end{enumerate}
\end{enumerate}




$ $\\
\textbf{Question 3: Simple Randomized Approximation Algorithm} Given directed $G=(V,E)$ with $w_e \geq 0$ for all $e\in E$. Goal is to partition vertices into $k$ disjoint sets $V_1,\cdots, V_k$ such that the objective is maximized 
\[
    \sum_{i=1}^{k-1} \sum_{j=i+1}^k  \sum_{u,v: (u,v)\in E,u\in V_i, v\in V_j} w_{uv}    
\]
i.e. total weights of edges that go from $V_i$ from $V_j$ such that $i<j$. Give a randomized polynomial approximation algorithm that achieves approximation ratio of $\textstyle\frac{k-1}{2k}$ in expectation. Justify approx ratio and running time 


\begin{proof}
    The algorithm works by assigning each $v\in V$ to $V_1,\cdots, V_k$ with uniform probability, i.e. $\textstyle\frac{1}{k}$. We prove that such assignment guarantees a $\textstyle\frac{k-1}{2k}$-approximation algorithm. Let $X:V\rightarrow \{1,\cdots, k\}$ be a random variable representing the random assignment of $v$ to $V_1,\cdots ,V_k$. Note 
    \[
        \P{X_u < X_v} = \sum_{i=1}^{k-1}\sum_{j=i+1}^k \P{X_u=i,X_v=j} =\sum_{i=1}^{k-1}\sum_{j=i+1}^k \frac{1}{k^2} = \frac{k-1}{2k}    
    \]
    Let $Y_{uv} = \mathbbm{1}_{u\in V_i, v\in V_j, i < j}$. Note 
    \[  
        \E{Y_{uv}} = \P{X_u < X_v} = \frac{k-1}{2k}  
    \]
    We then derive expectation of objective as follows 
    \begin{align*}
        \E{\sum_{(u,v)\in E} Y_{uv}w_{uv}}
        = \sum_{(u,v)\in E} w_{uv} \E{Y_{uv}}
        = \frac{k-1}{2k} \sum_{(u,v)\in E} w_{uv} 
        \geq \frac{k-1}{2k} \, opt
    \end{align*}
    where last inequality follow from the fact that optimal assignment is upper bounded by the sum of weight of all edges. The upper bound on expectation of objective is $opt$ by definition. Therefore, the algorithm is $\textstyle \frac{k-1}{2k}$-approximation algorithm. The algorithm has polynomial complexity $O(m)$, where $m=|E|$ since the algorithm simply samples uniformly from $\{ 1,\cdots, k\}$. 
\end{proof}







\end{document}
