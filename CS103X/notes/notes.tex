\documentclass[11pt]{article}
\input{/Users/markwang/.preamble}




\begin{document}


\section*{Asymptotic Notation}

Properties
\begin{enumerate}
  \item For $u,v\in\R$, if $u<v$ then $n^u = O(n^v)$ (A bigger power swallows a smaller one)
  \item If $f(n)$ is a degree-d polynomial in $n$ then $f(n)=O(n^d)$ If the coefficient of $n^d$ is nonzero then $f(n) = \Theta (n^d)$
  \item For any real $b>1$ and $p$. $n^p = O(b^n)$ (An exponential swallows a power)
  \item For any real $q>0$ and $p$. $(\ln n)^p = O(n^q)$ (A power swallows logarithm)
  \item For any real $a,b>0$, $\log_a n = \Theta(log_b n)$ Theefore we can write $O(\log n)$ without writing base of the logarithm. (Asymptotic notation swallows bases of logirithm)
\end{enumerate}

$ $\\
Propositions
\begin{enumerate}
  \item If $f(n) = O(g(n))$ and $p\in\N$ is a constant then $p\cdot f(n) = O(g(n))$
  \item If $f(n) = O(h(n))$ and $g(n) = O(w(n))$ then $f(n) + g(n) = O(max(| h(n) |, | w(n) |))$
  \item If $f(n) = O(h(n))$ and $g(n) = O(w(n))$ then $f(n)\cdot g(n) = O(h(n)\cdot w(n))$
\end{enumerate}


\section*{Graphs}

A graph $G$ is an ordered pair $(V,E)$, where $V$ is a set and $E$ is a set of two-elements subsets of $V$. That is,
\[
  E \subseteq \left\{ \left\{ x, y \right\}: x,y\in V, x\neq y\right\}
\]

Elements of $V$ are vertices (nodes) of the graph and elements of $E$ are the edges. If $e = \{ x,y \} \in E$ we say that $x$ and $y$ are adjacent in the graph $G$, that $y$ is a neighbor of $x$ in $G$ and vice versa, and that the edge $e$ is incident to $x$ and $y$

\subsection*{Common Graphs}

\begin{itemize}
  \item \textbf{Cliques} A graph on $n$ vertices where every pair of vertices is connected is called a clique and is denoted by $K_n$. Formally, $K_n = (V,E)$, where $V = \{ 1,2,\dots , n\}$ and $E = \{ \{ i, j\}: 1\leq i < j \leq n \}$. The number of edges in $K_n$ is $\binom{n}{2}$
  \item \textbf{Paths} A path on $n$ vertices, denoted by $P_n$, is the graph $P_n = (V, E)$, where $V= \left\{ 1,2, \dots n \right\}$ and $E = \left\{ \left\{ i, i+1 \right\} : 1 \leq i \leq n-1 \right\}$. The number of edges in $P_n$ is $n-1$. The vertices $1$ and $n$ are called the endpoints of $P_n$
  \item \textbf{Cycles} A cycle on $n\geq 3$ vertices is the graph $C_n = (V, E)$, where $V = \{ 1,2, \dots , n\}$ and $E = \left\{ \left\{ i, i+1 \right\} : 1 \leq i \leq n-1 \right\} \cup \{ \{ 1, n\}\}$. The number of edges in $C_n$ is $n$
\end{itemize}


\subsection*{Important concepts}

\begin{theorem*}
  \textbf{Graph Isomorphism} Two graphs $G=(V,E)$ annd $G' = (V', E')$ are said to be isomophic if there exists a bijection $f: V\to V'$ such that
  \[
    \{ x,y \} \in E \iff \{ f(x), f(y)\} \in E'
  \]
  In this case we write $G \equiv G'$ and the function $f$ is called an isomorphism of $G$ and $G'$.
\end{theorem*}

$ $\\
\textbf{Size} $  $ The number of edges of a graph is called its size. The size of an $n$-vertex graph is at most $\binom{n}{2}$, achieved by $n-clique$

$ $\\
\textbf{Degree} $  $ The degree (valency) of a vertex $v$ in a graph $G = (V,E)$, denoted by $d_G (v)$ is the number of neighbors of $v$ in $G$. More formally, this degree is
\[
  d_G(v) = | \{ u\in V: \{ v, u\}\in E\} |
\]
A graph in which every vertex has degree $k$ is called $k$-regular and a graph is said to be regular if it is $k$-regular for some $k$

\begin{proposition*}
  \textbf{Handshake lemma}
  The number of odd-degree vertices in a graph is even
  \begin{proof}
    For a graph $G = (V,E)$, consider the sum of the degree of its vertices,
    \[
      s = \sum_{v\in V} d_G (v)
    \]
  \end{proof}
  Since the sume counts every edge twice. Thus $s = 2|E|$. Subtracting from $s$ the degrees of even degree vertices of $G$, we see that the resulting quantity is the sum of the degrees of odd-degree vertices and is still even. Notice that this only works when subtracting even degree vertices since they themselves are even and one has no knowledge of the degrees of neighbors of an even-degree node.
\end{proposition*}


\subsection*{Subgraphs and Connectivity}

\textbf{Subgraph} Given a graph $G = (V,E)$,
\begin{itemize}
  \item A graph $G' = (V', E')$ is said to be a subgraph of $G$ if and only if $V'\subseteq V$ and $E'\subseteq E$
  \item A graph $G' = (V', E')$ is said to be an induced subgraph of $G$ if and only if $V'\subseteq V$ and $E'= \{ \{ u,v\} \in E: u,v\in V'\}$
\end{itemize}
\begin{rem}
  Two vertices $u,v$ of $G$ are \textbf{connected} if and only if there is a path in $G$ with endpoints in $u$ and $v$. A graph $G$ is a whole is said to be connected if and only if every pair of vertices in $G$ is connected. \\
  A subgraph $G'$ of $G$ is \textbf{connected component} of $G$ if it is connected (as a whole) and no other graph $G''$, such that $G' \subset G'' \subseteq G$, is connected. So a graph (as a whole) is connected if and only if it has a \textit{single} connected component \\
  Given $G = (V,E)$, a \textbf{walk} $W$ in $G$ is a sequence $W = (v_1, e_1, v_2, e_2, \dots, v_{n-1}, e_{n-1}, v_n)$ of vertices and edges in $G$ that are not necessarily distinct, such that $\{ v_1, v_2, \dots , v_n\} \subseteq V$ and $\{ e_1, e_2, \dots , e_n\}\in E$ and $e_i = \{ v_i, v_{i+1}\}$ for all $1\leq i\leq n-1$. A walk differs from a path in that vertices and edges cannot be repeated. The set of edges $\{ e_1, e_2, \dots , e_n \}$ covered by $W$ is denoted by $E(W)$; the set of vertices covered by $W$ is $V(W) = \{ v_1, v_2, \dots, v_n\}$
\end{rem}


\subsection*{Kinds of graphs}

A \textbf{directed} (simple, unweighted) graph $G$ is an ordered pair $(V,E)$, where $V$ is a set and $E$ is a set of ordered pairs from $V$. That is,
\[
  E\subseteq \{ (x,y): x,y\in V, x\neq y\}
\]
Here we denote \textbf{indegree} and \textbf{outdegree} as $| \{ u\in V: (u,v)\in E\} |$ and $| \{ u\in V: (v,u)\in E\} |$

$ $\\
A graph that is not simple can have \textbf{multi-edges} and \textbf{self-loops}. Multi-edges are multiple edges between the same pair of vertices. (implies that collection of edges $E$ is no longer a set but a multiset, which allows duplicates) A self-loop is an edges to and from a single vertex $v$.

$ $\\
A graph can also be \textbf{weighted}, in the sense that numerical weights are associated with edges.



\subsection*{Eulerian graphs}

A \textbf{Tour} of $G$ is a walk $T = (v_1, e_1, v_2, e_2, \dots, v_n, e_n, v_{n+1})$ in $g$ such that $T$ does not trace any edge more than once (i.e. $e_i \neq e_j$ for all $1\leq i < j \leq n$) The tour is said to be \textbf{Eularian} if, in addition, $v_{n+1} = v_1, V(T) = V$ and $E(T) = E$.

\begin{theorem*}
  A graph is Eulerian if and only if it is connected and each of its vertices has even degree
\end{theorem*}


\subsection*{Graph coloring}

A \textbf{k-coloring} of $G$ is said to be a funcion
\[
  c: V\to \{ 1,2,, \dots , k\}
\]
such that if $\{ u,v\}\in E$ then $c(v) \neq c(u)$\\\\
The smallest $k\in \N$ for which a k-coloring exists is called a
\textbf{chromatic number} of $G$. If a k-coloring of $G$ exists, the graph is said to be \textbf{k-colorable}.

\begin{proposition*}
  If the degree of every vertex in a graph is at most $k$, then the chromatic number of $G$ is at most $k+1$
\end{proposition*}


\subsection*{Bipartite graphs and matchings}

A bipartite graph is one that can be partitioned into two parts, such that eddges of the graph only go between the parts, but not inside of them. Formally, a graph $G=(V,E)$ is said to be bipartite if and only if there exists $U\subseteq V$ such that
\[
  E\subseteq \{ \{u, u'\}: u\in U \land u'\in V \setminus U\}
\]
The sets $U$ and $V\setminus U$ are called the classes of $G$. \\\\
A \textbf{complete bipartite graph} $K_{m,n}$ is a graph in which all the edges between the two classes are present. Namely, $K_{m,n} = (V,E)$, where $V = \{ 1,2, \dots , m+n\}$ and
\[
  E = \{ \{ i,j\}: 1\leq i \leq m, m+1 \leq j \leq m+n\}
\]
The number of edges in $K_n$ is $mn$.


\begin{proposition*}
  By definition of coloring, A graph is bipartite if and only if it is 2-colorable.
\end{proposition*}

\begin{proposition*}
  A graph is bipartite if and only if it contains no cycle of odd length.
\end{proposition*}

\begin{lemma*}
  Given a graph $G = (V,E)$, let $P=(v_1, v_2, \dots, v_n)$ be a shortest path between two vertices $v_1$ and $v_n$ in $G$. Then for all $1\leq i <j \leq n$, $P_i = (v_i, v_{i+1}, \dots, v_j)$ is a shortest path between $v_i$ and $v_j$
\end{lemma*}

\begin{defn*}
  Given a bipartite graph $G = (V,E)$,  a \textbf{matching} $B$ in $G$ is a set of disjoint edges. Namely, $B\subseteq E$ and $e_1 \cap e_2 = \emptyset$ for any $e_1, e_2 \in B$. A matching is said to be \textbf{perfect} if $\bigcup_{e\in B}e = V$
\end{defn*}

\begin{theorem*}
  \textbf{Hall's Theorem} A bipartite graph $W = (V,E)$ with classes $B$ and $G$ has a perfect matching if and only if $|B|=|G|$ and $|\Gamma(S)| \geq |S|$ for all $S\subseteq B$.
  \begin{rem}
    Given a subset $S$ of the vertices of $W$, let $\Gamma(S)$ be the set of vertices of $W$ adjacent to at least one of the verties of $S$
  \end{rem}
\end{theorem*}

\end{document}
