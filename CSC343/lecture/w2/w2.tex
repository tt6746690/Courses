\documentclass[11pt]{article}
\input{/Users/markwang/.preamble}
\begin{document}

\subsection{2.4 An Algebraic Query language}

\begin{defn*}
    \textbf{Set operation} 
    \begin{enumerate}
        \item \textbf{Union} $R\cup S$, the union of $R$ and $S$, is set of elements that are in $R$ or $S$ or both, element only appear once 
        \item \textbf{Intersection} $R\cap S$, the intersection of $R$ and $S$, the set of elements that are in both $R$ and $S$
        \item \textbf{Difference} $R - S$, difference of $R$ and $S$, is the set of elements that are in $R$ but not in $S$
    \end{enumerate}
    Note $R$ and $S$ must have schemas with identical set of attributes, the types (domains) for each attribute must be same. Also, $R$ and $S$ must be ordered so that the order of attributes is same for both relations
\end{defn*}


\begin{defn*}
    \textbf{subset operation}
    \begin{enumerate}
        \item \textbf{Projection} produce from a relation $R$ a new relation that has some of $R$'s columns. 
        \[
            \pi_{A_1, A_2, \cdots, A_n}(R)
        \] 
        is a relation that has only the columns for attributes $A_1, A_2, \cdots, A_n$ of $R$\\
        \item \textbf{Selection} produce from a relation $R$ a new relation with a subset of $R$'s tuples satisfying some condition $C$ that invovles the attributes of $R$
        \[
            \sigma_{C}(R)
        \]
    \end{enumerate}
\end{defn*}


\begin{defn*}
    \textbf{Renaming} applies to a relation $R$, change the name of relation to $S$ and its attribute name to $A_1, \cdots, A_n$
    \[
        \rho_{S(A_1, \cdots, A_n)}(R)
    \]
\end{defn*}

\begin{defn*}
    \textbf{Relation between operations}
    \begin{enumerate}
        \item intersection can be expressed in terms of set difference 
        \[
            R\cap S = R - (R - S)
        \]
        \item theta joins can be expressed by taking selection of a product 
        \[
            R \bowtie_C S = \sigma_C (R\times S)
        \]
        \item natural join can be expressed by starting with the product then apply selection with a condition $C$ of the form 
        \[
            R.A_1 = S.A_1 \land \cdots \land R.A_n = S.A_n
        \]
        where $A_1, \cdots, A_n$ are attributes appearing in schemas of both $R$ and $S$. Finally have to project out one copy of each of the equatedf attributes. Let $L$ be thet list of attributes in shcema of $R$ followed by those attributres in schema of $S$ that are not also in the schema of $R$
        \[
            R\bowtie S = \pi_L (\sigma_C (R\times S))
        \]
        \item union, difference, selection, projection, product, renaming forms a set where none can be written in terms of the other five
    \end{enumerate}
\end{defn*}


\begin{defn*}
    \textbf{linear notation} 
    \begin{enumerate}
        \item \textbf{assignment}
        \[
            R(A_1, \cdots, A_n) := <expr>
        \]
    \end{enumerate}
\end{defn*}


\subsection{2.5 Constraint on Relations}

\begin{defn*}
    \textbf{Relational algebra as a constriant language}
    \begin{enumerate}
        \item If $R$ is expression of relational algebra, 
        \[
            R = \emptyset \quad (R \subseteq \emptyset)
        \]
        is a constraint that says no tuples in result of $R$
        \item if $R$ and $S$ are expressions of relational algebra, then 
        \[
            R \subseteq S \quad (R - S = \emptyset)
        \]
        is a constriant that says every tuple in result of $R$ must also be in result of $S$
    \end{enumerate}
\end{defn*}


\begin{defn*}
    \textbf{Referential Integrity Constriants} asserts one value appearing in one context also appears in another, related context. In general, if we have any value $v$ (maybe be represented by >1 attribute) as the component in attribute $A$ of some tuple in relation $R$, then because of design intentions we may expect that $v$ appear in a particular component (for attribute $B$) of some tuple of another relation $S$. We can express this integrity constraint as 
    \[
        \pi_A (R) \subseteq \pi_B (S) \iff \pi_A(R) - \pi_B(S) = \emptyset
    \]
\end{defn*}


\begin{defn*}
    \textbf{Key Constraint} constraints that a set of attributes is a key for a relation (i.e. no two tuple agree on the key) 
    \[
        \rho_{MS1}(name, address, gender, birthdate)(MovieStar) \quad \rho_{MS2}(name, address, gender, birthdate)(MovieStar)
    \]
    \[
        \rho_{MS1.name=MS2.name \land MS1.address!=MS2.address}(MS1 \times MS2) = \emptyset
    \]
    represents $(name, address)$ is the key for $MovieStar$
\end{defn*}

\begin{defn*}
    \textbf{Domain constriant}
    Want to enforce a type constrinat on values of a particular attribute, i.e. integer only.
    \[
        \sigma_{gender!=F \land gender!='M'}(MovieStar) = \emptyset
    \]
\end{defn*}



\end{document}
